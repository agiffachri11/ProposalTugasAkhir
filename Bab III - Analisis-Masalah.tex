% ============================================================================================
% BAB III ANALISIS MASALAH
% Pembagian subbab tidak rigid dan dapat bervariasi. Bab ini minimal berisi analisis kebutuhan
% fungsional dan nonfungsional, analisis berbagai alternatif solusi yang dapat ditawarkan, dan
% metode pemilihan solusi yang diusulkan.
% ============================================================================================
\chapter{ANALISIS MASALAH}
\label{chap:analisis-masalah}
\section{Gambaran Umum Perusahaan}
\subsection{Profil Paragon Corp}
Paragon Corp adalah perusahaan \textit{Fast Moving Consumer Goods} (FMCG) bidang kosmetik di Indonesia yang mengemban tujuan utama menciptakan kebaikan yang lebih besar bagi masyarakat melalui inovasi. 
Perusahaan ini didukung lebih dari 10.000 karyawan yang mereka sebut paragonian yang tersebar di Indonesia hingga Malaysia. 
Perusahaan ini terus berinovasi dalam produk, program, dan cara kerja untuk menyebarkan manfaat ke seluruh lapisan masyarakat \autocite{ParagonInnovation}.

Paragon Corp memiliki visi yaitu menjadi perusahaan yang berkomitmen pada tata kelola terbaik dan perbaikan berkelanjutan agar lebih baik dari kemarin melalui produk berkualitas tinggi yang memberikan manfaat bagi paragonian, mitra, masyarakat, dan lingkungan.
Misi Paragon Corp berfokus kepada enam pilar utama yaitu:
\begin{enumerate}
    \item Mengembangkan dan mendidik paragonian yang kompeten dengan keunggulan kompetitif.
    \item Mendengarkan kebutuhan konsumen dan menciptakan produk yang melampaui ekspetasi mereka.
    \item Meningkatkan kualitas produk melalui inovasi.
    \item Bekerja sama dengan mitra bisnis demi keuntungan bersama.
    \item Berusaha keras untuk menjaga bumi secara berkelanjutan.
    \item Mendukung pengembangan generasi baru melalui pendidikan dan kesehatan untuk menciptakan sumber daya manusia yang berpengetahuan dan sehat.
    \item Memperluas jangkauan produk, layanan, dan area bisnis.
  \end{enumerate}

\subsection{Penerapan Arsitektur \textit{Enterprise} di Paragon Corp}
Pertumbuhan pada Direktorat Teknologi Informasi (TI) Paragon Corp menimbulkan banyak tantangan.
Manajemen mulai kesulitan untuk menentukan arah strategis perusahaan karena bertambahnya aplikasi dan layanan.
Kondisi ini mendorong untuk membentuk Tim \textit{Enterprise Architecture}.

Pada awal pembentukannya, Paragon Corp menjalin kerja sama dengan konsultan PwC untuk melakukan penilaian terhadap kapabilitas Arsitektur \textit{Enterprise} (AE).
Hasil penilaian menunjukkan bahwa tingkat \textit{maturity} AE di Paragon Corp berada di level 1. 
Saat ini, tim masih melakukan pengumpulan data terkait aset Teknologi Informasi (TI) dari berbagai pemangku kepentingan.
Beriringan dengan tahap ini, akan segera dijalani juga implementasi proses tata kelola AE yag terstruktur untuk memastikan pengembangan TI di Paragon Corp sesuai dengan tujuan yang ditetapkan.

\section{Analisis Kondisi Tata Kelola Arsitektur \textit{Enterprise} Saat Ini}
\subsection{Struktur dan Mekanisme Tata Kelola Arsitektur \textit{Enterprise} Saat Ini}
Tata kelola Arsitektur \textit{Enterprise} (AE) di Paragon Corp melibatkan beberapa peran yang bekerja secara lintas fungsi. 
Secara garis besar, struktur ini terdiri atas \textit{Business Architect, Enterprise Architect, Cloud Infrastructure, IT Security,} dan \textit{Engineering Manager}.
\begin{enumerate}
  \item {\textit{Business Architect}} \\
  \textit{Business Architect} bertanggung jawab dalam menginisiasi perubahan atau inisiatif melalui penyusunan dokumen \textit{User Requirement Specification} (URS), melakukan \textit{EA impact analysis}, dan melakukan \textit{architecture impact scoring} terhadap arsitektur. \\

  \item {\textit{Enterprise Architect} (EA)} \\
  \textit{Enterprise Architect} bertindak sebagai konsultan arsitektur yang melakukan pengecekan kelengkapan artefak, memberikan arahan arsitektural, dan memonitor jalannya proyek di setiap fase. \\

  \item {\textit{Cloud Infrastructure} (CI) dan \textit{IT Security}} \\
  \textit{Cloud Infrastructure} dan \textit{IT Security} bertugas dalam menyusun dan memperbarui artefak terkait infrastruktur dan keamanan, memastikan perubahan sejalan dengan standar keamanan dan kebutuhan infrastruktur yang relevan. \\

  \item {\textit{Engineering Manager} (EM)} \\
  \textit{Engineering Manager} berperan dalam mereview dokumen URS, memonitor pengembangan aplikasi, serta memastikan seluruh artefak aplikasi dan data terdokumentasi dengan baik. \\
\end{enumerate}

Alur tata kelola AE terbagi menjadi lima tahap, dengan detail sebagai berikut:
\begin{enumerate}
  \item {Tahap \textit{Ideation}} \\
  Tahap \textit{ideation} dimulai ketika muncul kebutuhan perubahan atau pengembangan sistem, biasanya diinisiasi oleh tim bisnis. 
  Inisiatif ini ditindaklanjuti dengan penyusunan dokumen \textit{User Requirement Specification} (URS) oleh \textit{Business Architect}. 
  Langkah berikutnya, \textit{Business Architect} melakukan \textit{architecture impact scoring} untuk mengidentifikasi seberapa besar pengaruh perubahan tersebut terhadap arsitektur bisnis dan aplikasi. 
  Penilaian ini dikategorikan menjadi dua yaitu \textit{high impact} dan \textit{low impact}.
  \textit{High impact} dikategorikan jika perubahannya besar seperti modifikasi proses bisnis utama, pergantian sistem, atau perubahan integrasi kritikal, sedangkan \textit{low impact} dikategorikan jika perubahannya minor seperti modifikasi fitur kecil atau tampilan. \\

  \textit{Business Architect} kemudian mendokumentasikan artefak bisnis seperti \textit{business process flow} dan \textit{user flow diagram} pada \textit{platform} SAP LeanIX. 
  Informasi yang dimasukkan di URS dan SAP LeanIX berupa \textit{current state} dan \textit{desired state} seaat perubahan diimplementasikan.
  Semua progres dipantau melalui excel \textit{tracker} untuk memastikan ketercapaian setiap aktivitas pada tahapan \textit{ideation}. 
  Tahap \textit{ideation} ini memastikan seluruh rencana perubahan telah tervalidasi sejak awal sebelum masuk ke tahap \textit{risk and impact}. 
  Gambar \ref{gambar:asis_ideation} menunjukkan tahapan \textit{ideation} saat ini.\\
  \begin{figure}[h] % pilihan opsi yang disarankan: t = top, b = bottom, h = here
    \centering
    \captionsetup{justification=centering}
        \includegraphics[width=1\textwidth]{image/as-isIdeation.png}
    \caption{Tahapan \textit{Ideation} Saat Ini}
    \label{gambar:asis_ideation}
  \end{figure}

  \item {Tahap \textit{Risk and Impact}} \\
  Tahap \textit{risk and impact} berfokus kepada analisis dampak yang muncul dari rencana perubahan sistem. 
  Setelah dokumen URS selesai diinisiasi pada tahap \textit{ideation}, dokumen tersebut akan dikirim oleh \textit{Business Architect} kepada \textit{Engineering Manager} untuk dilakukan \textit {review}.
  \textit{Engineering Manager} diberikan waktu maksimal 7 hari untuk memeriksa URS yang telah dibagikan. 
  Hasil \textit{review} tersebut disampaikan kembali kepada \textit{Business Architect} melalui grup Microsoft Teams. \\

  Setelah proses \textit{review}, \textit{Engineering Manager} memulai pembuatan artefak arsitektur aplikasi dan data pada \textit{platform} SAP LeanIX.. 
  Kegiatan ini berjalan secara paralel dengan langkah \textit{Cloud Infrastructure} dan \textit{IT Security} dalam membuat rancangan awal (\textit{draft}) arsitektur infrastruktur dan keamanan pada \textit{platform} SAP LeanIX.. 
  Tahapan \textit{Risk and Impact} bertujuan agar semua risiko potensial serta dampak perubahan dapat teridentifikasi dan dianalisis sebelum memasuki tahap \textit{development}. 
  Gambar \ref{gambar:asis_riskImpact} menunjukkan tahapan \textit{risk and impact} saat ini.\\
  \begin{figure}[h] % pilihan opsi yang disarankan: t = top, b = bottom, h = here
    \centering
    \captionsetup{justification=centering}
        \includegraphics[width=1\textwidth]{image/as-isRiskImpact.png}
    \caption{Tahapan \textit{Risk and Impact} Saat Ini}
    \label{gambar:asis_riskImpact}
  \end{figure}

  \item {Tahap \textit{Development}} \\
  Tahap \textit{development} berfokus dalam aktivitas pengembangan kode dimulai berdasarkan \textit{task} yang telah diberikan oleh \textit{Engineering Manager} kepada \textit{software engineering}. 
  Tahapan \textit{development} diatur berdasarkan hasil penilaian pada dokumen URS, khususnya keputusan terkait dampak perubahan (\textit{high impact} atau \textit{minimum impact}).
  Jika sebuah inisiatif bernilai \textit{high impact}, maka \textit{Business Architect} akan melakukan \textit{EA Project Monitoring} untuk fase \textit{development} dan melanjutkan artefak bisnis. 
  \textit{Enterprise Architect} akan menyediakan konsultasi terkait arsitektur, memastikan rancangan serta implementasi tetap sejalan dengan standar dan aturan AE perusahaan. 
  \textit{Cloud Infrastructure} dan \textit{IT Security} melanjutkan artefak infrastruktur dan keamanan. 
  Selain itu \textit{Engineering Manager}  melanjutkan artefak aplikasi dan data. 
  Pada \textit{minimum impact}, pembaruan artefak dilakukan tanpa konsultasi dengan \textit{Enterprise Architect} dan tanpa melakukan \textit{EA Project Monitoring}.\\

  Selama proses \textit {development}, pembaruan artefak dapat dilakukan secara paralel sesuai kebutuhan. 
  Sebelum transisi ke tahap \textit{UAT and Go-Live}, \textit{Enterprise Architect} akan melakukan \textit{review} kelengkapan \textit{EA Project Monitoring} serta memastikan seluruh \textit{checklist} terpenuhi. 
  Gambar \ref{gambar:asis_dev} menunjukkan tahapan \textit{development} saat ini.\\
  \begin{figure}[h] % pilihan opsi yang disarankan: t = top, b = bottom, h = here
    \centering
    \captionsetup{justification=centering}
        \includegraphics[width=1\textwidth]{image/as-isDev.png}
    \caption{Tahapan \textit{Development} Saat Ini}
    \label{gambar:asis_Dev}
  \end{figure}

  \item {Tahap \textit{UAT and Go-Live}} \\
  Tahap \textit{UAT and Go-Live} merupakan fase validasi akhir dan peluncuran sistem yang telah dikembangkan. 
  Prosesnya diawali dengan identifikasi kembali status \textit{high impact} atau \textit{minimum impact} atas perubahan yang akan diterapkan. 
  Jika perubahan dikategorikan \textit{high impact}, dilakukan \textit{EA Project Monitoring} untuk fase \textit{UAT and Go-Live}. 
  \textit{Enterprise Architect} memastikan seluruh \textit{checklist} sudah lengkap sebelum dinyatakan siap diproduksi.
  Sementara itu, untuk \textit{minimum impact}, proses \textit{EA Project Monitoring} tidak dilakukan. \\

  Pada tahap ini, seluruh artefak arsitektur (bisnis, aplikasi, data, infrastruktur, keamanan) harus difinalisasi sebelum sistem masuk ke produksi. 
  Setelah artefak final, maka dilanjutkan pengecekan penyelesaian artefak saat \textit{Change Advisory Board} (CAB) jika perubahan bersifat \textit{high impact}, sedangkan artefak akan di cek saat \textit{Bi-weekly Architecture Change Review} jika perubahannya bersifat \textit{minimum impact}. 
  Setelah seluruh proses verifikasi dilalui dan artefak dinyatakan lengkap, status perubahan yang \textit{minimum impact} akan ditandai "\textit{complete}" pada \textit{tracker}, sedangkan untuk \textit{high impact} diterima dalam \textit{Request for Change} (RFC) yang kemudian dapat dilanjutkan ke produksi. 
  Gambar \ref{gambar:asis_UAT} menunjukkan tahapan \textit{UAT and Go-Live} saat ini.\\
  \begin{figure}[h] % pilihan opsi yang disarankan: t = top, b = bottom, h = here
    \centering
    \captionsetup{justification=centering}
        \includegraphics[width=1\textwidth]{image/as-isUAT.png}
    \caption{Tahapan \textit{UAT and Go-Live} Saat Ini}
    \label{gambar:asis_UAT}
  \end{figure}

  \item {Tahap \textit{Hypercare}} \\
  Tahap \textit{hypercare} adalah fase akhir setelah sistem dinyatakan berhasil \textit{Go Live}. 
  Fase ini berfungsi sebagai masa pengawasan dan pendampingan operasional untuk memastikan bahwa hasil perubahan bisa berjalan stabil, tidak ada gangguan kritis, dan seluruh dokumentasi arsitektural telah lengkap. 
  Tahapan \textit{hypercare} diawali identifikasi kembali status \textit{high impact} atau \textit{minimum impact} atas perubahan yang akan diterapkan.
  Jika \textit{minimum impact}, proyek dapat langsung ditutup (\textit{closing project}) setelah konfirmasi stabilitas sistem dan kelengkapan artefak. 
  Jika \textit{high impact}, dilakukan \textit{EA Project Monitoring} untuk fase \textit{hypercare}. 
  \textit{Enterprise Architect} akan memastikan seluruh \textit{checklist} sudah lengkap sebelum \textit{closing project}. 
  Selain itu, \textit{Enterprise Architect} memberikan konsultasi arsitektural selama periode \textit{hypercare} untuk membantu verifikasi artefak. 
  Gambar \ref{gambar:asis_hypercare} menunjukkan tahapan \textit{hypercare} saat ini.\\
  \begin{figure}[h] % pilihan opsi yang disarankan: t = top, b = bottom, h = here
    \centering
    \captionsetup{justification=centering}
        \includegraphics[width=1\textwidth]{image/as-isHypercare.png}
    \caption{Tahapan \textit{Hypercare} Saat Ini}
    \label{gambar:asis_hypercare}
  \end{figure}
\end{enumerate}

\subsection{Pemanfaatan SAP LeanIX}
SAP LeanIX berperan penting sebagai \textit{platform} utama dalam mendukung proses tata kelola AE di Paragon Corp. 
Semua artefak arsitektur kondisi saat ini (\textit{current state}) dan kondisi yang diharapkan (\textit{desired state}) dicatat dan dikelola secara terpusat di SAP LeanIX.
\textit{Platform} ini memungkinkan seluruh tim yang terlibat dalam proses tata kelola AE dapat mengakses dan memperbarui dokumentasi secara kolaboratif.

\subsection{Permasalahan yang Ditemukan}
Setelah melakukan wawancara dengan Tim \textit{Enterprise Architect} Paragon Corp, pelaksanaan tata kelola AE di Paragon Corp masih menghadapi berbagai tantangan. 
Permasalahan tersebut dapat dikategorikan sebagai berikut:
\begin{enumerate}
  \item Keterbatasan Sumber Daya Manusia dan Beban Kerja \\
  Keterbatasan sumber daya manusia pada tim \textit{Cloud Infrastructure} dan \textit{IT Security} menyebabkan pembaruan dan dokumentasi artefak arsitektur belum dapat dilakukan secara optimal. 
  Pada proses saat ini artefak infrastruktur dan keamanan hanya diperbarui di SAP LeanIX dan tidak tercatat secara formal dalam dokumen URS. 
  Disisi lain, \textit{Business Architect} juga memiliki beban kerja yang tinggi yang menyebabkan keterbatasan waktu dalam mengambil peran pada proses tata kelola AE. \\

  \item Rendahnya Kesadaran Pentingnya Dokumentasi \\
  Kesadaran terhadap pentingnya dokumentasi pada artefak arsitektur belum sepenuhnya tertanam di seluruh anggota tim. 
  Dokumentasi sering dianggap sekadar formalitas atau beban administratif, bukan sebagai kebutuhan strategis untuk pengelolaan dan pengendalian arsitektur perusahaan ke depan.
  Hal ini berdampak pada kurangnya kedisiplinan dalam memperbarui dan melengkapi seluruh artefak AE secara berkala.\\

  \item Ketidakjelasan Proses Konsultasi ke \textit{Enterprise Architect} \\
  Saat ini belum ada parameter atau standar yang jelas mengenai aspek apa saja yang harus dikonsultasikan ke \textit{Enterprise Architect}. 
  Proses konsultasi sering dilakukan secara informal, tanpa aturan baku yang mengikat. 
  Di samping itu, Paragon Corp juga belum membentuk forum formal seperti \textit{Architecture Review Board} (ARB), sehingga pengambilan keputusan strategis terkait arsitektur masih kurang terstruktur dan kurang terpantau oleh lintas divisi. \\

  \item \textit{EA Impact Analysis} yang Belum Sesuai dengan Standar \\
  \textit{EA impact analysis} pada domain infrastruktur dan keamanan belum dilakukan secara komprehensif sejak awal. 
  Saat ini \textit{impact analysis} untuk kedua domain ini baru diperbarui atau di-(\textit{input}) ke SAP LeanIX setelah perubahan terlaksana. 
  Selain itu, \textit{impact analysis} untuk kedua domain ini tidak tercatat di dokumen URS, melainkan di SAP LeanIX saja.
  Hal ini membuka celah risiko operasional dan dapat menurunkan kualitas tata kelola arsitektur. \\

  Tim \textit{Enterprise Architect} juga menekankan masih perlu analisis untuk lima domain AE terkait \textit{EA impact analysis} ini.
  Hal ini dikarenakan prosedur untuk melakukan \textit{impact analysis} masih belum terdefinisi dengan jelas.\\

  \item Tantangan Budaya Organisasi \\
  Budaya “tidak ada paksaan” dalam organisasi menjadi tantangan tersendiri untuk menciptakan disiplin dan konsistensi dalam dokumentasi. 
  Program-program seperti penetapan \textit{Objective Key Results} (OKR) di Direktorat Teknologi Informasi (TI) telah diimplementasikan untuk mendorong kebiasaan mendokumentasi, namun efektivitasnya sangat bergantung pada komitmen pribadi masing-masing anggota, bukan pada sistem kontrol formal dari manajemen. \\
\end{enumerate}

Permasalahan yang akan diselesaikan dalam tugas akhir ini yaitu pelaksanaan \textit{EA impact analysis} yang belum sesuai dengan tata kelola yang diharapkan.
Permasalahan ini dipilih karena Tim \textit{Enterprise Architect} Paragon Corp menilai bahwa prosedur \textit{EA impact analysis} yang ada saat ini masih belum jelas.
Oleh karena itu, pada Bab IV akan dibahas rancangan solusi untuk memperbaiki proses \textit{EA impact analysis} dengan standar yang diharapkan organisasi.

\section{Analisis \textit{Gap} terhadap \textit{EA Maturity} Level 3 \textit{Defined}}
% \subsection{Perbandingan dan Evaluasi Tingkat Kematangan AE}
% Setelah wawancara dengan Tim \textit{Enterprise Architect}, penilaian terakhir terkait \textit{EA maturity} Paragon Corp berada di level 2.
% Evaluasi tingkat \textit{EA maturity} dilakukan dengan membandingkan kondisi AE saat ini di Paragon Corp dengan standar EAMM yang diadopsi dari TOGAF (mengacu ke Tabel \ref{tbl:EAMM}). 
% Penilaian dimulai dengan menguraikan karakteristik tiap level pada masing-masing dimensi, khususnya pada target level 3 (\textit{Defined}), lalu mengidentifikasi \textit{gap} yang didapatkan.
% Tabel \ref{tbl:Level3EAMM} menunjukkan hasil evaluasi antara kondisi aktual terhadap indikator level 3 (\textit{Defined}).

% \begin{longtable}{@{\extracolsep{\fill}}
    >{\raggedright\arraybackslash}p{3.0cm}
    >{\raggedright\arraybackslash}p{5.0cm}
    >{\raggedright\arraybackslash}p{5.0cm}
}
\caption{Penilaian Dimensi AE pada Level 3 \textit{Defined} Berdasarkan TOGAF dan Kondisi Aktual Paragon Corp}
\label{tbl:Level3EAMM} \\
\toprule
\textbf{Dimensi} &
\textbf{Indikator Level 3 (\textit{Defined}) (TOGAF)} &
\textbf{Kondisi Aktual Paragon Corp (Wawancara)} \\
\midrule
\endfirsthead

\caption[]{Penilaian Dimensi EA pada Level 3 \textit{Defined} (lanjutan)} \\
\toprule
\textbf{Dimensi} &
\textbf{Indikator Level 3 (\textit{Defined}) (TOGAF)} &
\textbf{Kondisi Aktual Paragon Corp (Wawancara)} \\
\midrule
\endhead

\midrule
\multicolumn{3}{r}{\textit{Bersambung ke halaman berikutnya}} \\
\endfoot

\bottomrule
\endlastfoot

% ====================== ROWS ======================

\textbf{\textit{Architecture Process}} &
Proses arsitektur sudah dijelaskan dengan jelas, terdokumentasi, dan disosialisasikan. Selain itu sudah ada \textit{gap analysis} dan rencana migrasi menuju kondisi ideal. &
Proses sudah disosialisasikan melalui dokumen dan SAP LeanIX. Namun, dalam pelaksanaannya, beberapa tim masih menjalankan proses dengan cara yang berbeda-beda. 
\textit{Gap analysis} juga sudah didokumentasikan.\\

\textbf{\textit{Architecture Development}} &    
TRM dan \textit{standards profile} sudah lengkap. Selain itu, \textit{gap analysis} dan rencana migrasi juga sudah selesai dibuat. &
Dokumentasi arsitektur untuk \textit{domain} bisnis dan aplikasi akan dimulai sejak awal perencanaan inisiatif. 
Namun, untuk \textit{domain} arsitektur dan keamanan dimulai beriringan berjalannya suatu inisiatif.\\

\textbf{\textit{Business Alignment}} &
AE sudah menjadi bagian dari proses perencanaan investasi dan pengendalian proyek. &
\textit{Business Architect} aktif menjaga agar kebutuhan TI tetap selaras dengan tujuan bisnis. 
Namun, beberapa keputusan masih berjalan berdasarkan kebutuhan operasional, bukan berdasarkan evaluasi AE secara menyeluruh.\\

\textbf{\textit{Organization}} &
Manajemen memberikan dukungan penuh dan keterlibatan lintas unit berjalan secara berkelanjutan dan kolaboratif. &
Struktur peran sudah ditetapkan dengan jelas. Namun, belum dilakukan secara proaktif tanpa dorongan konteks tertentu. \\

\textbf{\textit{Architecture Governance}} &
Tata kelola sudah terdokumentasi dan mencakup sebagian besar investasi TI dan adanya mekanisme pengecualian resmi (\textit{waiver}). &
Tanggung jawab dalam pengelolaan arsitektur sudah dibagi, dan terdapat proses pemantauan artefak.
Namun, forum \textit{Architecture Review Board} (ARB) belum berjalan secara formal.
Selain itu, kepatuhan terhadap standar masih bersifat himbauan, sehingga belum ada aturan wajib yang harus dipatuhi oleh semua tim. \\

\textbf{\textit{Architecture Communication}} &
Dokumentasi arsitektur diperbarui secara berkala dan dikomunikasikan secara rutin ke tim TI dan bisnis. &
SAP LeanIX sudah digunakan sebagai sumber referensi utama untuk dokumen arsitektur, dan pembaruan dokumen mulai dilakukan secara rutin.\\

\end{longtable}

% \subsection{Analisis \textit{Gap}}
% Berdasarkan permasalahan yang telah diidentifikasi sebelumnya, dilakukan analisis \textit{gap} dengan membandingkan kondisi tata kelola AE saat ini Paragon Corp terhadap indikator pada level 3 (\textit{Defined}) dalam TOGAF EAMM. 
% Setiap dimensi dievaluasi untuk mengidentifikasi aspek mana saja yang belum sesuai standar, sekaligus mencari faktor penyebab serta contoh konkret dari kesenjangan tersebut.
% Tabel \ref{tbl:GapEAMM} menunjukkan hasil \textit{gap} dan akar masalah \textit{EA Governance} Paragon Corp.

% \begin{longtable}{@{\extracolsep{\fill}}
    >{\raggedright\arraybackslash}p{2.5cm}
    >{\raggedright\arraybackslash}p{4cm}
    >{\raggedright\arraybackslash}p{4.5cm}
    >{\raggedright\arraybackslash}p{3cm}
}
\caption{\textit{Gap Analysis} EA Governance Paragon Corp}
\label{tbl:GapEAMM} \\
\toprule
\textbf{Dimensi} &
\textbf{Gap yang Teridentifikasi} &
\textbf{Faktor Penyebab (Akar Masalah)} &
\textbf{Contoh Konkret} \\
\midrule
\endfirsthead

\caption[]{\textit{Gap Analysis} EA Governance Paragon Corp (lanjutan)} \\
\toprule
\textbf{Dimensi} &
\textbf{\textit{Gap} yang Teridentifikasi} &
\textbf{Faktor Penyebab (Akar Masalah)} &
\textbf{Contoh Konkret} \\
\midrule
\endhead

\midrule
\multicolumn{4}{r}{\textit{Bersambung ke halaman berikutnya}} \\
\endfoot

\bottomrule
\endlastfoot

% ====================== ROWS ======================

\textbf{\textit{Architecture Process}} &
Konsultasi pada EA hanya untuk proyek \textit{high impact}; proyek \textit{minimum impact} tidak wajib konsultasi EA sehingga proses EA belum berjalan merata di seluruh proyek. &
\textit{Governance} belum mewajibkan konsultasi EA secara formal ke semua proyek; validasi EA baru opsional untuk proyek dengan dampak besar. &
Beberapa proyek minor berjalan tanpa konsultasi EA; dokumen URS hanya berisi aspek bisnis/aplikasi tanpa diskusi arsitektur. \\

\textbf{\textit{Architecture Development}} &
Artefak infrastruktur dan keamanan baru dibuat setelah proyek berjalan, tidak sejak awal proses secara proaktif. &
\textit{Awareness} tim \textit{Cloud Infrastructure} dan \textit{IT Security} rendah; tidak ada aturan tegas untuk dokumentasi dari awal. &
Artefak infrastruktur dan keamanan di SAP LeanIX hanya diisi setelah implementasi berjalan, tidak dari awal. \\

\textbf{\textit{Business Alignment}} &
Beberapa proyek IT berjalan tanpa validasi EA; analisis kebutuhan bisnis kadang belum formal di proyek kecil atau yang dianggap sederhana/\textit{urgent}. &
Belum ada kebijakan seluruh investasi IT/aktivitas wajib \textit{assessment} EA; kebutuhan bisnis kadang cukup didiskusikan secara singkat saja. &
Pembuatan aplikasi baru dilakukan tanpa melibatkan konsultasi formal ke EA. \\

\textbf{\textit{Organization}} &
Pelibatan seluruh unit dalam keputusan arsitektur belum berjalan kolektif; keputusan biasanya diambil hanya oleh tim inti EA/IT. &
Forum \textit{Architecture Review Board} (ARB) kolektif belum terbentuk; \textit{governance} kolaborasi masih berupa \textit{draft} desain. &
\textit{Review} dan \textit{approval} strategis cenderung didiskusikan pada tim EA/IT saja, tidak melalui forum bersama yang resmi. \\

\textbf{\textit{Architecture Governance}} &
Tidak ada forum \textit{Architecture Review Board} (ARB); pengambilan keputusan EA belum kolektif dan \textit{check-and-balance} masih terbatas. &
\textit{Governance structure} belum matang, belum ada proses formal sidang bersama atau \textit{waiver} yang jelas. &
Persetujuan perubahan strategis dilakukan hanya oleh kelompok inti tanpa diskusi/forum resmi. \\

\textbf{\textit{Architecture Communication}} &
Pemanfaatan dan komunikasi artefak EA masih terbatas pada IT; unit bisnis dan manajemen jarang mengakses artefak EA. &
Distribusi dokumen EA dan pelatihan LeanIX belum berjalan lintas unit/organisasi, \textit{awareness} belum cukup tinggi. &
SAP LeanIX aktif digunakan tim TI, namun \textit{stakeholder} bisnis belum mengakses perubahan arsitektur. \\

\end{longtable}

Setelah wawancara dengan Tim \textit{Enterprise Architect} Paragon Corp, penilaian terakhir terkait \textit{EA maturity} Paragon Corp berada di level 2.
Evaluasi tingkat \textit{EA maturity} dilakukan dengan membandingkan kondisi AE saat ini di Paragon Corp dengan standar EAMM yang diadopsi dari TOGAF (mengacu ke Tabel \ref{tbl:EAMM}). 
Penilaian dimulai dengan menguraikan karakteristik tiap level pada masing-masing dimensi, khususnya pada target level 3 (\textit{Defined}), lalu mengidentifikasi \textit{gap} yang didapatkan.
Pada Tabel \ref{tbl:GapEA} menunjukkan hasil analisis \textit{gap}.

\begin{landscape}
\begin{longtable}{@{\extracolsep{\fill}}
    p{2.8cm}
    p{2.8cm}
    p{2.8cm}
    p{2.8cm}
    p{2.8cm}}
\caption{Identifikasi \textit{Gap} dan Faktor Penyebab pada Setiap Dimensi EA}
\label{tbl:GapEA} \\
\toprule
\textbf{Dimensi} &
\textbf{Indikator Level 3 (TOGAF)} &
\textbf{Kondisi Aktual} &
\textbf{Gap} &
\textbf{Faktor Penyebab} \\
\midrule
\endfirsthead

\caption[]{Identifikasi \textit{Gap} dan Faktor Penyebab pada Setiap Dimensi EA (lanjutan)} \\
\toprule
\textbf{Dimensi} &
\textbf{Indikator Level 3 (TOGAF)} &
\textbf{Kondisi Aktual} &
\textbf{Gap} &
\textbf{Faktor Penyebab} \\
\midrule
\endhead

\midrule
\multicolumn{5}{r}{\textit{Bersambung ke halaman berikutnya}} \\
\endfoot

\bottomrule
\endlastfoot

% =========================================================
% ROWS
% =========================================================

\textbf{\textit{Architecture Process}} &
Proses arsitektur sudah dijelaskan dengan jelas, terdokumentasi, dan disosialisasikan. Selain itu sudah ada \textit{gap analysis} dan rencana migrasi menuju kondisi ideal. &
Proses sudah jelas dan disosialisasikan, namun, beberapa tim masih menjalankan proses dengan cara yang berbeda. Analisis \textit{gap} juga sudah ada, tetapi mekanismenya dan aspeknya masih belum jelas. &
Penerapan proses belum konsisten di seluruh tim. &
Perbedaan cara kerja antar tim dan belum ada mekanisme kontrol penerapan proses. \\

\textbf{\textit{Architecture Development}} &
TRM dan \textit{standards profile} sudah lengkap. Selain itu, \textit{gap analysis} dan rencana migrasi juga sudah selesai dibuat. &
Domain bisnis dan aplikasi mulai terdokumentasi sejak awal. Sementara itu, domain data, infrastruktur, dan keamanan terdokumentasi sambil berjalannya suatu inisiatif. &
Dokumentasi belum dilakukan sejak tahap inisiasi untuk semua domain. &
Belum semua domain memiliki kebiasaan dokumentasi awal. \\

\textbf{\textit{Business Alignment}} &
AE sudah menjadi bagian dari proses perencanaan investasi dan pengendalian proyek. &
\textit{Business Architect} aktif selaraskan kebutuhan TI dan bisnis. Namun beberapa tim tidak selalu melewati proses analisis EA dan tidak memberitahukan tim EA. &
Kepatuhan belum sempurna karena beberapa investasi/proyek tetap berjalan tanpa proses analisis EA yang lengkap. & 
Pengawasan dan mekanisme penegakan masih lemah. Selain itu proses koordinasi tidak selalu diikuti oleh beberapa tim \\

\textbf{\textit{Organization}} &
Manajemen memberikan dukungan penuh dan keterlibatan lintas unit berjalan secara berkelanjutan dan kolaboratif. &
Struktur peran sudah ditetapkan dengan jelas. Namun, belum dilakukan secara proaktif tanpa dorongan konteks tertentu. &
Kolaborasi lintas unit belum sepenuhnya menjadi kebiasaan proaktif di semua konteks. &
Belum ada mekanisme atau kebiasaan kerja yang mendorong kolaborasi lintas unit secara otomatis dan koordinasi lebih sering muncul setelah ada kebutuhan yang memicu. \\

\textbf{\textit{Architecture Governance}} &
Tata kelola sudah terdokumentasi dan mencakup sebagian besar investasi TI dan adanya mekanisme pengecualian resmi (\textit{waiver}). &
Forum \textit{Architecture Review Board} (ARB) belum berjalan secara formal.
Selain itu, kepatuhan terhadap standar masih bersifat himbauan, sehingga belum ada aturan wajib yang harus dipatuhi oleh semua tim. &
Tata kelola belum menyeluruh dan mekanisme \textit{waiver} belum tersedia. &
Ketiadaan ARB aktif, tidak ada aturan wajib, dan budaya kepatuhan belum kuat. \\

\textbf{\textit{Architecture Communication}} &
Dokumentasi arsitektur diperbarui secara berkala dan dikomunikasikan secara rutin ke tim TI dan bisnis. &
SAP LeanIX sudah digunakan sebagai sumber referensi utama untuk dokumen arsitektur, dan pembaruan dokumen mulai dilakukan secara rutin.&
-- Tidak teridentifikasi gap signifikan berdasarkan data saat ini -- &
-- Tidak ada faktor penyebab yang perlu dicatat karena kondisi aktual sejauh ini konsisten dengan indikator Level 3. \\

\end{longtable}

\end{landscape}

\section{Analisis Kebutuhan Perbaikan Tata Kelola Arsitektur \textit{Enterprise}}
Berdasarkan hasil identifikasi masalah dan analisis \textit{gap} dalam implementasi tata kelola AE di Paragon Corp, terdapat sejumlah kebutuhan mendesak yang harus diperbaiki agar tata kelola arsitektur dapat berjalan lebih efektif dan konsisten.
\begin{enumerate}
  \item Konsistensi Proses Konsultasi dan \textit{Review} EA \\
  Penegasan kebijakan diperlukan pada setiap proyek, baik itu \textit{high impact} maupun \textit{minimum impact} tetap melewati tahap konsultasi dan \textit{review} oleh tim EA. 
  Standardisasi proses ini penting agar tidak ada perubahan yang luput dari pengawasan arsitektur, sehingga risiko ketidaksesuaian atau tumpang tindih bisa dicegah sejak dini. \\

  \item Peningkatan Kesadaran dan Kebijakan Dokumentasi Sejak Awal \\
  Semua tim, terutama yang menangani infrastruktur dan keamanan, harus didorong untuk secara proaktif mendokumentasikan artefak arsitektur sejak tahap awal proyek, bukan setelah implementasi berjalan. 
  Peningkatan kesadaran ini penting agar risiko dapat teridentifikasi lebih awal dan kualitas dokumentasi lebih terjamin. \\

  \item Pembentukan \textit{Architecture Review Board} (ARB) \\
  Paragon Corp perlu segera membentuk forum formal seperti ARB yang melibatkan berbagai peran dan unit. Forum ini berfungsi untuk mengkaji, mengesahkan, dan memantau keputusan arsitektur agar lebih transparan, kolektif, serta tidak hanya didominasi oleh tim inti TI saja. \\

  \item Integrasi Evaluasi Bisnis dalam Setiap Pengembangan IT \\
  Penilaian atas kebutuhan bisnis harus menjadi landasan utama dalam setiap perubahan atau investasi TI. Setiap proyek atau pengembangan solusi teknologi sebaiknya diwajibkan melalui validasi dan kajian formal bersama \textit{Business Architect}, 
  sehingga seluruh inisiatif benar-benar selaras dengan prioritas dan strategi perusahaan. \\

  \item Optimalisasi Pemanfaatan SAP LeanIX Lintas Unit \\
  SAP LeanIX tidak cukup hanya digunakan oleh tim TI. Perlu dilakukan edukasi dan pelatihan agar unit bisnis dan manajemen juga mampu mengakses, memahami, serta memanfaatkan artefak EA sebagai bahan referensi untuk pengambilan keputusan bersama. \\

  \item Pembiasaan Budaya Dokumentasi dan Komunikasi Terbuka \\
  Membangun budaya dokumentasi yang kuat serta komunikasi artefak arsitektur yang terbuka ke seluruh pemangku kepentingan sangat diperlukan. Semua pihak harus sadar bahwa dokumen arsitektur bukan hanya pelengkap administrasi, namun landasan pengendalian dan perbaikan tata kelola TI di masa depan. 
\end{enumerate}

\section{Analisis Pemilihan Solusi}
\subsection{Alternatif Solusi}
Berdasarkan permasalahan yang akan diselesaikan, yaitu pelaksanaan \textit{EA impact analysis} yang belum memiliki tata kelola yang jelas.
Bagian ini merumuskan tiga alternatif solusi yang dapat diterapkan oleh Paragon Corp. 
Ketiga alternatif ini dirancang setara sehingga depat dibandingkan melalui analisis keputusan pada bagian berikutnya.
\begin{enumerate}
  \item Penyusunan SOP \textit{EA Impact Analysis} Terstandarisasi yang Didukung SAP LeanIX \\
  Penyusunan SOP \textit{EA impact analysis} mengacu pada kerangka kerja TOGAF sebagai referensi.
  TOGAF dipilih karena memberikan panduan mengenai langkah-langkah analisis arsitektur, termasuk bagaimana menilai dampak terhadap \textit{domain} bisnis, data, aplikasi, infrastruktur, dan keamanan.
  Selain itu, perusahaan akan memiliki dasar atau panduan yang kuat dalam menjalankan proses \textit{EA impact analysis} karena prosesnya tidak lagi bergantung pada interpretasi setiap tim.
  SOP ini juga memastikan seluruh aktivitas atau proses analisis memiliki urutan yang jelas.\\
 
  Dalam penerapan solusi ini, SOP akan didukung oleh \textit{platform} SAP LeanIX untuk mengisi dan menyimpan hasil analisis.
  SAP LeanIX sudah digunakan oleh Paragon Corp sebagai repositori AE sehingga integrasi dengan SOP akan memastikan semua analisis terdokumentasi di satu tempat yang dapat diakses semua tim terkait.
  Dokumentasi hasil analisis tidak lagi tersebar atau berbeda-beda formatnya karena seluruh tim mengisi pada \textit{platform} yang sama.
  Selain itu, SAP LeanIX juga menyediakan katalog arsitektur dan relasi antar domain sehingga proses menjadi lebih efisien.\\
  
  Solusi ini menekankan konsistensi, kejelasan, dan keseragaman dalam pelaksanaan \textit{EA impact analysis}. 
  Solusi ini juga mengatasi masalah utama yaitu belum adanya prosedur formal yang diikuti dan adanya perbedaan cara pelaksanaan antar tim.
  Oleh karena itu, SOP ini akan berperan menjadi pedoman dan SAP LeanIX akan berperan dalam menjaga dokumentasi.
  Tabel \ref{tbl:alt1} menunjukkan kelebihan dan kekurangan solusi ini. \\
  \begin{table}[h]
    \centering
    \begin{tabular}{|p{3cm}|p{4cm}|p{4cm}|}
    \hline
    \textbf{Aspek} & \textbf{Kelebihan} & \textbf{Kekurangan} \\
    \hline
    Kejelasan Proses &
    Memberikan alur \textit{EA impact analysis} yang jelas dan terstruktur karena mengikuti praktik TOGAF. &
    Penyusunan SOP membutuhkan waktu, diskusi lintas unit, serta validasi berulang agar disepakati semua pihak. \\
    \hline
    Standarisasi &
    Menghasilkan keseragaman proses sehingga seluruh tim bekerja dengan pedoman yang sama. &
    Standarisasi dapat dirasakan terlalu formal oleh beberapa tim yang belum terbiasa bekerja dengan prosedur terstruktur. \\
    \hline
    Integrasi LeanIX &
    LeanIX mendukung dokumentasi sehingga mempermudah pencatatan dan pelacakan hasil analisis. &
    Perlu penyesuaian konfigurasi LeanIX agar sepenuhnya selaras dengan SOP baru yang disusun. \\
    \hline
    \end{tabular}
    \caption{Kelebihan dan Kekurangan Alternatif 1: SOP \textit{EA Impact Analysis} berbasis TOGAF dan LeanIX}
    \label{tbl:alt1}
    \end{table}
    

  \item Penerapan \textit{Checklist} dan \textit{Template EA Impact Analysis} \\
  Penyusunan \textit{checklist} digunakan setiap ada inisiatif atau perubahan yang dapat mempengaruhi arsitektur.
  \textit{Checklist} ini disusun untuk memastikan bahwa setiap aspek penting dalam arsitektur diperiksa.
  Aspek yang diperiksa seperti dampak terhadap proses bisnis, aplikasi terkait, data yang digunakan, potensi risiko keamanan, kesiapan infrastruktur, dan dependensi yang perlu diperhatikan.
  Dengan solusi ini, tim tidak perlu menafsirkan sendiri apa saja yang harus dianalisis sehingga kualitas analisis dapat lebih konsisten.\\

  Selain \textit{checklist}, \textit{template EA impact analysis} disediakan untuk mendokumentasikan hasil analisis.
  \textit{Template} ini memastikan laporan mudah dibaca dan dapat ditinjau oleh Tim \textit{Enterprise Architect}.
  \textit{Template} dapat memuat bagian-bagian seperti ringkasan inisiatif, \textit{domain} terdampak dan risiko yang ditemukan.\\

  Pendekatan solusi ini dapat diterapkan dengan cepat tanpa memerlukan perubahan besar pada struktur organisasi atau alur proses.
  Solusi ini cocok diterapkan ketika organisasi membutuhkan perbaikan yang cepat, mudah diadopsi, dan tidak membebani tim.
  Selain itu, solusi ini mendukung tim yang mungkin belum terbiasa melakukan analisis arsitektur secara menyeluruh, sehingga mereka tetap memiliki panduan jelas dalam melakukan pekerjaan.
  Tabel \ref{tbl:alt2} menunjukkan kelebihan dan kekurangan solusi ini. \\
  \begin{table}[h]
    \centering
    \begin{tabular}{|p{3cm}|p{4cm}|p{4cm}|}
    \hline
    \textbf{Aspek} & \textbf{Kelebihan} & \textbf{Kekurangan} \\
    \hline
    Kemudahan Adopsi &
    \textit{Checklist} dan \textit{template} mudah dipahami serta cepat diterapkan oleh seluruh tim. &
    Tidak memberikan panduan proses yang selengkap SOP sehingga beberapa interpretasi masih dapat berbeda antar unit. \\
    \hline
    Konsistensi \textit{Output} &
    Membantu memastikan hasil analisis lebih seragam dan lengkap. &
    Hasil analisis sangat bergantung pada kedisiplinan pengguna dalam mengisi \textit{checklist} secara benar. \\
    \hline
    Implementasi Cepat &
    Tidak membutuhkan perubahan besar pada tata kelola atau struktur organisasi. &
    Dapat kurang efektif untuk kasus yang kompleks karena \textit{checklist} bersifat ringkas dan tidak mendalami proses. \\
    \hline
    \end{tabular}
    \caption{Kelebihan dan Kekurangan Alternatif 2: \textit{Checklist} dan \textit{Template EA Impact Analysis}}
    \label{tbl:alt2}
    \end{table}
    

  \item Pembentukan Forum Untuk Peninjauan \textit{EA Impact Analysis} \\
  Pembentukan forum berfokus kepada peninjauan hasil \textit{EA impact analysis}.
  Forum berfungsi sebagai mekanisme yang lebih ringkas dan cepat untuk memverifikasi bahwa analisis yang dilakukan sudah benar dan tidak ada aspek penting yang terlewat.
  Forum ini menjadi tempat untuk mengklarifikasi temuan, menilai risiko, dan memastikan bahwa keputusan diambil berdasarkan pemahaman yang lengkap.
  Forum juga memastikan bahwa seluruh \textit{domain} terdampak sudah ditinjau sesuai kebutuhan.
  Dengan adanya forum ini, hasil analisis dapat diperiksa secara lintas peran sebelum sebuah keputusan dibuat.\\
  
  Solusi ini memperkuat tata kelola AE tanpa membutuhkan perubahan besar pada alur proses.
  Forum memberikan mekanisme validasi formal namun tetap praktis, dan dapat diterapkan sesuai kebutuhan.
  Selain itu, forum ini membantu memastikan kualitas hasil analisis sekaligus meningkatkan koordinasi lintas unit.
  Tabel \ref{tbl:alt3} menunjukkan kelebihan dan kekurangan solusi ini. \\
  \begin{table}[h]
    \centering
    \begin{tabular}{|p{3cm}|p{4cm}|p{4cm}|}
    \hline
    \textbf{Aspek} & \textbf{Kelebihan} & \textbf{Kekurangan} \\
    \hline
    Validasi Kualitas &
    Memberikan mekanisme \textit{review} lintas tim sehingga hasil \textit{EA Impact Analysis} lebih akurat dan tidak ada aspek yang terlewat. &
    Membutuhkan komitmen waktu dari perwakilan unit yang terlibat sehingga dapat menambah beban koordinasi. \\
    \hline
    Tata Kelola &
    Menghadirkan proses pengawasan yang lebih formal dibandingkan hanya \textit{checklist}. &
    Jika tidak dijalankan konsisten, forum berpotensi menjadi formalitas tanpa peningkatan kualitas yang nyata. \\
    \hline
    Kolaborasi Lintas Unit &
    Memperkuat kolaborasi sehingga persepsi dampak lebih menyeluruh. &
    Tanpa dokumentasi yang kuat, keputusan forum dapat sulit ditelusuri atau direferensikan kembali. \\
    \hline
    \end{tabular}
    \caption{Kelebihan dan Kekurangan Alternatif 3: Forum Peninjauan \textit{EA Impact Analysis}}
    \label{tbl:alt3}
    \end{table}
    
\end{enumerate}

Tiga alternatif solusi yang telah dijelaskan sebelumnya merupakan pilihan dalam perbaikan tata kelola AE Paragon Corp terkhusus pada proses \textit{EA impact analysis}. 
Setiap solusi memiliki karakteristik, keunggulan, serta potensi tantangannya sendiri. 
Pada bagian berikutnya, akan dilakukan analisis penentuan solusi untuk memilih alternatif yang paling sesuai dengan kebutuhan dan tujuan organisasi.

\subsection{Analisis Penentuan Solusi}
Untuk memilih solusi terbaik dari ketiga alternatif solusi tata kelola EA dipilih metode \textit{decision matrix}. 
Pemilihan metode ini dikarenakan pengambilan keputusan menjadi transparan dengan menetapkan bobot numerik pada setiap kriteria.

Kriteria penilaian meliputi struktur tata kelola, kemudahan implementasi, keselarasan dengan strategi bisnis, efisiensi operasional, fleksibilitas adaptasi, serta kontrol dan audit. 
Bobot setiap kriteria diberikan berdasarkan prioritas dan dampaknya terhadap efektivitas tata kelola EA di Paragon Corp.

Struktur tata kelola berbobot menjadi dasar tata kelola EA yang akan menentukan peran tanggung jawab, proses, dan pengawasan aktivitas proyek.
Keselerasan dengan strategi bisnis memastikan proses EA mendukung tujuan jangka panjang organisasi.
Kemudahan implementasi memastikan solusi tata kelola EA yang dipilih akan mempercepat proses, mengurangi duplikasi, dan meningkatkan produktivitas.
Fleksibilitas adaptasi memastikan kerangka kerja relevan dengan perkembangan bisnis dan teknologi.
Kontrol dan audit berbobot memastikan kepatuhan dan akuntabilitas proses tata kelola EA.

Bobot tertinggi (20\%) diberikan untuk kriteria struktur tata kelola dan kriteria keselarasan dengan strategi bisnis karena hasil evaluasi dengan pemangku kepentingan menunjukkan
permasalahan di Paragon Corp yaitu belum jelasnya peran dan proses tata kelola EA. 
Tata kelola EA yang lemah akan menyebabkan risiko duplikasi sistem dan respon perubahan yang lambat.
Bobot 15\% pada kritiria kemudahan implementasi diberikan karena keterbatasan SDM dan kesiapan adopsi kerangka kerja baru, sehingga solusi yang mudah diimplementasikan
akan memberikan dampak yang nyata.

Efisiensi operasional, fleksibilitas adaptasi, dan kontrol \& audit masing-masing mendapat bobot 15\% karena ketiganya sama penting untuk memastikan perbaikan tata kelola EA 
akan meningkatkan kualitas layanan, beradaptasi dengan kebutuhan bisnis, dan memenuhi standar kepatuhan arsitektur perusahaan. 
Bobot-bobot ini didasarkan pada analisis \textit{gap} terkait tata kelola EA di Paragon Corp.
Tabel \ref{tbl:decision_matrix_EA} menunjukkan hasil dari \textit{decision matrix} dalam memilih solusi terbaik.

\begin{table}[h]
    \centering
    \begin{tabular}{|p{4cm}|p{2cm}|p{2cm}|p{2cm}|p{2cm}|}
    \hline
    \textbf{Kriteria} & \textbf{Bobot (\%)} & \textbf{Alternatif 1} & \textbf{Alternatif 2} & \textbf{Alternatif 3} \\
    \hline
    Kesesuaian dengan Tata Kelola EA & 30 & 5 & 3 & 4 \\
    \hline
    Kemudahan Implementasi & 20 & 3 & 5 & 3 \\
    \hline
    Dampak terhadap Konsistensi Proses & 25 & 5 & 3 & 4 \\
    \hline
    Kebutuhan Koordinasi Antar Unit & 15 & 4 & 5 & 3 \\
    \hline
    Kesesuaian dengan SAP LeanIX & 10 & 5 & 3 & 3 \\
    \hline
    \textbf{Total nilai}       & \textbf{100} & \textbf{4.45} & \textbf{3.75} & \textbf{3.55} \\
    \hline
    \end{tabular}
    \caption{\textit{Decision Matrix} Penentuan Alternatif Solusi Perbaikan Proses \textit{EA Impact Analysis}}
    \label{tbl:decision_matrix_EA}
    \end{table}
    

Berdasarkan hasil \textit{decision matrix}, solusi dengan skor total tertinggi yaitu \textit{EA Governance} berbasis PMBOK dan TOGAF.
Hal ini didasarkan karena keunggulan struktur tata kelola yang jelas, kemudahan dalam menyelaraskan dengan strategi bisnis, dan fleksibel terhadap pertumbuhan perusahaan.
Selain itu, solusi ini akan mengatasi tantangan dokumentasi, proses, dan mengawasi tata kelola.

COBIT menunjukkan aspek kontrol dan audit menjadi kelebihannya, sehingga menjadi alternatif solusi untuk perusahaan yang fokus ke tata kelola berbasis kontrol dan penilaian formal.
Namun, fleksibilitas COBIT lebih rendah untuk organisasi yang masih berubah secara dinamis dan terus berinovasi.

Sementara itu, ITIL sangat optimal dalam mendukung efisiensi pelayanan dan kualitas layanan TI di tingkat operasional.
Namun, kerangka kerja ini kurang mampu mengakomodasi kebutuhan tata kelola EA yang bersifat strategis dan lintas domain.
Selain itu, kerangka ini tidak mendalami pengembangan dan desain arsitektur bisnis teknologi secara menyeluruh.

Dengan demikian, solusi \textit{EA Governance} berbasis PMBOK dan TOGAF dipilih menjadi model yang sesuai bagi Paragon Corp.
Kerangka kerja ini memberikan keseimbangan antara tata kelola EA, pengelolaan arsitektur, dan kemampuan adaptasi terhadap kebutuhan perusahaan.
Pilihan ini memungkinkan evaluasi efektivitas dan mencapai \textit{maturity governance EA} dilakukan secara objektif dan berkelanjutan.

