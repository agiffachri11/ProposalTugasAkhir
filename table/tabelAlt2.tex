\begin{table}[h]
    \centering
    \begin{tabular}{|p{3cm}|p{4cm}|p{4cm}|}
    \hline
    \textbf{Aspek} & \textbf{Kelebihan} & \textbf{Kekurangan} \\
    \hline
    Kemudahan Adopsi &
    \textit{Checklist} dan \textit{template} mudah dipahami serta cepat diterapkan oleh seluruh tim. &
    Tidak memberikan panduan proses yang selengkap SOP sehingga beberapa interpretasi masih dapat berbeda antar unit. \\
    \hline
    Konsistensi \textit{Output} &
    Membantu memastikan hasil analisis lebih seragam dan lengkap. &
    Hasil analisis sangat bergantung pada kedisiplinan pengguna dalam mengisi \textit{checklist} secara benar. \\
    \hline
    Implementasi Cepat &
    Tidak membutuhkan perubahan besar pada tata kelola atau struktur organisasi. &
    Dapat kurang efektif untuk kasus yang kompleks karena \textit{checklist} bersifat ringkas dan tidak mendalami proses. \\
    \hline
    \end{tabular}
    \caption{Kelebihan dan Kekurangan Alternatif 2: \textit{Checklist} dan \textit{Template EA Impact Analysis}}
    \label{tbl:alt2}
    \end{table}
    