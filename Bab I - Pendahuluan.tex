% ==========================================
% BAB I PENDAHULUAN
% ==========================================
\chapter{PENDAHULUAN}
\label{chap:pendahuluan}
% --- Latar Belakang ---
\section{Latar Belakang}
Transformasi digital merupakan salah satu faktor yang memengaruhi keberlangsungan organisasi.
Perkembangan Teknologi Informasi (TI) memaksa organisasi untuk menyesuaikan proses bisnis dan layanan.
Menurut \textcite{juraida2024enterprise}, Arsitektur \textit{Enterprise} (AE) merupakan elemen yang berperan dalam mendukung transformasi digital
melalui penyelarasan antara strategi bisnis dan strategi Teknologi Informasi (TI) organisasi.

Aspek AE yang diperlukan salah satunya yaitu tata kelola yang jelas untuk mengelola arsitektur, mengambil keputusan, dan kepatuhan terhadap arsitektur yang dirancang.
Karakterisitik AE harus diintegrasikan dengan praktik manajemen tata kelola agar AE dapat dikelola secara berkelanjutan \autocite{nugroho2016enterprise}.

\textcite{virantina2020digital} menegaskan efektivitas AE sangat bergantung kepada kemampuan organisasi dalam mengelola siklus hidup arsitektur, termasuk bagaimana AE digunakan untuk mendukung proses transformasi digital, manajemen data, dan efisiensi operasional. 
Penerapan tata kelola AE juga memastikan pembagian peran, tanggung jawab, dan mekanisme evaluasi untuk memastikan keberlanjutan AE.

Untuk memastikan tata kelola AE berjalan efektif, TOGAF menyediakan \textit{Architecture Capability Maturity Model} (ACMM) yang digunakan untuk menilai tingkat kematangan kapabilitas arsitektur organisasi.
Model ini memiliki 5 tingkatan dan pada level 3 (\textit{defined}), proses tata kelola AE telah terdokumentasi, dilaksanakan secara konsisten, dan memiliki struktur peran yang jelas \autocite{TOGAF_Standard_10th}.
Saat ini, tingkat kematangan tata kelola AE Paragon Corp berada pada level 2.
Oleh karena itu level 3 akan dijadikan acuan untuk menilai tata kelola AE Paragon Corp dan mengevaluasi dimensi yang perlu diperbaiki.

Penerapan AE memerlukan evaluasi terhadap seberapa efektif tata kelolanya.
Efektivitas ini bisa diketahui melalui tingkat kepatuhan proyek terhadap standar, dokumentasi yang dihasilkan, dan keterlibatan pemangku kepentingan \autocite{foorthuis2016theory}.
Cara ini selaras dengan TOGAF, yaitu pentingnya melakukan evaluasi terhadap kapabilitas arsitektur untuk meningkatkan tingkat kematangan tata kelola organisasi.

Pada praktik organisasi, tata kelola AE yang gagal menimbulkan banyak risiko.
Paragon Corp merupakan salah satu perusahaan yang memiliki alur tata kelola AE tetapi belum dijalankan secara formal. 
Meskipun tim \textit{Enterprise Architecture} di Paragon Corp baru dibentuk sejak tahun lalu, belum ada mekanisme rutin untuk \textit{architecture review} dan persetujuan arsitektur.
Kondisi ini menyebabkan duplikasi sistem dan ketiadaan standar arsitektural karena belum ada referensi terhadap arsitektur sebelumnya.
Evaluasi tata kelola AE menemukan bahwa aspek perancangan AE yang sudah didefinisikan, namun belum dikelola secara menyeluruh akan menghambat efektivitas AE sebagai kerangka strategis \autocite{ghiffari2022evaluasi}.

Dengan mempertimbangkan tantangan tersebut, Tugas Akhir ini berfokus kepada analisis perbaikan tata kelola AE sehingga meningkatkan efektivitas peran AE dalam mengambil keputusan di Paragon Corp. 

% --- Rumusan Masalah ---
\section{Rumusan Masalah}
Berdasarkan latar belakang yang telah diuraikan sebelumnya, masalah utama yang dihadapi yaitu tata kelola AE pada Paragon Corp belum dijalankan secara formal.
Hal ini bisa mengakibatkan duplikasi sistem dan ketiadaan standar arsitektural.
Masalah ini menjadi landasan yang akan diselesaikan pada Tugas Akhir ini.
Rumusan masalah dijabarkan sebagai berikut.
\begin{enumerate}
\item	Apa saja \textit{gap} untuk mencapai tingkat kematangan level 3 tata kelola AE?
\item	Bagaimana rancangan tata kelola AE yang lebih terstruktur?
\end{enumerate}

% --- Tujuan ---
\section{Tujuan}
Berdasarkan rumusan masalah yang telah dipaparkan sebelumnya, tujuan utama dari pengerjaan Tugas Akhir ini yaitu merancang model perbaikan tata kelola AE untuk meningkatkan efektivitas dalam mengambil keputusan arsitektur.
Untuk mendukung tujuan tersebut, ada tujuan pendorong agar tujuan utama berhasil diimplementasikan.
Tujuan dari Tugas Akhir ini adalah sebagai berikut:
\begin{enumerate}
\item	Mengidentifikasi \textit{gap} antara kondisi saat ini dengan \textit{best practice} AE pada tingkat kematangan level 3 tata kelola AE.
\item	Merancang model perbaikan tata kelola AE yang efektif dan sesuai dengan konteks organisasi.
\end{enumerate}

% --- Metodologi Pengerjaan TA ---
\section{Metodologi}
Tahapan dalam pengerjaan Tugas Akhir ini dilakukan dengan menggunakan metodologi \textit{Design Research Science Methodology} (DSRM).
DSRM adalah pendekatan yang berfokus dalam memecahkan masalah dan dirancang untuk memberikan solusi terhadap suatu masalah yang dihadapi organisasi \autocite{absari2022design}.
Metodologi DSRM terdiri dari enam tahapan yang dijalankan secara sistematis.
\begin{enumerate}
    \item \textit{Problem Identification} \\
    Pada tahapan ini, penulis melakukan eksporasi dalam proses tata kelola AE.
    Tahapan ini dilakukan dengan analisis dokumen terkait tata kelola AE dan melakukan wawancara dengan Tim \textit{Enterprise Architect} Paragon Corp untuk mendapatkan informasi tentang proses tata kelola saat ini.
    Tahapan ini menghasilkan pernyataan masalah yang jelas dan ruang lingkup pengerjaan Tugas Akhir yang difokuskan. \\

    \item \textit{Define Objectives of a Solution} \\
    Pada tahapan ini, penulis menetapkan tujuan yang harus dicapai oleh solusi berdasarkan masalah yang telah dianalisis.
    Tujuan dirumuskan dengan mempertimbangkan kebutuhan perbaikan tata kelola AE dan kondisi operasional Paragon Corp.
    Tujuan ini menjadi panduan dalam menentukan arah desain solusi pada tahapan berikutnya. \\

    \item \textit{Design and Development} \\
    Pada tahapan ini, penulis menentukan \textit{requirement} kemudian merancang solusi berdasarkan \textit{requirement} yang sudah ditentukan.
    Solusi disusun berdasarkan kerangka kerja arsitektur TOGAF untuk memastikan solusi mencakup aspek penting.
    Selain itu, penulis merancang solusi untuk menjawab permasalahan tata kelola AE Paragon Corp.
    Peracangan dilakukan berdasarkan prinsip-prinsip AE dan standar perusahaan. \\
    
    \item \textit{Demonstration} \\
    Pada tahapan ini, penulis akan menunjukkan bagaimana solusi yang dirancang dapat digunakan dalam konteks perusahaan.
    Tahap ini bertujuan memastikan solusi dapat diterapkan secara praktis. \\

    \item \textit{Evaluation} \\
    Pada tahapan ini, solusi yang telah dikembangkan dinilai efektivitas, kelayakan, dan kesesuaiannya dengan tujuan yang ditetapkan.
    Evaluasi ini akan melibatkan diskusi dan validasi bersama Tim \textit{Enterprise Architect} Paragon Corp maupun dosen pembimbing untuk menilai apakah solusi telah mengatasi permasalahan yang terjadi.
    Evaluasi ini memberikan masukan untuk melakukan penyempurnaan terhadap solusi. \\

    \item \textit{Communication} \\
    Pada tahapan ini, penulis menyampika hasil solusi yang telah dirancang dan cara solusi tersebut menyelesaikan masalah yang terjadi.
    Penyampaian ini dilakukan dengan Tim \textit{Enterprise Architect} Paragon Corp dan dosen pembimbing untuk memastikan hasil solusi dapat dipahami dan diterapkan sesuai kebutuhan perusahaan. \\
\end{enumerate}