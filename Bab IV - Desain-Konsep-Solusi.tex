% ==========================================
% BAB IV DESAIN KONSEP SOLUSI
% ==========================================
\chapter{DESAIN KONSEP SOLUSI}
\label{chap:desain-konsep-solusi}
\section{Gambaran Sistem dan Proses Saat Ini (\textit{As-Is})}
Pada alur tata kelola Arsitektur \textit{Enterprise} (AE) saat ini terdiri dari 5 tahapan, yaitu \textit{Ideation, Risk and Impact, Development, UAT and Go Live,} dan \textit{Hypercare}.
Proses \textit{EA Impact Analysis} terjadi pada tahap \textit{Ideation} dan merupakan bagian dari dokumen \textit{User Requirement Specification} (URS).
Pada tahap ini, \textit{Business Architect} wajib mengisi bagian \textit{EA Impact Analysis} untuk memberikan gambaran awal mengenai dampak perubahan terhadap tiga \textit{domain} arsitektur saja, yaitu \textit{Business Architecture, Application Architecture,} dan \textit{Data Architecture}.
Dengan demikian, proses \textit{EA Impact Analysis} tidak mencakup semua \textit{domain} arsitektur.

Pada kondisi saat ini, proses \textit{EA Impact Analysis} di URS terdiri dari 2 bagian utama, yaitu \textit{current state} dan \textit{desired state}.
Sementara itu, \textit{domain} yang tercakup adalah \textit{Business Architecture, Application Architecture,} dan \textit{Data Architecture}.
\begin{enumerate}
    \item \textit{Business Architecture} \\
    Pada bagian \textit{current state}, memuat gambaran proses bisnis yang berjalan saat ini. 
    Informasi yang disajikan dalam bentuk deskripsi proses saat ini.
    Sementara itu, pada bagian \textit{desired state}, menjelaskan target proses bisnis yang diharapkan, termasuk proses baru, perubahan alur, atau penyempurnaan proses. \\
  
    \item \textit{Application Architecture}  \\
    Pada bagian \textit{current state}, memuat gambaran \textit{current application landscape}, fitur aplikasi yang akan berubah, dan integrasi aplikasi yang mungkin terdampak. 
    Sementara itu, pada bagian \textit{desired state}, memuat \textit{target application landscape}, perubahan fitur yang diharapkan, dan ekspetasi perubahan pada integrasi aplikasi. \\ 
  
    \item \textit{Data Architecture} \\
    Pada \textit{domain} ini, proses \textit{EA Impact Analysis} disusun berdasarkan tiga lapisan, yaitu \textit{Conceptual Layer, Logical Layer,} dan \textit{Technical Layer}.
    Pada bagian \textit{current state}, berisi penjelasan kebutuhan bisnis berbasis data (\textit{conceptual}), aliran data antar aplikasi (\textit{logical}), dan teknologi yang saat ini digunakan (\textit{technical}).
    Sementara itu, pada bagian \textit{desired state}, berisi penjelasan kebutuhan bisnis yang ingin dicapai melalui data (\textit{conceptual}), aliran data yang diinginkan antar aplikasi (\textit{logical}), dan teknologi yang diharapkan akan digunakan (\textit{technical}).\\ 
\end{enumerate}

Dari struktur dan proses \textit{EA Impact Analysis} di URS memiliki beberapa keterbatasan. Pada proses \textit{EA Impact Analysis} tidak mencakup seluruh \textit{domain} arsitektur karena \textit{domain Infrastructure} dan \textit{Security} tidak disertakan dalam URS.
Informasi mengenai dampak perubahan pada kedua \textit{domain} ini hanya ditemukan melalui diagram yang ada di LeanIX, dan harus memerlukan komunikasi dengan pembuat diagram karena informasi perubahan tidak dijelaskan secara eksplisit.
Pada \textit{domain Application Architecture} juga mengalami keterbatasan karena analisis integrasi aplikasi tidak dijelaskan kedalaman informasi yang harus dimasukkan. 
Selain itu, pada \textit{domain Data Architecture}, belum ada perbedaan detail antara \textit{conceptual, logical,} dan \textit{technical} sehingga analisis belum sepenuhnya akurat.
Oleh karena itu, untuk keseluruhan proses \textit{EA Impact Analysis} ini akan disesuaikan kembali dengan kerangka kerja TOGAF. 

\section{Model Konseptual Mekanisme \textit{EA Impact Analysis} (\textit{To-Be})}
\subsection{Prinsip dan Acuan Perancangan (Mengacu pada TOGAF)}
Penyusunan mekanisme \textit{EA Impact Analysis} (\textit{To-Be}) pada Paragon Corp mengacu pada prinsip TOGAF yaitu pada bagian \textit{Architecture Development Method} (ADM).
Gambar \ref{gambar:adm} menunjukkan siklus \textit{Architecture Development Method} (ADM) pada TOGAF.
TOGAF ADM menjelaskan bahwa analisis dan pengembangan arsitektur dilakukan dalam empat fase inti yaitu \textit{Business Architecture (\textit{Phase B}), Application Architecture and Data Architecture (\textit{Phase C}),} dan \textit{Technology Architecture (\textit{Phase D})}. 
Sementara itu, \textit{Security Architecture}, TOGAF tidak menempatkannya pada fase khusus, melainkan sebagai \textit{concern} yang harus dianalisis pada setiap fase.
Pada kondisi \textit{To-Be}, \textit{domain} arsitektur Paragon akan melakukan analisis sesuai elemen yang dideskripsikan pada fase-fase ADM tersebut.

Selain itu, TOGAF menyediakan \textit{Content Model} yang mendefinisikan komponen arsitektur pada setiap \textit{domain}.
Mekanisme \textit{To-Be} menggunakan metamodel ini sebagai acuan untuk menentukan elemen mana yang harus dianalisis oleh setiap \textit{domain} supaya hasil analisis menjadi terstruktur.

TOGAF menekankan bahwa setiap perubahan harus melalui \textit{architecture impact assessment} untuk menilai risiko, dependensi, dan dampak arsitektural.
Proses ini tercantum pada bagian \textit{Preliminary Phase} dan \textit{phase} B hingga \textit{phase} D.
Pada \textit{To-Be}, prinsip ini diimplementasikan dengan analisis formal pada \textit{domain} arsitektur dan penilaian berbasis TOGAF.

\subsection{Model Konseptual \textit{EA Impact Analysis}}
Model \textit{EA Impact Analysis} (\textit{To-Be}) menggambarkan bagaimana proses analisis dampak arsitektur dilakukan secara formal dengan mengacu pada elemen-elemen TOGAF dan tanpa mengubah alur tata kelola yang berlaku di Paragon.
Proses \textit{EA Impact Analysis} diisi pada dokumen URS saat disepakatinya suatu inisiatif.
Setiap \textit{domain} melakukan analisis terhadap elemen arsitektur berdasarkan fase ADM dan \textit{Content Metamodel} TOGAF.
\begin{enumerate}
    \item \textit{Business Architecture} (\textit{Phase B})\\
    \textit{Domain} ini menilai dampak perubahan terhadap \textit{business process, business capability, business service,} dan \textit{role} yang terdampak. \\
  
    \item \textit{Application Architecture} (\textit{Phase C})\\
    \textit{Domain} ini menilai dampak perubahan terhadap \textit{application component, application service, information flow} dan \textit{integrasi} aplikasi. \\

    \item \textit{Data Architecture} (\textit{Phase C})\\
    Analisis pada \textit{domain} ini dibagi menjadi tiga struktur yaitu \textit{conceptual, logical,} dan \textit{technical}.
    Pada \textit{conceptual} didefinisikan entitas konseptual yang diperlukan untuk mendukung proses bisnis, seperti \textit{employee, customer,} atau \textit{order} berdasarkan kebutuhan informasi terkait.
    Pada \textit{logical} menggambarkan struktur data dan alur perpindahan data antar aplikasi atau proses.
    Sementara itu, pada \textit{technical} menjelaskan teknologi penyimpanan dan \textit{platform} data yang digunaka. \\

    \item \textit{Technology Architecture} (\textit{Phase D})\\
    \textit{Domain} ini menilai dampak perubahan terhadap \textit{technology component, platform service} dan aspek non-fungsional (\textit{availability, scalability, performance}). \\

    \item \textit{Security Architecture} (\textit{Architecture Concern})\\
    \textit{Security} dianalisis sebagai \textit{concern} lintas domain. 
    Hal yang dianalisis yaitu  \textit{Role-Based Access Control} (RBAC), mekanisme autentikasi, perlindungan data, dan kebutuhan kepatuhan keamanan. \\
\end{enumerate}

Setelah melakukan analisis, semua hasil akan digabungkan ke satu dokumen yang memuat rangkuman dampak bisnis, aplikasi, data, teknologi, dan keamanan.

\section{Perbandingan (\textit{As-Is}) dan (\textit{To-Be})}
Perbandingan antara mekanisme \textit{EA Impact Analysis} pada kondsi (\textit{As-Is}) dan (\textit{To-Be}) dilakukan untuk menampilkan perubahan konseptual yang terjadi pada proses analisis dampak arsitektur.
Perubahan yang dijelaskan tidak mengubah alur tata kelola, karena perubahan hanya berfokus kepada proses \textit{EA Impact Analysis} agar lebih selaras dengan TOGAF dan kebutuhan Paragon.
Tabel \ref{tbl:perbandinganKondisi} perbedaan antara proses \textit{EA Impact Analysis} saat ini dan rancangan. \\
\begin{longtable}{@{\extracolsep{\fill}}
    >{\raggedright\arraybackslash}p{3cm}
    >{\raggedright\arraybackslash}p{4.5cm}
    >{\raggedright\arraybackslash}p{4.5cm}
}
\caption{Perbandingan Mekanisme EA Impact Analysis (AS-IS vs TO-BE)}
\label{tbl:perbandinganKondisi} \\
\toprule
\textbf{Aspek} &
\textbf{AS-IS} &
\textbf{TO-BE} \\
\midrule
\endfirsthead

\caption[]{Perbandingan Mekanisme EA Impact Analysis (AS-IS vs TO-BE) (lanjutan)} \\
\toprule
\textbf{Aspek} &
\textbf{AS-IS} &
\textbf{TO-BE} \\
\midrule
\endhead

\midrule
\multicolumn{3}{r}{\textit{Bersambung ke halaman berikutnya}} \\
\endfoot

\bottomrule
\endlastfoot

% ====================== ROWS ======================

Domain yang Terlibat &
Domain yang dianalisis yaitu \textit{Business, Application,} dan \textit{Data}.  
Sementara itu, untuk \textit{Technology} dan \textit{Security} tidak dianalisis. &
Seluruh domain dianalisis sesuai TOGAF. \\

Analisis \textit{Technology Architecture} &
Tidak disertakan dalam dokumen URS dan harus melihat LeanIX secara manual. &
Menjadi aktivitas wajib pada proses \textit{EA Impact Analysis}. \\

Analisis \textit{Security Architecture} &
Tidak disertakan dalam dokumen URS dan harus melihat LeanIX secara manual. &
\textit{Security} menjadi \textit{architecture concern} yang dianalisis pada proses \textit{EA Impact Analysis}. \\

Konsolidasi Hasil Analisis &
Tidak ada dokumen yang menyatukan seluruh hasil analisis &
Menghasilkan satu dokumen sebagai ringkasan dan rangkuman untuk keputusan hasil proses \textit{EA Impact Analysis}. \\

Peran Domain \textit{Architect} &
Peran masing-masing domain tidak didefinisikan dengan jelas. &
Domain \textit{architect} terdefinisi jelas. \\

Kelengkapan dan Akurasi &
Struktur dan proses masih belum jelas. &
Tervalidasi melalui struktur TOGAF ADM dan \textit{Content Metamodel} sehingga lebih terstruktur. \\

\end{longtable}


Gambar berikut memperlihatkan ilustrasi perbandingan proses \textit{EA Impact Analysis} pada kondisi (\textit{As-Is}) dan (\textit{To-Be}).
Gambar \ref{gambar:asis_EAImpact} menunjukkan proses \textit{EA Impact Analysis} pada kondisi (\textit{As-Is}).
Sementara itu \ref{gambar:tobe_EAImpact} menunjukkan proses \textit{EA Impact Analysis} pada kondisi (\textit{To-Be}).\\
\begin{figure}[H] % pilihan opsi yang disarankan: t = top, b = bottom, h = here
  \centering
  \captionsetup{justification=centering}
      \includegraphics[width=0.5\textwidth]{image/AsIsImpactAnalysis.png}
  \caption{Proses \textit{EA Impact Analysis} pada kondisi (\textit{As-Is})}
  \label{gambar:asis_EAImpact}
\end{figure}

\begin{figure}[H] % pilihan opsi yang disarankan: t = top, b = bottom, h = here
    \centering
    \captionsetup{justification=centering}
        \includegraphics[width=0.5\textwidth]{image/ToBeImpactAnalysis.png}
    \caption{Proses \textit{EA Impact Analysis} pada kondisi (\textit{To-Be})}
    \label{gambar:tobe_EAImpact}
  \end{figure}