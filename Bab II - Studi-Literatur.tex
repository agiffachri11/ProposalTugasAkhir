% ==========================================
% BAB II STUDI LITERATUR
% ==========================================
\chapter{STUDI LITERATUR}
\label{chap:studi-literatur}
\section{Konsep \textit{Enterprise Architecture} (EA)}

\subsection{Pengertian dan Tujuan \textit{Enterprise Architecture}}
% Contoh gambar dapat dilihat pada Gambar \ref{gambar:jaringan}. Gambar dan judulnya diposisikan di tengah. Nomor gambar tidak diakhiri tanda titik. Gambar tersebut dibuat menggunakan aplikasi draw.io dan disimpan ke format PNG setelah dengan zoom setting pada angka 300\%. Ukuran gambar yang ditampilkan dapat diatur dengan mengubah nilai \textit{width} dalam sintaks \textit{includegraphics}.
% Definisi dari \textit{Enterprise Architecture} (EA) antara lain sebagai berikut:
% \begin{enumerate}
%   \item Menurut \textcite{ross2006enterprise}, \textit{enterprise architecture} adalah cara organisasi menghubungkan proses bisnis dengan infrastruktur Teknologi Informasi (TI) untuk memastikan integrasi dan standarisasi sesuai dengan model operasional perusahaan.
%   \item Menurut \textcite{ahlemann2012strategic}, \textit{enterprise architecture} adalah penetapan, pemeliharaan, dan penerapan serangkaian panduan, prinsip desain, dan aturan tata kelola yang terpadu sehingga struktur teknologi dan proses bisnis selalu selaras dan efektif dalam mencapai visi dan strategi jangka panjang perusahaan.
%   \item Menurut \textcite{luisi2014pragmatic}, \textit{enterprise architecture} adalah pengembangan kerangka kerja Teknologi Informasi (TI) yang memfasilitasi arah bisnis dan mengatasi kendala utama, dengan mewakili berbagai kepentingan pemangku kepentingan bisnis di seluruh perusahaan. \\
% \end{enumerate}

% Dari pengertian-pengertian tersebut dapat disimpulkan bahwa \textit{enterprise architecture} merupakan fondasi atau panduan tertentu yang memastikan Teknologi Informasi (TI) 
% dan cara kerja perusahaan atau proses bisnis saling terhubung untuk mendukung pencapaian tujuan strategis organisasi.

% Menurut \textcite{minoli2008enterprise}, sasaran EA adalah menciptakan lingkungan Teknologi Informasi (TI) yang terpadu. 
% Hal ini dicapai melalui standarisasi perangkat keras dan sistem perangkat lunak, mendorong penggunaan kembali aset TI, dan menerapkan metode yang sama dalam manajemen 
% proyek untuk pengembangan perangkat lunak di seluruh unit organisasi. Semua upaya ini dilakukan sambil tetap memastikan keselarasan penuh antara TI dengan strategi dan aspek bisnis organisasi. 
% Hasil yang diharapkan dari penerapan EA adalah menjadikan fungsi TI menjadi lebih hemat biaya, lebih strategis, dan lebih responsif terhadap kebutuhan bisnis.

% Tujuan dari EA adalah merancang sebuah \textit{roadmap} yang mencakup aset Teknologi Informasi (TI), proses bisnis, dan prinsip tata kelola. 
% \textit{Roadmap} ini harus mampu mendukung tercapainya strategi bisnis sekaligus menjelaskan secara rinci bagaimana TI akan digunakan untuk merealisasikan strategi tersebut \autocite{minoli2008enterprise}. 
% Tabel \ref{tbl:tujuanEA} menunjukkan persentase dari tujuan mengapa perusahaan membangun \textit{enterprise architecture}.

%     \begin{table}[t] % pilihan opsi yang disarankan: t = top, b = bottom, h = here
    \begin{tabular}{ | p{5cm} | p{2cm} |}
      \hline
      Tujuan Spesifik 	& Persentase Perusahaan \\
      \hline
      \textit{Business–IT alignment} 	& 20 \\
      \textit{Business change}	& 15	 \\
      \textit{Transformation roadmap}	& 15 \\
    \textit{Infrastructure renewal} 	& 12 \\
      \textit{Legacy transformation}	& 11	 \\
      \textit{ERP implementation}	& 11 \\
    \textit{Application renewal} 	& 10 \\
      \textit{Mergers/acquisition}	& 4	 \\
      \textit{Other}	& 2 \\
      \hline
      \end{tabular}
  \caption{Tujuan penerapan \textit{Enterprise Architecture} (EA) di perusahaan}
  \label{tbl:tujuanEA}
  \end{table}
\textit{Enterprise Architecture} (EA) adalah metode yang membantu organisasi memetakan strategi bisnis, proses, informasi, dan teknologi yang mendukungnya.
EA berfungsi sebagai kerangka yang memastikan proses bisnis dan Teknologi Informasi (TI) berjalan sesuai standar untuk mencapai tujuan operasional perusahaan \autocite{ross2006enterprise}.

\textcite{ahlemann2012strategic} menjelaskan EA merupakan suatu prinsip, aturan, dan tata kelola yang digunakan oleh organisasi untuk memastikan konsistensi arsitektur dalam mendukung tujuan perusahaan.
Definisi ini menegaskan bahwa EA bukan hanya sekedar dokumentasi, tetapi juga pendekatan manajemen agar tercapainya tujuan perusahaan.

Sementara itu, TOGAF menjelaskan EA sebagai metode dalam merancang, mengembangkan, dan mengelola arsitektur bisnis, aplikasi, data, dan teknologi.
EA berfungsi sebagai peta yang menggambarkan bagaimana organisasi terhubung dan berkembang untuk mendukung strategi perusahaan \autocite{TOGAF_Standard_10th}.

Dari berbagai definisi tersebut, disimpulkan bahwa EA berperan sebagai landasan arsitektur yang menghubungkan strategi perusahaan dan operasional secara terstruktur.
Dengan demikian, EA dapat meningkatkan efisiensi operasional, mengurangi redundansi aplikasi, dan memastikan Teknologi Informasi (TI) tetapi selaras dengan kebutuhan bisnis.

\subsection{Domain Utama \textit{Enterprise Architecture}}
\textit{Enterprise Architecture} (EA) dibentuk oleh beberapa kompoenen yang disebut domain arsitektur.
Domain ini mengkategorikan aspek yang harus diperhatikan oleh arsitek untuk merancang sistem yang kompleks \autocite{jager2023getting}.

Dalam TOGAF, EA terbagi menjadi empat domain arsitektur yang berfungsi sebagai subsistem dari keleuruhan arsitektur perusahaan.
Pembagian domain ini menjadi struktur utama dalam \textit{Architecture Development Method} (ADM) \autocite{TOGAF_Standard_10th}.
Gambar \ref{gambar:adm} menunjukkan siklus \textit{Architecture Development Method} (ADM) yang digunakan untuk mengembangkan, mengelola, dan memelihara siklus EA pada organisasi.
Sementara itu, Gambar \ref{gambar:domain} menunjukkan struktur domain arsitektur EA.

\begin{figure}[h] % pilihan opsi yang disarankan: t = top, b = bottom, h = here
	\centering
  \captionsetup{justification=centering}
    	\includegraphics[width=0.7\textwidth]{image/adm.png}
	\caption{Siklus \textit{Architecture Development Method} (ADM)}
	\label{gambar:adm}
\end{figure}

\begin{figure}[t] % pilihan opsi yang disarankan: t = top, b = bottom, h = here
	\centering
  \captionsetup{justification=centering}
    	\includegraphics[width=0.7\textwidth]{image/domainlayer.png}
	\caption{Domain Arsitektur \textit{Layer} EA}
	\label{gambar:domain}
\end{figure}

\begin{enumerate}
    \item \textit{Business Architecture} \\
    Domain ini terletak pada \textit{phase} B dari \textit{Architecture Development Method} (ADM) yang menjelaskan tentang strategi, tujuan, kapabilitas, dan proses bisnis organisasi. 
    Domain ini menggambarkan cara organisasi untuk mencapai nilai yang diberikan kepada pelanggan.\\

    \item \textit{Information Systems Architectures} \\
    Domain ini berperan dalam menjembatani kebutuhan bisnis dengan Teknologi Informasi (TI) yang mendukungnya, dan dipecah menjadi dua sub-komponen utama.
    \begin{enumerate}[a.]
        \item \textit{Data Architecture} \\
        Domain ini terletak pada \textit{phase} C dari \textit{Architecture Development Method} (ADM) yang menjelaskan struktur dan jenis data yang dapat mendukung proses bisnis secara efektif. 

        \item \textit{Application Architecture} \\
        Domain ini terletak pada \textit{phase} C dari \textit{Architecture Development Method} (ADM) yang menjelaskan jenis aplikasi aplikasi yang digunakan untuk mendukung proses bisnis. \\
    \end{enumerate}

    \item \textit{Technology Architecture} \\
    Domain ini terletak pada \textit{phase} D dari \textit{Architecture Development Method} (ADM) yang menjelaskan infrastruktur Teknologi Informasi (TI) yang diperlukan untuk menjalankan aplikasi dan mengelola data perusahaan. 
\end{enumerate}

\subsection{Peran \textit{Enterprise Architecture} dalam Transformasi Digital}
Menurut \textcite{josey2017togaf}, \textit{Enterprise Architecture} (EA) berperan dalam melakukan transformasi digital melalui metode \textit{Architecture Development Method} (ADM).
Selain itu EA membantu perusahaan dalam menata ulang proses bisnis dan Teknologi Informasi (TI) melalui proses tata kelola yang disediakan TOGAF.

\textcite{vandewetering2021dynamicenterprisearchitecturecapabilities} menjelaskan bahwa penerapan EA akan membantu perusahaan dalam meningkatkan kegiatan operasional dan mendukung keberhasilan transformasi digital.
Kapabilitas EA seperti kemampuan adaptif dan integrasi antar domain akan membantu perusahaan terhadap inovasi proses dan keselarasan bisnis dan Teknologi Informasi (TI).

Walaupun EA memberikan kerangka arsitektur perusahaan, penerapannya tidak akan efektif tanpa mekanisme tata kelola arsitektur.
TOGAF menjelaskan bahwa tata kelola arsitektur diperlukan supaya pengembangan dan implementasi arsitektur berjalan secara konsisten dan terkontrol \autocite{TOGAF_Standard_10th}.
Oleh karena itu, perusahaan harus memiliki tata kelola arsitektur agar EA dapat tetap berjalan dan mendukung tujuan strategis perusahaan

\section{\textit{Enterprise Architecture Governance}}
\subsection{Definisi dan Tujuan \textit{Enterprise Architecture Governance}}
Menurut \textcite{korhonen2009ea}, \textit{EA Governance} adalah mekanisme kontrol yang berorientasi ke masa depan yang bertujuan untuk merancang kondisi arsitektur perusahaan di masa depan demi mendukung strategi bisnis. 
\textit{EA Governance} menekankan pada perencanaan dan efektivitas eksternal organisasi secara keseluruhan.
Tujuan \textit{EA Governance} adalah memastikan pengembangan arsitektur bersifat sistemik dan mengarah pada tujuan strategis jangka panjang.

TOGAF menjelaskan bahwa \textit{EA Governance} merupakan proses yang digunakan untuk mengelola dan mengawasi pengembangan implementasi arsitektur perusahaan supaya sesuai dengan tujuan perusahaan \autocite{TOGAF_Standard_10th}.
\textcite{foorthuis2016theory} menambahkan bahwa \textit{EA Governance} diperlukan supaya arsitektur dijalankan secara konsisten dalam setiap proyek dan keputusan arsitektur dapat dipertanggungjawabkan.
Praktik EA akan efektif ketika perusahaan memiliki mekanisme pengawasan, kontrol terhadap kepatuhan, dan forum resmi dalam mengambil keputusan.

\subsection{Komponen Utama \textit{Enterprise Architecture Governance}}
Menurut TOGAF, \textit{EA Governance} memiliki komponen-komponen utama sebagai berikut \autocite{TOGAF_Standard_10th}:
\begin{enumerate}
  \item Struktur Tata Kelola \\
  Struktur tata kelola menjelaskan tentang peran formal seperti \textit{architecture board}.
  TOGAF menjelaskan bahwa \textit{architecture board} merupakan komponen penting dalam tata kelola yang berfungsi dalam menjaga konsistensi dan pengawasan arsitektur pada perusahaan. \\

  \item Proses Tata Kelola \\
  Menurut TOGAF, proses tata kelola berfungsi untuk mengidentifikasi, mengelola, dan menyebarluaskan informasi yang berkaitan dengan arsitektur.
  Komponen \textit{governance process} yang dijelaskan dalam TOGAF meliputi: 
  \begin{enumerate}[a.]
    \item \textit{Policy Management and Take-On} \\
    Proses ini mengatur perubahan arsitektur dan artefak yang baru dipublikasikan.

    \item \textit{Compliance} \\
    TOGAF menyatakan bahwa kepatuhan terhadap standar harus dinilai secara konsisten untuk memastikan kesesuaian arsitektur.

    \item \textit{Dispensation (Waiver)} \\
    TOGAF menyediakan mekanisme \textit{waiver} jika desain atau teknologi tidak dapat memenuhi standar.

    \item \textit{Monitoring and Reporting} \\
    Proses ini memastikan arsitektur dipantau berdasarkan kriteria yang telah disepakati.

    \item \textit{Business Control and Environment Management} \\
    Proses ini terkait dengan pengelolaan dokumen, arsitektur, dan pengelolaan informasi.
  \end{enumerate}

  \textcite{foorthuis2016theory} juga menekankan bahwa \textit{governance process} berperan penting dalam mencegah ketidakkonsistenan arsitektur.\\

  \item Artefak Arsitektur \\
  TOGAF menegaskan bahwa semua artefak arsitektur harus dikelola melalui \textit{governance process}.
  Artefak ini berfungsi sebagai pedoman bagi seluruh tim dalam mengembangkan solusi Teknologi Informasi (TI) yang konsisten.\\
\end{enumerate}

\subsection{Implementasi \textit{Enterprise Architecture Governance}}
Menurut TOGAF, implementasi tata kelola diatur dalam \textit{phase} G dari \textit{Architecture Development Method} (ADM).
Pada bagian ini memastikan bahwa proyek implementasi mengikuti arsitektur yang telah disetujui dan mematuhi standar perusahaan \autocite{TOGAF_Standard_10th}.
Langkah-langkah dalam fase G adalah sebagai berikut:
\begin{enumerate}
  \item Konfirmasi Ruang Lingkup dan Prioritas untuk \textit{Deployment} \\
  Pada tahap awal, ruang lingkup implementasi dan prioritas dikonfirmasi kembali seperti meninjau \textit{roadmap}, dependensi, dan \textit{gap analysis} arsitektur saat ini dan arsitektur target. \\

  \item Mengidentifikasi Sumber Daya \\
  Perusahaan perlu mengidentifikasi sumber daya dan kompetensi yang diperlukan untuk implementasi.
  TOGAF menekankan perlu kesiapan tim proyek, kesesuaian metode, dan terjalinnya komunikasi antar unit.\\

  \item Memandu Pengembangan \textit{Deployment} \\
  Pada proses ini, arsitek memberikan arahan selama implementasi, termasuk standar, batasan teknis, dan model integrasi.\\

  \item Meninjau Kepatuhan Arsitektur \\
  Selama implementasi, peninjauan kepatuhan dilakukan untuk memastikan desain dan hasil sesuai dengan arsitektur yang telah ditetapkan.\\

  \item Implementasi Proses Bisnis dan Teknologi Informasi \\
  Solusi yang dibangun akan dijalankan pada lingkungan operasional yang akan diikuti dokumentasi dan \textit{updating baseline architecture} di \textit{repository} perusahaan.\\

  \item Meninjau \textit{Post Implementation} dan \textit{Close Implementation} \\
  Setelah sistem diluncurkan, TOGAF mengharuskan tinjauan untuk menilai pencapaian tujuan arsitektur dan memperbarui artefak arsitektur.
  Setelah itu proyek ditutup secara formal.
\end{enumerate}

\subsection{Tantangan Penerapan \textit{Enterprise Architecture Governance}}
Menurut \textcite{korhonen2009ea}, tantangan penerapan \textit{EA governance} antara lain sebagai berikut:
\begin{enumerate}
  \item Konsep \textit{EA Governance} belum didefinisikan secara memadai yang menyulitkan implementasi konsisten dan terarah.
  \item Minimnya keterlibatan sisi bisnis sehingga potensi EA di perusahaan tidak terealisasi.
  \item Proses manajemen yang digunakan dalam \textit{IT Governance} tidak memadai untuk \textit{EA Governance} yang bersifat holistik dan strategis.
  \item Kurangnya badan tata kelola perantara \textit{(intermediating governance body)} antara \textit{Chief Enterprise Architect} dan Eksekutif Bisnis.
\end{enumerate}

\section{\textit{Enterprise Architecture Maturity Model (EAMM)}}
\subsection{Pengertian dan Tujuan \textit{Enterprise Architecture Maturity Model (EAMM)}}
Menurut \textcite{jager2023getting}, \textit{Enterprise Architecture Maturity Model} (EAMM) adalah kerangka yang digunakan untuk menilai sejauh mana penerapan EA dalam organisasi
telah berjalan secara efektif dan berkesinambungan. Model ini membantu organisasi memahami posisi mereka dalam perjalanan pengembangan arsitektur perusahaan, serta memberikan panduan untuk meningkatkan kualitas tata kelola dan implementasi EA secara bertahap.

Tujuan utama dari EAMM adalah untuk menyediakan alat ukur yang sistematis dalam mengevaluasi kemampuan organisasi dalam mengelola proses arsitektur, sekaligus mengidentifikasi area yang perlu ditingkatkan. 
Dengan memahami tingkat kematangan EA, organisasi dapat merencanakan langkah perbaikan yang lebih terarah, memastikan keselarasan antara strategi bisnis dan teknologi informasi, serta meningkatkan nilai tambah EA terhadap pengambilan keputusan manajerial \autocite{jager2023getting}.

\subsection{Tingkatan \textit{Maturity Model}}
Berdasarkan \textit{TOGAF Architecture Maturity Models}, penilaian EAMM dilakukan untuk memahami sejauh mana kapabilitas arsitektur telah diterapkan, dikelola, dan diintegrasikan dalam organisasi \autocite{TOGAF_Standard_10th}.
Tabel \ref{tbl:EAMM} menunjukkan menyajikan \textit{maturity level} pada masing-masing dimensi sehingga memberikan gambaran komprehensif mengenai posisi kapabilitas arsitektur saat ini serta arah peningkatan yang diperlukan.

\begin{landscape}
\begin{longtable}{@{\extracolsep{\fill}}
    p{2.5cm}
    p{2.3cm}
    p{2.3cm}
    p{2.3cm}
    p{2.3cm}
    p{2.3cm}}
\caption{\textit{Enterprise Architecture Maturity Model}}
\label{tbl:EAMM} \\
\toprule
\textbf{Dimensi} &
\textbf{Level 1 (\textit{Initial})} &
\textbf{Level 2 (\textit{Development})} &
\textbf{Level 3 (\textit{Defined})} &
\textbf{Level 4 (\textit{Managed})} &
\textbf{Level 5 (\textit{Measured})} \\
\midrule
\endfirsthead

\caption{\textit{Enterprise Architecture Maturity Model} (lanjutan)} \\
\toprule
\textbf{Dimensi} &
\textbf{Level 1 (\textit{Initial})} &
\textbf{Level 2 (\textit{Development})} &
\textbf{Level 3 (\textit{Defined})} &
\textbf{Level 4 (\textit{Managed})} &
\textbf{Level 5 (\textit{Measured})} \\
\midrule
\endhead

\midrule
\multicolumn{6}{r}{\textit{Bersambung ke halaman berikutnya}} \\
\endfoot

\bottomrule
\endlastfoot

% ========== ROWS ==========

\textbf{\textit{Architecture Process}} &
Proses arsitektur masih \textit{ad-hoc}, tidak konsisten, dan bergantung pada individu. &
Proses dasar telah terdokumentasi dan peran mulai ditetapkan. &
Proses arsitektur terdefinisi, dikomunikasikan, dan konsisten; Sudah ada analisis \textit{gap} dan rencana migrasi. &
Proses menjadi budaya organisasi; kualitas proses diukur. &
Proses terus diperbaiki secara berkelanjutan menggunakan metrik. \\

\textbf{\textit{Architecture Development}} &
Dokumentasi dan standar masih sporadis dan tidak terpadu. &
TRM dan \textit{standards profile} lengkap; \textit{gap analysis} dan rencana migrasi diterapkan di seluruh \textit{domain}. &
TRM dan \textit{standards profile} lengkap; analisis kesenjangan dan rencana migrasi selesai. &
Dokumentasi diperbarui rutin; seluruh domain mengikuti standar formal. &
Ada proses standar dan mekanisme pengecualian untuk meningkatkan kualitas pengembangan. \\

\textbf{\textit{Business Alignment}} &
Hubungan dengan strategi bisnis minim dan tidak terdokumentasi. &
Keterkaitan EA dan strategi bisnis dijelaskan eksplisit. &
EA terintegrasi dengan perencanaan investasi dan pengendalian proyek. &
Keputusan investasi diperbarui berdasarkan masukan EA; \textit{business drivers} ditinjau berkala. &
Metrik EA mengoptimalkan hubungan bisnis; unit bisnis aktif dalam perbaikan berkelanjutan. \\

\textbf{\textit{Organization}} &
Manajemen belum terlibat; partisipasi unit rendah. &
Manajemen mulai menyadari pentingnya EA; beberapa unit terlibat. &
Manajemen mendukung penuh; seluruh unit terlibat secara berkelanjutan dan kolaboratif. &
Manajemen meninjau langsung EA; seluruh unit terlibat konsisten. &
Semua unit memberi umpan balik rutin; manajemen mendorong peningkatan berkelanjutan. \\

\textbf{\textit{Architecture Governance}} &
Tidak ada tata kelola; kepatuhan sangat rendah. &
Tata kelola mulai diterapkan pada beberapa standar. &
Tata kelola terdokumentasi dan mencakup mayoritas investasi TI; Terdapat mekanisme pengecualian (\textit{waiver}). &
Tata kelola menyeluruh; pengelolaan deviasi terintegrasi dengan EA. &
Tata kelola matang; mekanisme waiver mendukung peningkatan berkelanjutan; tidak ada investasi TI tak terencana. \\

\textbf{\textit{Architecture Communication}} &
Dokumentasi ada, tetapi komunikasi terbatas dan lokal. &
Media komunikasi EA diperbarui dan digunakan menyimpan dokumen arsitektur. &
Dokumentasi arsitektur diperbarui periodik dan dikomunikasikan ke TI dan bisnis secara rutin. &
Dokumentasi dan proses komunikasi mengikuti perkembangan terbaru. &
Dokumen EA digunakan oleh seluruh pengambil keputusan; komunikasi menjadi mekanisme utama EA. \\

\end{longtable}

\end{landscape}

\section{\textit{Best Practice} dan Studi Terkait}
\subsection{SAP LeanIX sebagai \textit{Platform} Pendukung Implementasi \textit{Enterprise Architecture}}
SAP LeanIX merupakan \textit{platform} manajemen arsitektur perusahaan \textit{Enterprise Architecture Management} (EAM) yang digunakan secara luas dalam mengoptimalkan pengelolaan aplikasi, proses bisnis, dan infrastruktur TI organisasi.
\textit{Platform} ini bertindak sebagai tempat utama untuk mendokumentasikan dan menganalisis arsitektur, sehingga membantu membuat keputusan strategis jadi lebih efektif. Beberapa fitur utama yang ditawarkan LeanIX meliputi \textit{Application Portfolio Management, 
Interface Catalog}, dan \textit{Fact Sheet}. Fitur-fitur ini membantu perusahaan memetakan serta mengatur aplikasi yang mereka gunakan, menemukan keterkaitan antara berbagai sistem, dan mencatat detail krusial tentang aset teknologi serta proses bisnis dengan cara yang terorganisir \autocite{LeanixEAGov}.

Selain fungsi manajemen aplikasi, SAP LeanIX juga mendukung praktik \textit{EA Governance} dengan menyediakan alat pengawasan, proses standarisasi, serta pembuatan kebijakan dan prosedur sesuai kebutuhan organisasi.
Perusahaan juga dapat membangun kerangka kerja \textit{governance} yang adaptif, menetapkan aturan pengelolaan perubahan arsitektur, serta memastikan setiap proyek IT selaras dengan tujuan bisnis \autocite{LeanixEAGov}.

\subsection{Studi Kasus Implementasi \textit{EA Governance}}
Fungsi-fungsi utama SAP LeanIX telah banyak diaplikasikan dalam berbagai studi kasus di industri dan sektor publik guna meningkatkan praktik \textit{EA Governance}. Salah satu contoh di perusahaan Reckitt, SAP LeanIX digunakan untuk mendukung migrasi aplikasi menuju \textit{cloud} dan memfasilitasi sinergi lintas unit bisnis, 
sehingga proses digitalisasi lebih efisien dan terukur \autocite{Christ2021LeanixReckitt}. Studi pada merger dua perusahaan asuransi besar, Helvetia dan Nationale Suisse, memperlihatkan LeanIX efektif dalam mempercepat penggabungan proses bisnis dan sistem teknologi, 
melalui integrasi \textit{governance} serta pelaporan digital yang lebih transparan \autocite{Mone2018Helvetia}.

\subsection{Analisis Temuan Penelian Terdahulu}
Hasil penelitian \textcite{Hanafi2023PeranEA} menegaskan bahwa pelaksanaan \textit{EA Governance} mampu memberikan dampak positif yang nyata terhadap peningkatan kinerja organisasi pemerintahan daerah. 
Praktik \textit{EA Governance} yang dijalankan dengan baik berpengaruh pada penyusunan struktur organisasi yang rapi, memperlancar proses bisnis, serta meningkatkan kelincahan institusi dalam menghadapi perubahan lingkungan eksternal. Penelitian tersebut juga menunjukkan bahwa faktor kepemimpinan, bila diaplikasikan sebagai variabel moderasi, dapat memperkuat hubungan antara tata kelola EA dan efektivitas organisasi pemerintah.

Dalam penelitian tersebut memiliki kekurangan yang relevan sebagai dasar bagi tugas akhir ini seperti sebagai berikut: 
\begin{enumerate}
  \item Analisis terkait kondisi nyata praktik \textit{EA Governance} masih sangat minim.
  \item Penelitian belum berfokus kepada perancangan model perbaikan \textit{EA Governance} yang menyeluruh.
\end{enumerate}