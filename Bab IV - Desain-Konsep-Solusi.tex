% ==========================================
% BAB IV DESAIN KONSEP SOLUSI
% ==========================================
\chapter{DESAIN KONSEP SOLUSI}
\label{chap:desain-konsep-solusi}
\section{Gambaran Umum Solusi}
Berdasarkan kondisi aktual yang telah dijelaskan dalam Bab III, Paragon Corp menerapkan alur tata kelola AE yang terdiri dari 5 tahap, yaitu \textit{Ideation, Risk and Impact, Development, UAT and Go Live} dan \textit{Hypercare}.
Dalam alur ini, proses \textit{EA Impact Analysis} dilakukan pada tahap \textit{Risk and Impact} dan finalisasi arsitektur pada tahap \textit{UAT and Go Live}.
Mengacu ke referensi TOGAF, nantinya proses \textit{EA Impact Analysis} akan dilakukan dan langsung finalisasi pada tahap \textit{Risk and Impact}.
Sementara itu, tahapan \textit{Development} akan dimulai setelah seluruh arsitektur telah melakukan analisis.
Perubahan ini mengacu ke TOGAF yang menyatakan bahwa analisis dan keputusan arsitektur dilakukan sebelum perusahaan memasuki fase implementasi.

SAP LeanIX akan menjadi komponen kunci dalam solusi sebagai alat dokumentasi, repositori resmi, sumber untuk menganalisis dampak, dan pendukung visualisasi hubungan antar \textit{domain}.
Peran LeanIX dalam model solusi ini selaras dengan TOGAF yang mengatur bahwa \textit{Architecture Repository} harus berfungsi sebagai penyimpanan artefak arsitektur \autocite{TOGAF_Standard_10th}.

Solusi yang diusulkan yaitu SOP \textit{EA Impact Analysis} yang mencakup:
\begin{enumerate}
    \item Standarisasi proses \textit{EA Impact Analysis} pada tahap proses tata kelola AE saat ini.
    \item Seluruh artefak setiap \textit{domain} akan di finalisasi pada tahapan \textit{Risk and Impact} agar selaras dengan prinsip TOGAF.
    \item Penguatan peran SAP LeanIX sebagai repositori tunggal.
    \item Pemetaan proses ke tahapan tata kelola AE Paragon Corp agar solusi dapat dibandingkan dengan kondisi saat ini.
    \item Pengembangan model konseptual SOP.
  \end{enumerate}

\section{Desain Konsep \textit{SOP EA Impact Analysis}}
xx

\section{Perbandingan Sistem Saat Ini dan Desain Konsep Solusi}
xx