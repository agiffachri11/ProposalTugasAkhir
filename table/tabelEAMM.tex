\begin{longtable}{@{\extracolsep{\fill}}
    p{2.5cm}
    p{2.3cm}
    p{2.3cm}
    p{2.3cm}
    p{2.3cm}
    p{2.3cm}}
\caption{\textit{Enterprise Architecture Maturity Model}}
\label{tbl:EAMM} \\
\toprule
\textbf{Dimensi} &
\textbf{Level 1 (\textit{Initial})} &
\textbf{Level 2 (\textit{Development})} &
\textbf{Level 3 (\textit{Defined})} &
\textbf{Level 4 (\textit{Managed})} &
\textbf{Level 5 (\textit{Measured})} \\
\midrule
\endfirsthead

\caption[]{\textit{Enterprise Architecture Maturity Model} (lanjutan)} \\
\toprule
\textbf{Dimensi} &
\textbf{Level 1 (\textit{Initial})} &
\textbf{Level 2 (\textit{Development})} &
\textbf{Level 3 (\textit{Defined})} &
\textbf{Level 4 (\textit{Managed})} &
\textbf{Level 5 (\textit{Measured})} \\
\midrule
\endhead

\midrule
\multicolumn{6}{r}{\textit{Bersambung ke halaman berikutnya}} \\
\endfoot

\bottomrule
\endlastfoot

% ========== ROWS ==========

\textbf{\textit{Architecture Process}} &
Proses arsitektur masih dilakukan secara spontan, tidak konsisten, dan tergantung individu. &
Proses dasar sudah ditulis dan mulai jelas. Selain itu, peran-peran sudah ditentukan. &
Proses arsitektur sudah dijelaskan dengan jelas, terdokumentasi, dan disosialisasikan. Selain itu sudah ada \textit{gap analysis} dan rencana migrasi menuju kondisi ideal. &
Proses menjadi kebiasaan organisasi dan kualitas proses sudah diukur.&
Proses terus diperbaiki menggunakan data dan metrik untuk meningkatkan hasil. \\

\textbf{\textit{Architecture Development}} &
Dokumen dan standar arsitektur belum menyatu. &
Sudah ada TRM dan \textit{standards profile}. Selain itu, \textit{gap analysis} dan rencana migrasi diterapkan untuk semua domain arsitektur. &
TRM dan \textit{standards profile} sudah lengkap. Selain itu, \textit{gap analysis} dan rencana migrasi juga sudah selesai dibuat. &
Dokumentasi arsitektur diperbarui secara rutin dan semua domain mengikuti standar yang sama. &
Menerapkan mekanisme pengecualian (\textit{waiver}) yang digunakan untuk menjaga dan meningkatkan kualitas pengembangan arsitektur. \\

\textbf{\textit{Business Alignment}} &
AE hampir tidak terhubung dengan strategi bisnis dan tidak terdokumentasi. &
Hubungan AE dan strategi bisnis mulai dijelaskan dengan jelas. &
AE sudah menjadi bagian dari proses perencanaan investasi dan pengendalian proyek. &
Keputusan investasi diperbarui berdasarkan masukan dari AE dan tujuan bisnis ditinjau secara berkala. &
AE memiliki metrik yang membantu menyelaraskan TI dan bisnis. Selain itu, unit bisnis terlibat aktif dalam perbaikan berkelanjutan.\\

\textbf{\textit{Organization}} &
Manajemen tidak terlibat dan partisipasi unit sangat rendah. &
Manajemen mulai memahami pentingnya AE dan beberapa unit mulai ikut terlibat. &
Manajemen memberikan dukungan penuh dan keterlibatan lintas unit berjalan secara berkelanjutan dan kolaboratif. &
Manajemen meninjau langsung AE dan seluruh unit terlibat konsisten. &
Semua unit memberi umpan balik secara rutin dan manajemen selalu mendorong peningkatan secara terus-menerus. \\

\textbf{\textit{Architecture Governance}} &
Tidak ada tata kelola dan kepatuhan sangat rendah. &
Tata kelola mulai diterapkan pada beberapa standar. &
Tata kelola sudah terdokumentasi dan mencakup sebagian besar investasi TI dan adanya mekanisme pengecualian resmi (\textit{waiver}). &
Tata kelola berlaku menyeluruh dan pengelolaan penyimpangan (deviasi) terintegrasi dengan AE. &
Tata kelola sudah matang dan penggunaan mekanisme \textit{waiver} mendukung peningkatan berkelanjutan. Selain itu, tidak ada investasi TI yang dilakukan tanpa perencanaan. \\

\textbf{\textit{Architecture Communication}} &
Dokumentasi tersedia tetapi komunikasi masih terbatas dan hanya dilakukan di lingkup kecil. &
Media komunikasi AE diperbarui dan digunakan untuk menyimpan dokumen arsitektur. &
Dokumentasi arsitektur diperbarui secara berkala dan dikomunikasikan secara rutin ke tim TI dan bisnis. &
Dokumentasi dan komunikasi mengikuti perkembangan terbaru dan diperbarui secara konsisten. &
Dokumen EA digunakan oleh semua pengambil keputusan dan komunikasi menjadi alat utama dalam proses EA. \\

\end{longtable}
