\begin{longtable}{@{\extracolsep{\fill}}
    >{\raggedright\arraybackslash}p{2.0cm}
    >{\raggedright\arraybackslash}p{5.0cm}
    >{\raggedright\arraybackslash}p{5.0cm}
}
\caption{Timeline Rencana Implementasi Mekanisme \textit{EA Impact Analysis}} 
\label{tbl:timeline} \\
\toprule
\textbf{Periode} &
\textbf{Aktivitas} &
\textbf{Output} \\
\midrule
\endfirsthead

\caption[]{Timeline Rencana Implementasi Mekanisme \textit{EA Impact Analysis} (lanjutan)} \\
\toprule
\textbf{Periode} &
\textbf{Aktivitas} &
\textbf{Output} \\
\midrule
\endhead

\midrule
\multicolumn{3}{r}{\textit{Bersambung ke halaman berikutnya}} \\
\endfoot

\bottomrule
\endlastfoot

% ====================== ROWS ======================

Minggu 1-3 &
Studi literatur mendalam mengenai TOGAF ADM, \textit{Content Metamodel}, dan \textit{Impact Assessment} untuk memastikan mekanisme selaras dengan standar TOGAF. &
Ringkasan referensi TOGAF relevan untuk \textit{EA Impact Analysis}. \\

Minggu 4-6 &
Perancangan mekanisme \textit{EA Impact Analysis} \textit{To-Be} meliputi alur proses, peran, \textit{input} dan \textit{output}, serta elemen analisis per domain. &
Rancangan mekanisme \textit{EA Impact Analysis To-Be}. \\

Minggu 7-9 &
Penyusunan \textit{template} analisis per domain sesuai TOGAF. &
Rancangan \textit{template} domain. \\

Minggu 10-11 &
Konsultasi dan \textit{expert review} dengan \textit{Enterprise Architect} untuk mendapatkan masukan terhadap mekanisme dan \textit{template}. &
Daftar masukan terkait mekanisme dan \textit{template}. \\

Minggu 12 &
Revisi mekanisme dan \textit{template} berdasarkan masukan hasil \textit{expert review}. &
Final rancangan mekanisme dan \textit{template} \textit{EA Impact Analysis}. \\

Minggu 13 &
Penyusunan evaluasi berupa \textit{checklist} verifikasi berbasis TOGAF dan daftar pertanyaan validasi. &
Hasil evaluasi (verifikasi dan validasi). \\

Minggu 14 &
Simulasi proses menggunakan satu contoh URS untuk menguji kelayakan mekanisme \textit{To-Be}. &
Hasil simulasi dan catatan perbaikan. \\

Minggu 15-16 &
Finalisasi laporan dan penyusunan lampiran mekanisme, \textit{template}, dan hasil evaluasi. &
Hasil laporan akhir dan lampiran lengkap. \\

\end{longtable}
