\begin{longtable}{@{\extracolsep{\fill}}
    >{\raggedright\arraybackslash}p{2.0cm}
    >{\raggedright\arraybackslash}p{5.0cm}
    >{\raggedright\arraybackslash}p{5.0cm}
}
\caption{Kebutuhan Non-Fungsional \textit{EA Impact Analysis}}
\label{tbl:KNF} \\
\toprule
\textbf{Kode} &
\textbf{Kebutuhan Non-Fungsional} &
\textbf{Deskripsi} \\
\midrule
\endfirsthead

\caption[]{Kebutuhan Non-Fungsional \textit{EA Impact Analysis} (lanjutan)} \\
\toprule
\textbf{Kode} &
\textbf{Kebutuhan Non-Fungsional} &
\textbf{Deskripsi} \\
\midrule
\endhead

\midrule
\multicolumn{3}{r}{\textit{Bersambung ke halaman berikutnya}} \\
\endfoot

\bottomrule
\endlastfoot

% ====================== ROWS ======================

KNF-01 &
Konsistensi Proses &
Proses \textit{EA Impact Analysis} harus dijalankan secara konsisten menggunakan standar yang sama di seluruh tim. \\

KNF-02 &
Kemudahan Pemahaman &
\textit{Template} dan panduan harus mudah dipahami oleh seluruh peran terkait. \\

KNF-03 &
Kepatuhan ke Standar Arsitektur \textit{Enterprise} &
Proses mengikuti standar arsitektur seperti TOGAF dan standar internal Paragon Corp. \\

KNF-04 &
\textit{Auditability} &
Setiap masukan, revisi, dan keputusan dalam \textit{EA Impact Analysis} harus tercatat sehingga dapat diaudit. \\

KNF-05 &
Integrasi LeanIX &
Proses harus terintegrasi dengan SAP LeanIX sebagai repositori utama arsitektur. \\

KNF-06 &
Reliabilitas Dokumentasi &
Dokumen \textit{EA Impact Analysis} harus tersimpan dengan aman dan tidak mudah hilang atau rusak. \\

KNF-07 &
Keamanan Informasi &
Data terkait perubahan arsitektur harus terlindungi dari akses tidak berwenang. \\

KNF-08 &
Standarisasi Terminologi &
Istilah arsitektural yang digunakan dalam seluruh domain harus seragam dan terdokumentasi. \\

KNF-09 &
Akurasi Informasi &
Setiap informasi yang dituangkan dalam \textit{EA Impact Analysis} harus akurat, mutakhir, dan mencerminkan kondisi arsitektur yang sebenarnya. \\

\end{longtable}
