\begin{longtable}{@{\extracolsep{\fill}}
    p{2.8cm}
    p{2.8cm}
    p{2.8cm}
    p{2.8cm}
    p{2.8cm}}
\caption{Identifikasi \textit{Gap} dan Faktor Penyebab pada Setiap Dimensi EA}
\label{tbl:GapEA} \\
\toprule
\textbf{Dimensi} &
\textbf{Indikator Level 3 (TOGAF)} &
\textbf{Kondisi Aktual} &
\textbf{Gap} &
\textbf{Faktor Penyebab} \\
\midrule
\endfirsthead

\caption[]{Identifikasi \textit{Gap} dan Faktor Penyebab pada Setiap Dimensi EA (lanjutan)} \\
\toprule
\textbf{Dimensi} &
\textbf{Indikator Level 3 (TOGAF)} &
\textbf{Kondisi Aktual} &
\textbf{Gap} &
\textbf{Faktor Penyebab} \\
\midrule
\endhead

\midrule
\multicolumn{5}{r}{\textit{Bersambung ke halaman berikutnya}} \\
\endfoot

\bottomrule
\endlastfoot

% =========================================================
% ROWS
% =========================================================

\textbf{\textit{Architecture Process}} &
Proses arsitektur sudah dijelaskan dengan jelas, terdokumentasi, dan disosialisasikan. Selain itu sudah ada \textit{gap analysis} dan rencana migrasi menuju kondisi ideal. &
Proses sudah disosialisasikan melalui dokumen dan SAP LeanIX. Namun, dalam pelaksanaannya, beberapa tim masih menjalankan proses dengan cara yang berbeda-beda. \textit{Gap analysis} juga sudah didokumentasikan. &
Penerapan proses belum konsisten di seluruh tim. &
Perbedaan cara kerja antar tim dan belum ada mekanisme kontrol penerapan proses. \\

\textbf{\textit{Architecture Development}} &
TRM dan \textit{standards profile} sudah lengkap. Selain itu, \textit{gap analysis} dan rencana migrasi juga sudah selesai dibuat. &
\textit{Domain} bisnis dan aplikasi mulai terdokumentasi sejak awal. Sementara itu, \textit{domain} data, infrastruktur, dan keamanan terdokumentasi sambil berjalannya suatu inisiatif. &
Dokumentasi belum dilakukan sejak tahap inisiasi untuk semua domain. &
Belum semua \textit{domain} memiliki kebiasaan dokumentasi awal. \\

\textbf{\textit{Business Alignment}} &
AE sudah menjadi bagian dari proses perencanaan investasi dan pengendalian proyek. &
\textit{Business Architect} aktif selaraskan kebutuhan TI dan bisnis. Namun, beberapa keputusan masih berdasar kebutuhan operasional. &
AE belum menjadi acuan wajib untuk seluruh proyek atau investasi TI. &
Belum ada aturan formal bahwa semua investasi TI harus melewati analisis AE. \\

\textbf{\textit{Organization}} &
Manajemen memberikan dukungan penuh dan keterlibatan lintas unit berjalan secara berkelanjutan dan kolaboratif. &
Struktur peran sudah ditetapkan dengan jelas. Namun, belum dilakukan secara proaktif tanpa dorongan konteks tertentu. &
Kolaborasi belum menjadi kebiasaan natural lintas unit. &
Kolaborasi lebih sering muncul pada konteks atau inisiatif berdampak besar. \\

\textbf{\textit{Architecture Governance}} &
Tata kelola sudah terdokumentasi dan mencakup sebagian besar investasi TI dan adanya mekanisme pengecualian resmi (\textit{waiver}). &
Forum \textit{Architecture Review Board} (ARB) belum berjalan secara formal.
Selain itu, kepatuhan terhadap standar masih bersifat himbauan, sehingga belum ada aturan wajib yang harus dipatuhi oleh semua tim. &
Tata kelola belum menyeluruh dan mekanisme \textit{waiver} belum tersedia. &
Ketiadaan ARB aktif, tidak ada aturan wajib, dan budaya kepatuhan belum kuat. \\

\textbf{\textit{Architecture Communication}} &
Dokumentasi arsitektur diperbarui secara berkala dan dikomunikasikan secara rutin ke tim TI dan bisnis. &
SAP LeanIX sudah digunakan sebagai sumber referensi utama untuk dokumen arsitektur, dan pembaruan dokumen mulai dilakukan secara rutin.&
Komunikasi EA belum merata ke seluruh unit. &
Penyebaran informasi masih dominan teknis dan belum ada mekanisme komunikasi terstruktur. \\

\end{longtable}
