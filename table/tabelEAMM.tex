\begin{longtable}{@{\extracolsep{\fill}}
    p{2.5cm}
    p{2.3cm}
    p{2.3cm}
    p{2.3cm}
    p{2.3cm}
    p{2.3cm}}
\caption{\textit{Enterprise Architecture Maturity Model}}
\label{tbl:EAMM} \\
\toprule
\textbf{Dimensi} &
\textbf{Level 1 (\textit{Initial})} &
\textbf{Level 2 (\textit{Development})} &
\textbf{Level 3 (\textit{Defined})} &
\textbf{Level 4 (\textit{Managed})} &
\textbf{Level 5 (\textit{Measured})} \\
\midrule
\endfirsthead

\caption{\textit{Enterprise Architecture Maturity Model} (lanjutan)} \\
\toprule
\textbf{Dimensi} &
\textbf{Level 1 (\textit{Initial})} &
\textbf{Level 2 (\textit{Development})} &
\textbf{Level 3 (\textit{Defined})} &
\textbf{Level 4 (\textit{Managed})} &
\textbf{Level 5 (\textit{Measured})} \\
\midrule
\endhead

\midrule
\multicolumn{6}{r}{\textit{Bersambung ke halaman berikutnya}} \\
\endfoot

\bottomrule
\endlastfoot

% ========== ROWS ==========

\textbf{\textit{Architecture Process}} &
Proses arsitektur masih \textit{ad-hoc}, tidak konsisten, dan bergantung pada individu. &
Proses dasar telah terdokumentasi dan peran mulai ditetapkan. &
Proses arsitektur terdefinisi, dikomunikasikan, dan konsisten; Sudah ada analisis \textit{gap} dan rencana migrasi. &
Proses menjadi budaya organisasi; kualitas proses diukur. &
Proses terus diperbaiki secara berkelanjutan menggunakan metrik. \\

\textbf{\textit{Architecture Development}} &
Dokumentasi dan standar masih sporadis dan tidak terpadu. &
TRM dan \textit{standards profile} lengkap; \textit{gap analysis} dan rencana migrasi diterapkan di seluruh \textit{domain}. &
TRM dan \textit{standards profile} lengkap; analisis kesenjangan dan rencana migrasi selesai. &
Dokumentasi diperbarui rutin; seluruh domain mengikuti standar formal. &
Ada proses standar dan mekanisme pengecualian untuk meningkatkan kualitas pengembangan. \\

\textbf{\textit{Business Alignment}} &
Hubungan dengan strategi bisnis minim dan tidak terdokumentasi. &
Keterkaitan EA dan strategi bisnis dijelaskan eksplisit. &
EA terintegrasi dengan perencanaan investasi dan pengendalian proyek. &
Keputusan investasi diperbarui berdasarkan masukan EA; \textit{business drivers} ditinjau berkala. &
Metrik EA mengoptimalkan hubungan bisnis; unit bisnis aktif dalam perbaikan berkelanjutan. \\

\textbf{\textit{Organization}} &
Manajemen belum terlibat; partisipasi unit rendah. &
Manajemen mulai menyadari pentingnya EA; beberapa unit terlibat. &
Manajemen mendukung penuh; seluruh unit terlibat secara berkelanjutan dan kolaboratif. &
Manajemen meninjau langsung EA; seluruh unit terlibat konsisten. &
Semua unit memberi umpan balik rutin; manajemen mendorong peningkatan berkelanjutan. \\

\textbf{\textit{Architecture Governance}} &
Tidak ada tata kelola; kepatuhan sangat rendah. &
Tata kelola mulai diterapkan pada beberapa standar. &
Tata kelola terdokumentasi dan mencakup mayoritas investasi TI; Terdapat mekanisme pengecualian (\textit{waiver}). &
Tata kelola menyeluruh; pengelolaan deviasi terintegrasi dengan EA. &
Tata kelola matang; mekanisme waiver mendukung peningkatan berkelanjutan; tidak ada investasi TI tak terencana. \\

\textbf{\textit{Architecture Communication}} &
Dokumentasi ada, tetapi komunikasi terbatas dan lokal. &
Media komunikasi EA diperbarui dan digunakan menyimpan dokumen arsitektur. &
Dokumentasi arsitektur diperbarui periodik dan dikomunikasikan ke TI dan bisnis secara rutin. &
Dokumentasi dan proses komunikasi mengikuti perkembangan terbaru. &
Dokumen EA digunakan oleh seluruh pengambil keputusan; komunikasi menjadi mekanisme utama EA. \\

\end{longtable}
