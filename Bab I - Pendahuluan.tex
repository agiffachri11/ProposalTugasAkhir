% ==========================================
% BAB I PENDAHULUAN
% ==========================================
\chapter{PENDAHULUAN}
\label{chap:pendahuluan}
% --- Latar Belakang ---
\section{Latar Belakang}
Transformasi digital merupakan salah satu faktor yang memengaruhi keberlangsungan organisasi.
Perkembangan Teknologi Informasi (TI) memaksa organisasi untuk menyesuaikan proses bisnis dan layanan.
Menurut \textcite{juraida2024enterprise}, \textit{Enterprise Architecture} (EA)  merupakan elemen yang berperan dalam mendukung transformasi digital
melalui penyelarasan antara strategi bisnis dan strategi Teknologi Informasi (TI) organisasi.

Aspek EA yang diperlukan salah satunya yaitu tata kelola yang jelas untuk mengelola arsitektur, mengambil keputusan, dan kepatuhan terhadap arsitektur yang dirancang.
Karakterisitik EA harus diintegrasikan dengan praktik manajemen tata kelola agar EA dapat dikelola secara berkelanjutan \autocite{nugroho2016enterprise}.

\textcite{virantina2020digital} menegaskan efektivitas EA sangat bergantung kepada kemampuan organisasi dalam mengelola siklus hidup arsitektur, termasuk bagaimana EA digunakan untuk mendukung proses transformasi digital, manajemen data, dan efisiensi operasional. 
Penerapan tata kelola EA juga memastikan pembagian peran, tanggung jawab, dan mekanisme evaluasi untuk memastikan keberlanjutan EA.

Untuk memastikan tata kelola EA berjalan efektif, TOGAF menyediakan \textit{Architecture Capability Maturity Model} (ACMM) yang digunakan untuk menilai tingkat kematangan kapabilitas arsitektur organisasi.
Model ini memiliki 5 tingkatan dan pada level 3 (\textit{defined}), proses tata kelola EA telah terdokumentasi, dilaksanakan secara konsisten, dan memiliki struktur peran yang jelas \autocite{TOGAF_Standard_10th}.
Dengan demikian, tingkat kematangan ini dijadikan acuan untuk menilai tata kelola EA Paragon Corp dan mengevaluasi dimensi yang perlu diperbaiki.

Penerapan EA memerlukan evaluasi terhadap seberapa efektif tata kelolanya.
Efektivitas ini bisa diketahui melalui tingkat kepatuhan proyek terhadap standar, dokumentasi yang dihasilkan, dan keterlibatan pemangku kepentingan \autocite{foorthuis2016theory}.
Cara ini selaras dengan TOGAF, yaitu pentingnya melakukan evaluasi terhadap kapabilitas arsitektur untuk meningkatkan tingkat kematangan tata kelola organisasi.

Pada praktik organisasi, tata kelola EA yang gagal menimbulkan banyak risiko.
Paragon Corp merupakan salah satu perusahaan yang memiliki alur tata kelola EA tetapi belum dijalankan secara formal. 
Meskipun tim \textit{Enterprise Architecture} di Paragon Corp baru dibentuk sejak tahun lalu, belum ada mekanisme rutin untuk \textit{architecture review} dan persetujuan arsitektur.
Kondisi ini menyebabkan duplikasi sistem dan ketiadaan standar arsitektural karena belum ada referensi terhadap arsitektur sebelumnya.
Evaluasi tata kelola EA menemukan bahwa aspek perancangan EA yang sudah didefinisikan, namun belum dikelola secara menyeluruh akan menghambat efektivitas EA sebagai kerangka strategis \autocite{ghiffari2022evaluasi}.

Dengan mempertimbangkan tantangan tersebut, tugas akhir ini berfokus kepada analisis perbaikan tata kelola EA sehingga meningkatkan efektivitas peran EA dalam mengambil keputusan di Paragon Corp. 

% --- Rumusan Masalah ---
\section{Rumusan Masalah}
Berdasarkan latar belakang masalah yang telah diuraikan di atas, rumusan masalah yang akan menjadi pokok pembahasan dalam pengerjaan tugas akhir ini adalah sebagai berikut:
\begin{enumerate}
\item	Bagaimana efektivitas tata kelola EA saat ini di Paragon Corp?
\item	Apa saja \textit{gap} yang perlu ditangani untuk mencapai tingkat kematangan level 3 tata kelola EA?
\item	Bagaimana rancangan tata kelola EA yang lebih terstruktur dan mampu memberikan dampak terhadap pengambilan keputusan strategis perusahaan?
\item   Bagaimana cara mengukur peningkatan efektivitas dan dampak penerapan EA setelah dilakukan perbaikan tata kelola?
\end{enumerate}

% --- Tujuan ---
\section{Tujuan}
Tujuan dari tugas akhir ini adalah sebagai berikut:
\begin{enumerate}
\item	Menganalisis kondisi tata kelola EA yang saat ini diterapkan di Paragon Corp.
\item	Mengidentifikasi \textit{gap} antara kondisi saat ini dengan \textit{best practice} EA pada ingkat kematangan level 3 tata kelola EA.
\item	Merancang model perbaikan tata kelola EA yang efektif dan sesuai dengan konteks organisasi.
\item   Mengevaluasi dampak penerapan model perbaikan terhadap efektivitas kinerja EA.
\end{enumerate}

% --- Metodologi Pengerjaan TA ---
\section{Metodologi}
Metodologi pengerjaan tugas akhir berikut ini menjelaskan tahapan-tahapan yang dilakukan selama proses penyusunan tugas akhir untuk menjawab rumusan masalah dan mencapai tujuan tugas akhir.
Pendekatan yang digunakan bersifat deskriptif kualitatif, dengan fokus pada analisis kondisi saat ini, identifikasi \textit{gap}, serta perancangan perbaikan model tata kelola EA yang sesuai dengan konteks organisasi Paragon Corp.

Secara umum, tahapan metodologi tugas akhir ini terdiri atas beberapa langkah berikut:
\begin{enumerate}
    \item {Tahap investigasi dan pengumpulan fakta} \\
    Tahap ini bertujuan untuk memahami konteks organisasi dan kondisi penerapan EA di Paragon Corp. Kegiatan yang dilakukan meliputi:
    \begin{enumerate}[a.]
        \item Studi internal terkait artefak EA, alur kerja, dan kebijakan perusahaan.
        \item Observasi terhadap penggunaan \textit{platform} SAP LeanIX dalam proses manajemen arsitektur.
        \item Wawancara dengan pihak terkait untuk mengidentifikasi praktik tata kelola dan pola kolaborasi yang berjalan saat ini. \\
    \end{enumerate}

    \item {Tahap studi literatur dan analisis teoretis} \\
    Pada tahap ini dilakukan pengumpulan, pengelompokan, dan penelaahan literatur yang relevan mengenai \textit{Enterprise Architecture}, \textit{EA Governance}, dan \textit{EA Maturity Model}. Literatur yang digunakan mencakup standar internasional seperti TOGAF, artikel ilmiah, serta studi kasus penerapan EA di industri sejenis. 
    Hasil analisis literatur akan dijelaskan secara sistematis pada Bab II – Studi Literatur sebagai landasan teoretis. \\

    \item {Tahap analisis kondisi saat ini dan identifikasi \textit{gap}} \\
    Berdasarkan hasil investigasi dan teori pendukung, dilakukan analisis terhadap efektivitas tata kelola EA saat ini di Paragon Corp. Tahap ini mencakup:
    \begin{enumerate}[a.]
        \item Penilaian tingkat kematangan EA menggunakan kerangka \textit{Enterprise Architecture Maturity Model (EAMM)}.
        \item Melakukan (\textit{gap analysis}) antara kondisi saat ini dan \textit{best practice} tata kelola EA. \\
    \end{enumerate}

    \item {Tahap perancangan model perbaikan \textit{EA Governance}} \\
    Pada tahap ini dirancang model perbaikan tata kelola EA yang lebih terstruktur, selaras dengan \textit{maturity level} yang ditargetkan, dan sesuai konteks organisasi. 
%    Model rancangan akan mencakup komponen peran dan tanggung jawab, proses kerja, serta mekanisme evaluasi dan pembaruan artefak EA. \\

    \item {Tahap evaluasi dan validasi model} \\
    Model yang dihasilkan kemudian dievaluasi untuk menilai kelayakan dan dampaknya terhadap efektivitas pengelolaan EA. Evaluasi dilakukan melalui:
    \begin{enumerate}[a.]
        \item \textit{Expert review} bersama \textit{stakeholder} internal Paragon Corp.
        \item Analisis perbandingan antara kondisi sebelum dan sesudah penerapan model secara konseptual. \\
    \end{enumerate}
\end{enumerate}