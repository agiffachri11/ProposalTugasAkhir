% ==========================================
% BAB I PENDAHULUAN
% ==========================================
\chapter{PENDAHULUAN}
\label{chap:pendahuluan}
% --- Latar Belakang ---
\section{Latar Belakang}
Transformasi digital merupakan salah satu faktor yang memengaruhi keberlangsungan organisasi.
Perkembangan Teknologi Informasi (TI) memaksa organisasi untuk menyesuaikan proses bisnis dan layanan.
Menurut \textcite{juraida2024enterprise}, Arsitektur \textit{Enterprise} (AE) merupakan elemen yang berperan dalam mendukung transformasi digital
melalui penyelarasan antara strategi bisnis dan strategi Teknologi Informasi (TI) organisasi.

Aspek AE yang diperlukan salah satunya yaitu tata kelola yang jelas untuk mengelola arsitektur, mengambil keputusan, dan kepatuhan terhadap arsitektur yang dirancang.
Karakterisitik AE harus diintegrasikan dengan praktik manajemen tata kelola agar AE dapat dikelola secara berkelanjutan \autocite{nugroho2016enterprise}.

\textcite{virantina2020digital} menegaskan efektivitas AE sangat bergantung kepada kemampuan organisasi dalam mengelola siklus hidup arsitektur, termasuk bagaimana AE digunakan untuk mendukung proses transformasi digital, manajemen data, dan efisiensi operasional. 
Penerapan tata kelola AE juga memastikan pembagian peran, tanggung jawab, dan mekanisme evaluasi untuk memastikan keberlanjutan AE.

Untuk memastikan tata kelola AE berjalan efektif, TOGAF menyediakan \textit{Architecture Capability Maturity Model} (ACMM) yang digunakan untuk menilai tingkat kematangan kapabilitas arsitektur organisasi.
Model ini memiliki 5 tingkatan dan pada level 3 (\textit{defined}), proses tata kelola AE telah terdokumentasi, dilaksanakan secara konsisten, dan memiliki struktur peran yang jelas \autocite{TOGAF_Standard_10th}.
Dengan demikian, tingkat kematangan ini dijadikan acuan untuk menilai tata kelola AE Paragon Corp dan mengevaluasi dimensi yang perlu diperbaiki.

Penerapan AE memerlukan evaluasi terhadap seberapa efektif tata kelolanya.
Efektivitas ini bisa diketahui melalui tingkat kepatuhan proyek terhadap standar, dokumentasi yang dihasilkan, dan keterlibatan pemangku kepentingan \autocite{foorthuis2016theory}.
Cara ini selaras dengan TOGAF, yaitu pentingnya melakukan evaluasi terhadap kapabilitas arsitektur untuk meningkatkan tingkat kematangan tata kelola organisasi.

Pada praktik organisasi, tata kelola AE yang gagal menimbulkan banyak risiko.
Paragon Corp merupakan salah satu perusahaan yang memiliki alur tata kelola AE tetapi belum dijalankan secara formal. 
Meskipun tim \textit{Enterprise Architecture} di Paragon Corp baru dibentuk sejak tahun lalu, belum ada mekanisme rutin untuk \textit{architecture review} dan persetujuan arsitektur.
Kondisi ini menyebabkan duplikasi sistem dan ketiadaan standar arsitektural karena belum ada referensi terhadap arsitektur sebelumnya.
Evaluasi tata kelola AE menemukan bahwa aspek perancangan AE yang sudah didefinisikan, namun belum dikelola secara menyeluruh akan menghambat efektivitas AE sebagai kerangka strategis \autocite{ghiffari2022evaluasi}.

Dengan mempertimbangkan tantangan tersebut, tugas akhir ini berfokus kepada analisis perbaikan tata kelola AE sehingga meningkatkan efektivitas peran AE dalam mengambil keputusan di Paragon Corp. 

% --- Rumusan Masalah ---
\section{Rumusan Masalah}
Berdasarkan latar belakang masalah yang telah diuraikan di atas, rumusan masalah yang akan menjadi pokok pembahasan dalam pengerjaan tugas akhir ini adalah sebagai berikut:
\begin{enumerate}
\item	Bagaimana alur tata kelola AE saat ini di Paragon Corp?
\item	Apa saja \textit{gap} yang perlu ditangani untuk mencapai tingkat kematangan level 3 tata kelola AE?
\item	Bagaimana rancangan tata kelola AE yang lebih terstruktur dan mampu memberikan dampak terhadap pengambilan keputusan strategis perusahaan?
\item   Bagaimana cara mengukur peningkatan efektivitas dan dampak penerapan AE setelah dilakukan perbaikan tata kelola?
\end{enumerate}

% --- Tujuan ---
\section{Tujuan}
Tujuan dari tugas akhir ini adalah sebagai berikut:
\begin{enumerate}
\item	Menganalisis kondisi tata kelola AE yang saat ini diterapkan di Paragon Corp.
\item	Mengidentifikasi \textit{gap} antara kondisi saat ini dengan \textit{best practice} AE pada tingkat kematangan level 3 tata kelola AE.
\item	Merancang model perbaikan tata kelola AE yang efektif dan sesuai dengan konteks organisasi.
\item   Mengevaluasi dampak penerapan model perbaikan terhadap efektivitas kinerja AE.
\end{enumerate}

% --- Metodologi Pengerjaan TA ---
\section{Metodologi}
Metodologi pengerjaan tugas akhir berikut ini menjelaskan tahapan-tahapan yang dilakukan selama proses penyusunan tugas akhir untuk menjawab rumusan masalah dan mencapai tujuan tugas akhir.
Pendekatan yang digunakan bersifat deskriptif kualitatif, dengan fokus pada analisis kondisi saat ini, identifikasi \textit{gap}, serta perancangan perbaikan model tata kelola AE yang sesuai dengan konteks organisasi Paragon Corp.

Secara umum, tahapan metodologi tugas akhir ini terdiri atas beberapa langkah berikut:
\begin{enumerate}
    \item {Tahap investigasi dan pengumpulan fakta} \\
    Tahap ini bertujuan untuk memahami konteks organisasi dan kondisi penerapan AE di Paragon Corp. Kegiatan yang dilakukan meliputi:
    \begin{enumerate}[a.]
        \item Studi internal terkait artefak AE, alur kerja, dan kebijakan perusahaan.
        \item Observasi terhadap penggunaan \textit{platform} SAP LeanIX dalam proses manajemen arsitektur.
        \item Wawancara dengan pihak terkait untuk mengidentifikasi praktik tata kelola dan pola kolaborasi yang berjalan saat ini. \\
    \end{enumerate}

    \item {Tahap studi literatur dan analisis teoretis} \\
    Pada tahap ini dilakukan pengumpulan, pengelompokan, dan penelaahan literatur yang relevan mengenai Arsitektur \textit{Enterprise}, Tata Kelola Arsitektur \textit{Enterprise}, dan \textit{EA Maturity Model}. 
    Literatur yang digunakan mencakup standar internasional seperti TOGAF dan artikel ilmiah. 
    Hasil analisis literatur akan dijelaskan secara sistematis pada Bab II – Studi Literatur sebagai landasan teoretis. \\

    \item {Tahap analisis kondisi saat ini dan identifikasi \textit{gap}} \\
    Berdasarkan hasil investigasi dan teori pendukung, dilakukan analisis terhadap efektivitas tata kelola AE saat ini di Paragon Corp. Tahap ini mencakup:
    \begin{enumerate}[a.]
        \item Penilaian tingkat kematangan AE menggunakan kerangka \textit{Enterprise Architecture Maturity Model (EAMM)}.
        \item Melakukan analisis \textit{gap} antara kondisi saat ini dan \textit{best practice} tata kelola AE. \\
    \end{enumerate}

    \item {Tahap perancangan model perbaikan tata kelola AE} \\
    Pada tahap ini dirancang model perbaikan tata kelola AE yang lebih terstruktur, selaras dengan \textit{maturity level} yang ditargetkan, dan sesuai konteks organisasi. 
%    Model rancangan akan mencakup komponen peran dan tanggung jawab, proses kerja, serta mekanisme evaluasi dan pembaruan artefak EA. \\

    \item {Tahap evaluasi dan validasi model} \\
    Model yang dihasilkan kemudian dievaluasi untuk menilai kelayakan dan dampaknya terhadap efektivitas pengelolaan AE. Evaluasi dilakukan melalui:
    \begin{enumerate}[a.]
        \item \textit{Expert review} bersama pemnagku kepentingan internal Paragon Corp.
        \item Analisis perbandingan antara kondisi sebelum dan sesudah penerapan model. \\
    \end{enumerate}
\end{enumerate}