\begin{longtable}{@{\extracolsep{\fill}}
    p{2.8cm}
    p{2.8cm}
    p{2.8cm}
    p{2.8cm}
    p{2.8cm}}
\caption{Identifikasi \textit{Gap} dan Faktor Penyebab pada Setiap Dimensi EA}
\label{tbl:GapEA} \\
\toprule
\textbf{Dimensi} &
\textbf{Indikator Level 3 (TOGAF)} &
\textbf{Kondisi Aktual} &
\textbf{Gap} &
\textbf{Faktor Penyebab} \\
\midrule
\endfirsthead

\caption[]{Identifikasi \textit{Gap} dan Faktor Penyebab pada Setiap Dimensi EA (lanjutan)} \\
\toprule
\textbf{Dimensi} &
\textbf{Indikator Level 3 (TOGAF)} &
\textbf{Kondisi Aktual} &
\textbf{Gap} &
\textbf{Faktor Penyebab} \\
\midrule
\endhead

\midrule
\multicolumn{5}{r}{\textit{Bersambung ke halaman berikutnya}} \\
\endfoot

\bottomrule
\endlastfoot

% =========================================================
% ROWS
% =========================================================

\textbf{\textit{Architecture Process}} &
Proses arsitektur sudah dijelaskan dengan jelas, terdokumentasi, dan disosialisasikan. Selain itu sudah ada \textit{gap analysis} dan rencana migrasi menuju kondisi ideal. &
Proses sudah jelas dan disosialisasikan, namun, beberapa tim masih menjalankan proses dengan cara yang berbeda. Analisis \textit{gap} juga sudah ada, tetapi mekanismenya dan aspeknya masih belum jelas. &
Penerapan proses belum konsisten di seluruh tim. &
Perbedaan cara kerja antar tim dan belum ada mekanisme kontrol penerapan proses. \\

\textbf{\textit{Architecture Development}} &
TRM dan \textit{standards profile} sudah lengkap. Selain itu, \textit{gap analysis} dan rencana migrasi juga sudah selesai dibuat. &
Domain bisnis dan aplikasi mulai terdokumentasi sejak awal. Sementara itu, domain data, infrastruktur, dan keamanan terdokumentasi sambil berjalannya suatu inisiatif. &
Dokumentasi belum dilakukan sejak tahap inisiasi untuk semua domain. &
Belum semua domain memiliki kebiasaan dokumentasi awal. \\

\textbf{\textit{Business Alignment}} &
AE sudah menjadi bagian dari proses perencanaan investasi dan pengendalian proyek. &
\textit{Business Architect} aktif selaraskan kebutuhan TI dan bisnis. Namun beberapa tim tidak selalu melewati proses analisis EA dan tidak memberitahukan tim EA. &
Kepatuhan belum sempurna karena beberapa investasi/proyek tetap berjalan tanpa proses analisis EA yang lengkap. & 
Pengawasan dan mekanisme penegakan masih lemah. Selain itu proses koordinasi tidak selalu diikuti oleh beberapa tim \\

\textbf{\textit{Organization}} &
Manajemen memberikan dukungan penuh dan keterlibatan lintas unit berjalan secara berkelanjutan dan kolaboratif. &
Struktur peran sudah ditetapkan dengan jelas. Namun, belum dilakukan secara proaktif tanpa dorongan konteks tertentu. &
Kolaborasi lintas unit belum sepenuhnya menjadi kebiasaan proaktif di semua konteks. &
Belum ada mekanisme atau kebiasaan kerja yang mendorong kolaborasi lintas unit secara otomatis dan koordinasi lebih sering muncul setelah ada kebutuhan yang memicu. \\

\textbf{\textit{Architecture Governance}} &
Tata kelola sudah terdokumentasi dan mencakup sebagian besar investasi TI dan adanya mekanisme pengecualian resmi (\textit{waiver}). &
Forum \textit{Architecture Review Board} (ARB) belum berjalan secara formal.
Selain itu, kepatuhan terhadap standar masih bersifat himbauan, sehingga belum ada aturan wajib yang harus dipatuhi oleh semua tim. &
Tata kelola belum menyeluruh dan mekanisme \textit{waiver} belum tersedia. &
Ketiadaan ARB aktif, tidak ada aturan wajib, dan budaya kepatuhan belum kuat. \\

\textbf{\textit{Architecture Communication}} &
Dokumentasi arsitektur diperbarui secara berkala dan dikomunikasikan secara rutin ke tim TI dan bisnis. &
SAP LeanIX sudah digunakan sebagai sumber referensi utama untuk dokumen arsitektur, dan pembaruan dokumen mulai dilakukan secara rutin.&
-- Tidak teridentifikasi gap signifikan berdasarkan data saat ini -- &
-- Tidak ada faktor penyebab yang perlu dicatat karena kondisi aktual sejauh ini konsisten dengan indikator Level 3. \\

\end{longtable}
