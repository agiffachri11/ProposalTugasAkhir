\begin{longtable}{@{\extracolsep{\fill}}
    >{\raggedright\arraybackslash}p{2.0cm}
    >{\raggedright\arraybackslash}p{5.0cm}
    >{\raggedright\arraybackslash}p{5.0cm}
}
\caption{Kebutuhan Fungsional \textit{EA Impact Analysis}}
\label{tbl:KF} \\
\toprule
\textbf{Kode} &
\textbf{Kebutuhan Fungsional} &
\textbf{Deskripsi} \\
\midrule
\endfirsthead

\caption[]{Kebutuhan Fungsional \textit{EA Impact Analysis} (lanjutan)} \\
\toprule
\textbf{Kode} &
\textbf{Kebutuhan Fungsional} &
\textbf{Deskripsi} \\
\midrule
\endhead

\midrule
\multicolumn{3}{r}{\textit{Bersambung ke halaman berikutnya}} \\
\endfoot

\bottomrule
\endlastfoot

% ====================== ROWS ======================

KF-01 &
\textit{Template EA Impact Analysis} &
Proses harus menyediakan \textit{template} standar \textit{EA Impact Analysis} yang mencakup \textit{domain Business, Application, Data, Infrastructure,} dan \textit{Security}. \\

KF-02 &
Kewajiban pada Tahap \textit{Ideation} &
\textit{EA Impact Analysis} wajib dilakukan sejak tahap \textit{ideation} sebelum URS masuk ke fase \textit{Risk and Impact}. \\

KF-03 &
Pencatatan di SAP LeanIX &
Hasil \textit{EA Impact Analysis} wajib dicatat pada SAP LeanIX sebagai artefak arsitektur yang terdokumentasi. \\

KF-04 &
Validasi Kelengkapan &
Proses harus memverifikasi kelengkapan \textit{EA Impact Analysis} sebelum URS dapat masuk ke tahap \textit{Risk and Impact}. \\

KF-05 &
Mekanisme Revisi dan \textit{Tracking} &
Proses harus menyediakan pencatatan revisi dan histori perubahan. \\

KF-06 &
Sinkronisasi Artefak &
Proses harus menyinkronkan dampak perubahan terhadap artefak \textit{current state} dan \textit{desired state} agar tetap konsisten. \\

KF-07 &
Pelaporan Dampak Arsitektural &
Proses harus menghasilkan ringkasan dampak arsitektural sebagai bagian dari input untuk \textit{Risk and Impact} hingga proses implementasi. \\

\end{longtable}
