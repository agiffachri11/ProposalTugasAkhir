% ============================================================================================
% BAB III ANALISIS MASALAH
% Pembagian subbab tidak rigid dan dapat bervariasi. Bab ini minimal berisi analisis kebutuhan
% fungsional dan nonfungsional, analisis berbagai alternatif solusi yang dapat ditawarkan, dan
% metode pemilihan solusi yang diusulkan.
% ============================================================================================
\chapter{ANALISIS MASALAH}
\label{chap:analisis-masalah}
\section{Gambaran Umum Perusahaan}
\subsection{Profil Paragon Corp}
Paragon Corp adalah perusahaan \textit{Fast Moving Consumer Goods} (FMCG) bidang kosmetik di Indonesia yang mengemban tujuan utama menciptakan kebaikan yang lebih besar bagi masyarakat melalui inovasi. 
Perusahaan ini didukung lebih dari 10.000 karyawan yang mereka sebut paragonian yang tersebar di Indonesia hingga Malaysia. 
Perusahaan ini terus berinovasi dalam produk, program, dan cara kerja untuk menyebarkan manfaat ke seluruh lapisan masyarakat \autocite{ParagonInnovation}.

Paragon Corp memiliki visi yaitu menjadi perusahaan yang berkomitmen pada tata kelola terbaik dan perbaikan berkelanjutan agar lebih baik dari kemarin melalui produk berkualitas tinggi yang memberikan manfaat bagi paragonian, mitra, masyarakat, dan lingkungan.
Misi Paragon Corp berfokus kepada enam pilar utama yaitu:
\begin{enumerate}
    \item Mengembangkan dan mendidik paragonian yang kompeten dengan keunggulan kompetitif.
    \item Mendengarkan kebutuhan konsumen dan menciptakan produk yang melampaui ekspetasi mereka.
    \item Meningkatkan kualitas produk melalui inovasi.
    \item Bekerja sama dengan mitra bisnis demi keuntungan bersama.
    \item Berusaha keras untuk menjaga bumi secara berkelanjutan.
    \item Mendukung pengembangan generasi baru melalui pendidikan dan kesehatan untuk menciptakan sumber daya manusia yang berpengetahuan dan sehat.
    \item Memperluas jangkauan produk, layanan, dan area bisnis.
  \end{enumerate}

\subsection{Penerapan Arsitektur \textit{Enterprise} di Paragon Corp}
Pertumbuhan pada Direktorat Teknologi Informasi (TI) Paragon Corp menimbulkan banyak tantangan.
Manajemen mulai kesulitan untuk menentukan arah strategis perusahaan karena bertambahnya aplikasi dan layanan.
Kondisi ini mendorong untuk membentuk Tim \textit{Enterprise Architecture}.

Pada awal pembentukannya, Paragon Corp menjalin kerja sama dengan konsultan PwC untuk melakukan penilaian terhadap kapabilitas Arsitektur \textit{Enterprise} (AE).
Hasil penilaian menunjukkan bahwa tingkat \textit{maturity} AE di Paragon Corp berada di level 1. 
Saat ini, tim masih melakukan pengumpulan data terkait aset Teknologi Informasi (TI) dari berbagai pemangku kepentingan.
Beriringan dengan tahap ini, akan segera dijalani juga implementasi proses tata kelola AE yang terstruktur untuk memastikan pengembangan TI di Paragon Corp sesuai dengan tujuan yang ditetapkan.

\section{Analisis Kondisi Tata Kelola Arsitektur \textit{Enterprise} Saat Ini}
\subsection{Struktur dan Mekanisme Tata Kelola Arsitektur \textit{Enterprise} Saat Ini}
Tata kelola Arsitektur \textit{Enterprise} (AE) di Paragon Corp melibatkan beberapa peran yang bekerja secara lintas fungsi. 
Secara garis besar, struktur ini terdiri atas \textit{Business Architect, Enterprise Architect, Cloud Infrastructure, IT Security,} dan \textit{Engineering Manager}.
\begin{enumerate}
  \item {\textit{Business Architect}} \\
  \textit{Business Architect} bertanggung jawab dalam menginisiasi perubahan atau inisiatif melalui penyusunan dokumen \textit{User Requirement Specification} (URS), melakukan \textit{EA impact analysis}, dan melakukan \textit{architecture impact scoring} terhadap arsitektur. \\

  \item {\textit{Enterprise Architect} (EA)} \\
  \textit{Enterprise Architect} bertindak sebagai konsultan arsitektur yang melakukan pengecekan kelengkapan artefak, memberikan arahan arsitektural, dan memonitor jalannya proyek di setiap fase. \\

  \item {\textit{Cloud Infrastructure} (CI) dan \textit{IT Security}} \\
  \textit{Cloud Infrastructure} dan \textit{IT Security} bertugas dalam menyusun dan memperbarui artefak terkait infrastruktur dan keamanan, memastikan perubahan sejalan dengan standar keamanan dan kebutuhan infrastruktur yang relevan. \\

  \item {\textit{Engineering Manager} (EM)} \\
  \textit{Engineering Manager} berperan dalam mereview dokumen URS, memonitor pengembangan aplikasi, serta memastikan seluruh artefak aplikasi dan data terdokumentasi dengan baik. \\
\end{enumerate}

Alur tata kelola AE terbagi menjadi lima tahap, dengan detail sebagai berikut:
\begin{enumerate}
  \item {Tahap \textit{Ideation}} \\
  Tahap \textit{ideation} dimulai ketika muncul kebutuhan perubahan atau pengembangan sistem, biasanya diinisiasi oleh tim bisnis. 
  Inisiatif ini ditindaklanjuti dengan penyusunan dokumen \textit{User Requirement Specification} (URS) oleh \textit{Business Architect}. 
  Langkah berikutnya, \textit{Business Architect} melakukan \textit{architecture impact scoring} untuk mengidentifikasi seberapa besar pengaruh perubahan tersebut terhadap arsitektur bisnis dan aplikasi. 
  Penilaian ini dikategorikan menjadi dua yaitu \textit{high impact} dan \textit{low impact}.
  \textit{High impact} dikategorikan jika perubahannya besar seperti modifikasi proses bisnis utama, pergantian sistem, atau perubahan integrasi kritikal, sedangkan \textit{low impact} dikategorikan jika perubahannya minor seperti modifikasi fitur kecil atau tampilan. \\

  \textit{Business Architect} kemudian mendokumentasikan artefak bisnis seperti \textit{business process flow} dan \textit{user flow diagram} pada \textit{platform} SAP LeanIX. 
  Informasi yang dimasukkan di URS dan SAP LeanIX berupa \textit{current state} dan \textit{desired state} saat perubahan diimplementasikan.
  Semua progres dipantau melalui excel \textit{tracker} untuk memastikan ketercapaian setiap aktivitas pada tahapan \textit{ideation}. 
  Tahap \textit{ideation} ini memastikan seluruh rencana perubahan telah tervalidasi sejak awal sebelum masuk ke tahap \textit{risk and impact}. 
  Gambar \ref{gambar:asis_ideation} menunjukkan tahapan \textit{ideation} saat ini.\\
  \begin{figure}[H] % pilihan opsi yang disarankan: t = top, b = bottom, h = here
    \centering
    \captionsetup{justification=centering}
        \includegraphics[width=0.9\textwidth]{image/ideation2.png}
    \caption{Tahapan \textit{Ideation} Saat Ini}
    \label{gambar:asis_ideation}
  \end{figure}

  \item {Tahap \textit{Risk and Impact}} \\
  Tahap \textit{risk and impact} berfokus kepada analisis dampak yang muncul dari rencana perubahan sistem. 
  Setelah dokumen URS selesai diinisiasi pada tahap \textit{ideation}, dokumen tersebut akan dikirim oleh \textit{Business Architect} kepada \textit{Engineering Manager} untuk dilakukan \textit {review}.
  \textit{Engineering Manager} diberikan waktu maksimal 7 hari untuk memeriksa URS yang telah dibagikan. 
  Hasil \textit{review} tersebut disampaikan kembali kepada \textit{Business Architect} melalui grup Microsoft Teams. \\

  Setelah proses \textit{review}, \textit{Engineering Manager} memulai pembuatan artefak arsitektur aplikasi dan data pada \textit{platform} SAP LeanIX.. 
  Kegiatan ini berjalan secara paralel dengan langkah \textit{Cloud Infrastructure} dan \textit{IT Security} dalam membuat rancangan awal (\textit{draft}) arsitektur infrastruktur dan keamanan pada \textit{platform} SAP LeanIX.. 
  Tahapan \textit{Risk and Impact} bertujuan agar semua risiko potensial serta dampak perubahan dapat teridentifikasi dan dianalisis sebelum memasuki tahap \textit{development}. 
  Gambar \ref{gambar:asis_riskImpact} menunjukkan tahapan \textit{risk and impact} saat ini.\\
  \begin{figure}[H] % pilihan opsi yang disarankan: t = top, b = bottom, h = here
    \centering
    \captionsetup{justification=centering}
        \includegraphics[width=0.7\textwidth]{image/riskImpact2.png}
    \caption{Tahapan \textit{Risk and Impact} Saat Ini}
    \label{gambar:asis_riskImpact}
  \end{figure}

  \item {Tahap \textit{Development}} \\
  Tahap \textit{development} berfokus dalam aktivitas pengembangan kode dimulai berdasarkan \textit{task} yang telah diberikan oleh \textit{Engineering Manager} kepada \textit{software engineering}. 
  Tahapan \textit{development} diatur berdasarkan hasil penilaian pada dokumen URS, khususnya keputusan terkait dampak perubahan (\textit{high impact} atau \textit{minimum impact}).
  Jika sebuah inisiatif bernilai \textit{high impact}, maka \textit{Business Architect} akan melakukan \textit{EA Project Monitoring} untuk fase \textit{development} dan melanjutkan artefak bisnis. 
  \textit{Enterprise Architect} akan menyediakan konsultasi terkait arsitektur, memastikan rancangan serta implementasi tetap sejalan dengan standar dan aturan AE perusahaan. 
  \textit{Cloud Infrastructure} dan \textit{IT Security} melanjutkan artefak infrastruktur dan keamanan. 
  Selain itu \textit{Engineering Manager}  melanjutkan artefak aplikasi dan data. 
  Pada \textit{minimum impact}, pembaruan artefak dilakukan tanpa konsultasi dengan \textit{Enterprise Architect} dan tanpa melakukan \textit{EA Project Monitoring}.\\

  Selama proses \textit{development}, pembaruan artefak dapat dilakukan secara paralel sesuai kebutuhan. 
  Sebelum transisi ke tahap \textit{UAT and Go-Live}, \textit{Enterprise Architect} akan melakukan \textit{review} kelengkapan \textit{EA Project Monitoring} serta memastikan seluruh \textit{checklist} terpenuhi. 
  Gambar \ref{gambar:asis_dev} menunjukkan tahapan \textit{development} saat ini.\\
  \begin{figure}[H] % pilihan opsi yang disarankan: t = top, b = bottom, h = here
    \centering
    \captionsetup{justification=centering}
        \includegraphics[width=0.9\textwidth]{image/dev2.png}
    \caption{Tahapan \textit{Development} Saat Ini}
    \label{gambar:asis_dev}
  \end{figure}

  \item {Tahap \textit{UAT and Go-Live}} \\
  Tahap \textit{UAT and Go-Live} merupakan fase validasi akhir dan peluncuran sistem yang telah dikembangkan. 
  Prosesnya diawali dengan identifikasi kembali status \textit{high impact} atau \textit{minimum impact} atas perubahan yang akan diterapkan. 
  Jika perubahan dikategorikan \textit{high impact}, dilakukan \textit{EA Project Monitoring} untuk fase \textit{UAT and Go-Live}. 
  \textit{Enterprise Architect} memastikan seluruh \textit{checklist} sudah lengkap sebelum dinyatakan siap diproduksi.
  Sementara itu, untuk \textit{minimum impact}, proses \textit{EA Project Monitoring} tidak dilakukan. \\

  Pada tahap ini, seluruh artefak arsitektur (bisnis, aplikasi, data, infrastruktur, keamanan) harus difinalisasi sebelum sistem masuk ke produksi. 
  Setelah artefak final, maka dilanjutkan pengecekan penyelesaian artefak saat \textit{Change Advisory Board} (CAB) jika perubahan bersifat \textit{high impact}, sedangkan artefak akan di cek saat \textit{Bi-weekly Architecture Change Review} jika perubahannya bersifat \textit{minimum impact}. 
  Setelah seluruh proses verifikasi dilalui dan artefak dinyatakan lengkap, status perubahan yang \textit{minimum impact} akan ditandai "\textit{complete}" pada \textit{tracker}, sedangkan untuk \textit{high impact} diterima dalam \textit{Request for Change} (RFC) yang kemudian dapat dilanjutkan ke produksi. 
  Gambar \ref{gambar:asis_UAT} menunjukkan tahapan \textit{UAT and Go-Live} saat ini.\\
  \begin{figure}[H] % pilihan opsi yang disarankan: t = top, b = bottom, h = here
    \centering
    \captionsetup{justification=centering}
        \includegraphics[width=0.9\textwidth]{image/uat2.png}
    \caption{Tahapan \textit{UAT and Go-Live} Saat Ini}
    \label{gambar:asis_UAT}
  \end{figure}

  \item {Tahap \textit{Hypercare}} \\
  Tahap \textit{hypercare} adalah fase akhir setelah sistem dinyatakan berhasil \textit{Go Live}. 
  Fase ini berfungsi sebagai masa pengawasan dan pendampingan operasional untuk memastikan bahwa hasil perubahan bisa berjalan stabil, tidak ada gangguan kritis, dan seluruh dokumentasi arsitektural telah lengkap. 
  Tahapan \textit{hypercare} diawali identifikasi kembali status \textit{high impact} atau \textit{minimum impact} atas perubahan yang akan diterapkan.
  Jika \textit{minimum impact}, proyek dapat langsung ditutup (\textit{closing project}) setelah konfirmasi stabilitas sistem dan kelengkapan artefak. 
  Jika \textit{high impact}, dilakukan \textit{EA Project Monitoring} untuk fase \textit{hypercare}. 
  \textit{Enterprise Architect} akan memastikan seluruh \textit{checklist} sudah lengkap sebelum \textit{closing project}. 
  Selain itu, \textit{Enterprise Architect} memberikan konsultasi arsitektural selama periode \textit{hypercare} untuk membantu verifikasi artefak. 
  Gambar \ref{gambar:asis_hypercare} menunjukkan tahapan \textit{hypercare} saat ini.\\
  \begin{figure}[H] % pilihan opsi yang disarankan: t = top, b = bottom, h = here
    \centering
    \captionsetup{justification=centering}
        \includegraphics[width=0.7\textwidth]{image/hyper2.png}
    \caption{Tahapan \textit{Hypercare} Saat Ini}
    \label{gambar:asis_hypercare}
  \end{figure}
\end{enumerate}

\subsection{Pemanfaatan SAP LeanIX}
SAP LeanIX berperan penting sebagai \textit{platform} utama dalam mendukung proses tata kelola AE di Paragon Corp. 
Semua artefak arsitektur kondisi saat ini (\textit{current state}) dan kondisi yang diharapkan (\textit{desired state}) dicatat dan dikelola secara terpusat di SAP LeanIX.
\textit{Platform} ini memungkinkan seluruh tim yang terlibat dalam proses tata kelola AE dapat mengakses dan memperbarui dokumentasi secara kolaboratif.

\subsection{Permasalahan yang Ditemukan}
Setelah melakukan wawancara dengan Tim \textit{Enterprise Architect} Paragon Corp, pelaksanaan tata kelola AE di Paragon Corp masih menghadapi berbagai tantangan. 
Permasalahan tersebut dapat dikategorikan sebagai berikut:
\begin{enumerate}
  \item Keterbatasan Sumber Daya Manusia dan Beban Kerja \\
  Keterbatasan sumber daya manusia pada tim \textit{Cloud Infrastructure} dan \textit{IT Security} menyebabkan pembaruan dan dokumentasi artefak arsitektur belum dapat dilakukan secara optimal. 
  Pada proses saat ini artefak infrastruktur dan keamanan hanya diperbarui di SAP LeanIX dan tidak tercatat secara formal dalam dokumen URS. 
  Disisi lain, \textit{Business Architect} juga memiliki beban kerja yang tinggi yang menyebabkan keterbatasan waktu dalam mengambil peran pada proses tata kelola AE. \\

  \item Rendahnya Kesadaran Pentingnya Dokumentasi \\
  Kesadaran terhadap pentingnya dokumentasi pada artefak arsitektur belum sepenuhnya tertanam di seluruh anggota tim. 
  Dokumentasi sering dianggap sekadar formalitas atau beban administratif, bukan sebagai kebutuhan strategis untuk pengelolaan dan pengendalian arsitektur perusahaan ke depan.
  Hal ini berdampak pada kurangnya kedisiplinan dalam memperbarui dan melengkapi seluruh artefak AE secara berkala.\\

  \item Ketidakjelasan Proses Konsultasi ke \textit{Enterprise Architect} \\
  Saat ini belum ada parameter atau standar yang jelas mengenai aspek apa saja yang harus dikonsultasikan ke \textit{Enterprise Architect}. 
  Proses konsultasi sering dilakukan secara informal, tanpa aturan baku yang mengikat. 
  Di samping itu, Paragon Corp juga belum membentuk forum formal seperti \textit{Architecture Review Board} (ARB), sehingga pengambilan keputusan strategis terkait arsitektur masih kurang terstruktur dan kurang terpantau oleh lintas divisi. \\

  \item \textit{EA Impact Analysis} yang Belum Sesuai dengan Standar \\
  \textit{EA impact analysis} pada domain infrastruktur dan keamanan belum dilakukan secara komprehensif sejak awal. 
  Saat ini \textit{impact analysis} untuk kedua domain ini baru diperbarui atau di-(\textit{input}) ke SAP LeanIX setelah perubahan terlaksana. 
  Selain itu, \textit{impact analysis} untuk kedua domain ini tidak tercatat di dokumen URS, melainkan di SAP LeanIX saja.
  Hal ini membuka celah risiko operasional dan dapat menurunkan kualitas tata kelola arsitektur. \\

  Tim \textit{Enterprise Architect} juga menekankan masih perlu analisis untuk lima domain AE terkait \textit{EA impact analysis} ini.
  Hal ini dikarenakan prosedur untuk melakukan \textit{impact analysis} masih belum terdefinisi dengan jelas.\\

  \item Tantangan Budaya Organisasi \\
  Budaya “tidak ada paksaan” dalam organisasi menjadi tantangan tersendiri untuk menciptakan disiplin dan konsistensi dalam dokumentasi. 
  Program-program seperti penetapan \textit{Objective Key Results} (OKR) di Direktorat Teknologi Informasi (TI) telah diimplementasikan untuk mendorong kebiasaan mendokumentasi, namun efektivitasnya sangat bergantung pada komitmen pribadi masing-masing anggota, bukan pada sistem kontrol formal dari manajemen. \\
\end{enumerate}

Permasalahan yang akan diselesaikan dalam Tugas Akhir ini yaitu pelaksanaan \textit{EA impact analysis} yang belum sesuai dengan tata kelola yang diharapkan.
Permasalahan ini dipilih karena Tim \textit{Enterprise Architect} Paragon Corp menilai bahwa prosedur \textit{EA impact analysis} yang ada saat ini masih belum jelas.
Selain itu, permasalahan ini perlu diselesaikan karena proses ini merupakan proses yang mengendalikan setiap perubahan arsitektur agar tetap selaras dengan perusahaan.
Ketidakjelasan proses ini akan menyebabkan inkonsistensi analisis, munculnya perubahan yang tidak terkendali, dan potensi gangguan integrasi.
Oleh karena itu, pada Bab IV akan dibahas rancangan solusi untuk memperbaiki proses \textit{EA impact analysis} dengan standar yang diharapkan organisasi.

\section{Analisis \textit{Gap} terhadap \textit{EA Maturity} Level 3 \textit{Defined}}
Setelah wawancara dengan Tim \textit{Enterprise Architect} Paragon Corp, penilaian terakhir terkait \textit{EA maturity} Paragon Corp berada di level 2.
Evaluasi tingkat \textit{EA maturity} dilakukan dengan membandingkan kondisi AE saat ini di Paragon Corp dengan standar EAMM yang diadopsi dari TOGAF (mengacu ke Tabel \ref{tbl:EAMM}). 
Penilaian dimulai dengan menguraikan karakteristik tiap level pada masing-masing dimensi, khususnya pada target level 3 (\textit{Defined}), lalu mengidentifikasi \textit{gap} yang didapatkan.
Pada Tabel \ref{tbl:GapEA} menunjukkan identifikasi awal dari hasil analisis \textit{gap}.
Sementara itu untuk hasil analisis \textit{gap} secara detail akan dijelaskan di pelaksanaan Tugas Akhir 2.

\begin{landscape}
\begin{longtable}{@{\extracolsep{\fill}}
    p{2.8cm}
    p{2.8cm}
    p{2.8cm}
    p{2.8cm}
    p{2.8cm}}
\caption{Identifikasi \textit{Gap} dan Faktor Penyebab pada Setiap Dimensi EA}
\label{tbl:GapEA} \\
\toprule
\textbf{Dimensi} &
\textbf{Indikator Level 3 (TOGAF)} &
\textbf{Kondisi Aktual} &
\textbf{Gap} &
\textbf{Faktor Penyebab} \\
\midrule
\endfirsthead

\caption[]{Identifikasi \textit{Gap} dan Faktor Penyebab pada Setiap Dimensi EA (lanjutan)} \\
\toprule
\textbf{Dimensi} &
\textbf{Indikator Level 3 (TOGAF)} &
\textbf{Kondisi Aktual} &
\textbf{Gap} &
\textbf{Faktor Penyebab} \\
\midrule
\endhead

\midrule
\multicolumn{5}{r}{\textit{Bersambung ke halaman berikutnya}} \\
\endfoot

\bottomrule
\endlastfoot

% =========================================================
% ROWS
% =========================================================

\textbf{\textit{Architecture Process}} &
Proses arsitektur sudah dijelaskan dengan jelas, terdokumentasi, dan disosialisasikan. Selain itu sudah ada \textit{gap analysis} dan rencana migrasi menuju kondisi ideal. &
Proses sudah jelas dan disosialisasikan, namun, beberapa tim masih menjalankan proses dengan cara yang berbeda. Analisis \textit{gap} juga sudah ada, tetapi mekanismenya dan aspeknya masih belum jelas. &
Penerapan proses belum konsisten di seluruh tim. &
Perbedaan cara kerja antar tim dan belum ada mekanisme kontrol penerapan proses. \\

\textbf{\textit{Architecture Development}} &
TRM dan \textit{standards profile} sudah lengkap. Selain itu, \textit{gap analysis} dan rencana migrasi juga sudah selesai dibuat. &
Domain bisnis dan aplikasi mulai terdokumentasi sejak awal. Sementara itu, domain data, infrastruktur, dan keamanan terdokumentasi sambil berjalannya suatu inisiatif. &
Dokumentasi belum dilakukan sejak tahap inisiasi untuk semua domain. &
Belum semua domain memiliki kebiasaan dokumentasi awal. \\

\textbf{\textit{Business Alignment}} &
AE sudah menjadi bagian dari proses perencanaan investasi dan pengendalian proyek. &
\textit{Business Architect} aktif selaraskan kebutuhan TI dan bisnis. Namun beberapa tim tidak selalu melewati proses analisis EA dan tidak memberitahukan tim EA. &
Kepatuhan belum sempurna karena beberapa investasi/proyek tetap berjalan tanpa proses analisis EA yang lengkap. & 
Pengawasan dan mekanisme penegakan masih lemah. Selain itu proses koordinasi tidak selalu diikuti oleh beberapa tim \\

\textbf{\textit{Organization}} &
Manajemen memberikan dukungan penuh dan keterlibatan lintas unit berjalan secara berkelanjutan dan kolaboratif. &
Struktur peran sudah ditetapkan dengan jelas. Namun, belum dilakukan secara proaktif tanpa dorongan konteks tertentu. &
Kolaborasi lintas unit belum sepenuhnya menjadi kebiasaan proaktif di semua konteks. &
Belum ada mekanisme atau kebiasaan kerja yang mendorong kolaborasi lintas unit secara otomatis dan koordinasi lebih sering muncul setelah ada kebutuhan yang memicu. \\

\textbf{\textit{Architecture Governance}} &
Tata kelola sudah terdokumentasi dan mencakup sebagian besar investasi TI dan adanya mekanisme pengecualian resmi (\textit{waiver}). &
Forum \textit{Architecture Review Board} (ARB) belum berjalan secara formal.
Selain itu, kepatuhan terhadap standar masih bersifat himbauan, sehingga belum ada aturan wajib yang harus dipatuhi oleh semua tim. &
Tata kelola belum menyeluruh dan mekanisme \textit{waiver} belum tersedia. &
Ketiadaan ARB aktif, tidak ada aturan wajib, dan budaya kepatuhan belum kuat. \\

\textbf{\textit{Architecture Communication}} &
Dokumentasi arsitektur diperbarui secara berkala dan dikomunikasikan secara rutin ke tim TI dan bisnis. &
SAP LeanIX sudah digunakan sebagai sumber referensi utama untuk dokumen arsitektur, dan pembaruan dokumen mulai dilakukan secara rutin.&
-- Tidak teridentifikasi gap signifikan berdasarkan data saat ini -- &
-- Tidak ada faktor penyebab yang perlu dicatat karena kondisi aktual sejauh ini konsisten dengan indikator Level 3. \\

\end{longtable}

\end{landscape}

\section{Analisis Kebutuhan Perbaikan Tata Kelola Arsitektur \textit{Enterprise}}
Pada bagian ini, analisis kebutuhan perbaikan tata kelola Arsitektur \textit{Enterprise} (AE) difokuskan secara spesifik pada masalah utama yang dipilih untuk diselesaikan, yaitu proses \textit{EA Impact Analysis} yang belum menyeluruh dan belum mengikuti standar tata kelola yang diharapkan.
Pemilihan fokus ini didasarkan pada hasil analisis \textit{gap} pada dimensi \textit{Architecture Process}.
Pada dimensi tersebut, ditemukan penerapan proses EA tidak konsisten di seluruh tim.

Salah satu penyebab utama inkonsistensi tersebut adalah ketiadaan mekanisme yang dapat memastikan proses \textit{EA Impact Analysis} dilakukan secara terstruktur.
Oleh karena itu, identifikasi kebutuhan diarahkan untuk menyelesaikan proses \textit{EA Impact Analysis} yang dapat diterapkan secara konsisten pada seluruh unit terkait.
Kebutuhan dibedakan menjadi kebutuhan fungsional dan kebutuhan non-fungsional yang diperlukan untuk mendukung implementasi \textit{EA Impact Analysis} yang sesuai standar.

\subsection{Kebutuhan Fungsional}
Kebutuhan fungsional diperlukan untuk memastikan proses \textit{EA Impact Analysis} dapat berjalan sesuai standar tata kelola AE yang ditetapkan.
Tabel \ref{tbl:KF} menunjukkan kebutuhan fungsional yang menjawab masalah \textit{EA Impact Analysis}. \\
\begin{longtable}{@{\extracolsep{\fill}}
    >{\raggedright\arraybackslash}p{2.0cm}
    >{\raggedright\arraybackslash}p{5.0cm}
    >{\raggedright\arraybackslash}p{5.0cm}
}
\caption{Kebutuhan Fungsional \textit{EA Impact Analysis}}
\label{tbl:KF} \\
\toprule
\textbf{Kode} &
\textbf{Kebutuhan Fungsional} &
\textbf{Deskripsi} \\
\midrule
\endfirsthead

\caption[]{Kebutuhan Fungsional \textit{EA Impact Analysis} (lanjutan)} \\
\toprule
\textbf{Kode} &
\textbf{Kebutuhan Fungsional} &
\textbf{Deskripsi} \\
\midrule
\endhead

\midrule
\multicolumn{3}{r}{\textit{Bersambung ke halaman berikutnya}} \\
\endfoot

\bottomrule
\endlastfoot

% ====================== ROWS ======================

KF-01 &
\textit{Template EA Impact Analysis} &
Proses harus menyediakan \textit{template} standar \textit{EA Impact Analysis} yang mencakup \textit{domain Business, Application, Data, Infrastructure,} dan \textit{Security}. \\

KF-02 &
Kewajiban pada Tahap \textit{Ideation} &
\textit{EA Impact Analysis} wajib dilakukan sejak tahap \textit{ideation} sebelum URS masuk ke fase \textit{Risk and Impact}. \\

KF-03 &
Pencatatan di SAP LeanIX &
Hasil \textit{EA Impact Analysis} wajib dicatat pada SAP LeanIX sebagai artefak arsitektur yang terdokumentasi. \\

KF-04 &
Validasi Kelengkapan &
Proses harus memverifikasi kelengkapan \textit{EA Impact Analysis} sebelum URS dapat masuk ke tahap \textit{Risk and Impact}. \\

KF-05 &
Mekanisme Revisi dan \textit{Tracking} &
Proses harus menyediakan pencatatan revisi dan histori perubahan. \\

KF-06 &
Sinkronisasi Artefak &
Proses harus menyinkronkan dampak perubahan terhadap artefak \textit{current state} dan \textit{desired state} agar tetap konsisten. \\

KF-07 &
Pelaporan Dampak Arsitektural &
Proses harus menghasilkan ringkasan dampak arsitektural sebagai bagian dari input untuk \textit{Risk and Impact} hingga proses implementasi. \\

\end{longtable}


\subsection{Kebutuhan Non-Fungsional}
Kebutuhan non fungsional diperlukan untuk mendukung keberhasilan implementasi proses \textit{EA Impact Analysis}.
Tabel \ref{tbl:KNF} menunjukkan kebutuhan non-fungsional yang menjawab masalah \textit{EA Impact Analysis}. \\
\begin{longtable}{@{\extracolsep{\fill}}
    >{\raggedright\arraybackslash}p{2.0cm}
    >{\raggedright\arraybackslash}p{5.0cm}
    >{\raggedright\arraybackslash}p{5.0cm}
}
\caption{Kebutuhan Non-Fungsional \textit{EA Impact Analysis}}
\label{tbl:KNF} \\
\toprule
\textbf{Kode} &
\textbf{Kebutuhan Non-Fungsional} &
\textbf{Deskripsi} \\
\midrule
\endfirsthead

\caption[]{Kebutuhan Non-Fungsional \textit{EA Impact Analysis} (lanjutan)} \\
\toprule
\textbf{Kode} &
\textbf{Kebutuhan Non-Fungsional} &
\textbf{Deskripsi} \\
\midrule
\endhead

\midrule
\multicolumn{3}{r}{\textit{Bersambung ke halaman berikutnya}} \\
\endfoot

\bottomrule
\endlastfoot

% ====================== ROWS ======================

KNF-01 &
Konsistensi Proses &
Proses \textit{EA Impact Analysis} harus dijalankan secara konsisten menggunakan standar yang sama di seluruh tim. \\

KNF-02 &
Kemudahan Pemahaman &
\textit{Template} dan panduan harus mudah dipahami oleh seluruh peran terkait. \\

KNF-03 &
Kepatuhan ke Standar Arsitektur \textit{Enterprise} &
Proses mengikuti standar arsitektur seperti TOGAF dan standar internal Paragon Corp. \\

KNF-04 &
\textit{Auditability} &
Setiap masukan, revisi, dan keputusan dalam \textit{EA Impact Analysis} harus tercatat sehingga dapat diaudit. \\

KNF-05 &
Integrasi LeanIX &
Proses harus terintegrasi dengan SAP LeanIX sebagai repositori utama arsitektur. \\

KNF-06 &
Reliabilitas Dokumentasi &
Dokumen \textit{EA Impact Analysis} harus tersimpan dengan aman dan tidak mudah hilang atau rusak. \\

KNF-07 &
Keamanan Informasi &
Data terkait perubahan arsitektur harus terlindungi dari akses tidak berwenang. \\

KNF-08 &
Standarisasi Terminologi &
Istilah arsitektural yang digunakan dalam seluruh domain harus seragam dan terdokumentasi. \\

KNF-09 &
Akurasi Informasi &
Setiap informasi yang dituangkan dalam \textit{EA Impact Analysis} harus akurat, mutakhir, dan mencerminkan kondisi arsitektur yang sebenarnya. \\

\end{longtable}


\section{Analisis Pemilihan Solusi}
\subsection{Alternatif Solusi}
Berdasarkan permasalahan yang akan diselesaikan, yaitu pelaksanaan \textit{EA Impact Analysis} yang belum memiliki tata kelola yang jelas.
Bagian ini merumuskan tiga alternatif solusi yang dapat diterapkan oleh Paragon Corp. 
Ketiga alternatif ini dirancang setara sehingga dapat dibandingkan melalui analisis keputusan pada bagian berikutnya.
\begin{enumerate}
  \item Penyusunan Mekanisme \textit{EA Impact Analysis} yang Didukung SAP LeanIX \\
  Penyusunan mekanisme \textit{EA Impact Analysis} mengacu pada kerangka kerja TOGAF sebagai referensi.
  TOGAF dipilih karena menyediakan pedoman mengenai elemen-elemen analisis arsitektur, termasuk bagaimana meilai dampak terhadap domain bisnis, data, aplikasi, infrastruktur, dan keamanan.
  Dengan mekanisme ini, perusahaan memiliki kerangka yang jelas dan tidak lagi bergantung kepada interpretasi masing-masing tim.\\
 
  Dalam penerapan solusi ini, proses mekanisme akan didukung oleh \textit{platform} SAP LeanIX untuk mengisi dan menyimpan hasil analisis.
  SAP LeanIX sudah digunakan oleh Paragon Corp sebagai repositori AE sehingga semua analisis terdokumentasi di satu tempat yang dapat diakses semua tim terkait.
  Dokumentasi hasil analisis tidak lagi tersebar atau berbeda-beda formatnya karena seluruh tim mengisi pada \textit{platform} yang sama.
  Selain itu, SAP LeanIX juga menyediakan katalog arsitektur dan relasi antar domain sehingga proses menjadi lebih efisien.\\
  
  Solusi ini menekankan konsistensi, kejelasan, dan keseragaman dalam pelaksanaan \textit{EA impact analysis}. 
  Solusi ini juga mengatasi masalah utama yaitu belum adanya kerangka formal yang diikuti dan adanya perbedaan cara kerja antar tim.
  Oleh karena itu, mekanisme \textit{EA Impact Analysis} ini akan berperan menjadi pedoman dan SAP LeanIX akan berperan dalam menjaga dokumentasi.
  Tabel \ref{tbl:alt1} menunjukkan kelebihan dan kekurangan solusi ini. \\
  \begin{table}[h]
    \centering
    \begin{tabular}{|p{3cm}|p{4cm}|p{4cm}|}
    \hline
    \textbf{Aspek} & \textbf{Kelebihan} & \textbf{Kekurangan} \\
    \hline
    Kejelasan Proses &
    Memberikan alur \textit{EA impact analysis} yang jelas dan terstruktur karena mengikuti praktik TOGAF. &
    Penyusunan SOP membutuhkan waktu, diskusi lintas unit, serta validasi berulang agar disepakati semua pihak. \\
    \hline
    Standarisasi &
    Menghasilkan keseragaman proses sehingga seluruh tim bekerja dengan pedoman yang sama. &
    Standarisasi dapat dirasakan terlalu formal oleh beberapa tim yang belum terbiasa bekerja dengan prosedur terstruktur. \\
    \hline
    Integrasi LeanIX &
    LeanIX mendukung dokumentasi sehingga mempermudah pencatatan dan pelacakan hasil analisis. &
    Perlu penyesuaian konfigurasi LeanIX agar sepenuhnya selaras dengan SOP baru yang disusun. \\
    \hline
    \end{tabular}
    \caption{Kelebihan dan Kekurangan Alternatif 1: SOP \textit{EA Impact Analysis} berbasis TOGAF dan LeanIX}
    \label{tbl:alt1}
    \end{table}
    

  \item Penerapan \textit{Checklist} dan \textit{Template EA Impact Analysis} \\
  Penyusunan \textit{checklist} digunakan setiap ada inisiatif atau perubahan yang dapat mempengaruhi arsitektur.
  \textit{Checklist} ini disusun untuk memastikan bahwa setiap aspek penting dalam arsitektur diperiksa.
  Aspek yang diperiksa seperti dampak terhadap proses bisnis, aplikasi terkait, data yang digunakan, potensi risiko keamanan, kesiapan infrastruktur, dan dependensi yang perlu diperhatikan.
  Dengan solusi ini, tim tidak perlu menafsirkan sendiri apa saja yang harus dianalisis sehingga kualitas analisis dapat lebih konsisten.\\

  Selain \textit{checklist}, \textit{template EA Impact Analysis} disediakan untuk mendokumentasikan hasil analisis.
  \textit{Template} ini memastikan laporan mudah dibaca dan dapat ditinjau oleh Tim \textit{Enterprise Architect}.
  \textit{Template} dapat memuat bagian-bagian seperti ringkasan inisiatif, domain terdampak dan risiko yang ditemukan.\\

  Pendekatan solusi ini dapat diterapkan dengan cepat tanpa memerlukan perubahan besar pada struktur organisasi atau alur proses.
  Solusi ini cocok diterapkan ketika organisasi membutuhkan perbaikan yang cepat, mudah diadopsi, dan tidak membebani tim.
  Selain itu, solusi ini mendukung tim yang mungkin belum terbiasa melakukan analisis arsitektur secara menyeluruh, sehingga mereka tetap memiliki panduan jelas dalam melakukan pekerjaan.
  Tabel \ref{tbl:alt2} menunjukkan kelebihan dan kekurangan solusi ini. \\
  \begin{table}[h]
    \centering
    \begin{tabular}{|p{3cm}|p{4cm}|p{4cm}|}
    \hline
    \textbf{Aspek} & \textbf{Kelebihan} & \textbf{Kekurangan} \\
    \hline
    Kemudahan Adopsi &
    \textit{Checklist} dan \textit{template} mudah dipahami serta cepat diterapkan oleh seluruh tim. &
    Tidak memberikan panduan proses yang selengkap SOP sehingga beberapa interpretasi masih dapat berbeda antar unit. \\
    \hline
    Konsistensi \textit{Output} &
    Membantu memastikan hasil analisis lebih seragam dan lengkap. &
    Hasil analisis sangat bergantung pada kedisiplinan pengguna dalam mengisi \textit{checklist} secara benar. \\
    \hline
    Implementasi Cepat &
    Tidak membutuhkan perubahan besar pada tata kelola atau struktur organisasi. &
    Dapat kurang efektif untuk kasus yang kompleks karena \textit{checklist} bersifat ringkas dan tidak mendalami proses. \\
    \hline
    \end{tabular}
    \caption{Kelebihan dan Kekurangan Alternatif 2: \textit{Checklist} dan \textit{Template EA Impact Analysis}}
    \label{tbl:alt2}
    \end{table}
    

  \item Pembentukan Forum Untuk Peninjauan \textit{EA Impact Analysis} \\
  Pembentukan forum berfokus kepada peninjauan hasil \textit{EA Impact Analysis}.
  Forum berfungsi sebagai mekanisme yang lebih ringkas dan cepat untuk memverifikasi bahwa analisis yang dilakukan sudah benar dan tidak ada aspek penting yang terlewat.
  Forum ini menjadi tempat untuk mengklarifikasi temuan, menilai risiko, dan memastikan bahwa keputusan diambil berdasarkan pemahaman yang lengkap.
  Forum juga memastikan bahwa seluruh domain terdampak sudah ditinjau sesuai kebutuhan.
  Dengan adanya forum ini, hasil analisis dapat diperiksa secara lintas peran sebelum sebuah keputusan dibuat.\\
  
  Solusi ini memperkuat tata kelola AE tanpa membutuhkan perubahan besar pada alur proses.
  Forum memberikan mekanisme validasi formal namun tetap praktis, dan dapat diterapkan sesuai kebutuhan.
  Selain itu, forum ini membantu memastikan kualitas hasil analisis sekaligus meningkatkan koordinasi lintas unit.
  Tabel \ref{tbl:alt3} menunjukkan kelebihan dan kekurangan solusi ini. \\
  \begin{table}[h]
    \centering
    \begin{tabular}{|p{3cm}|p{4cm}|p{4cm}|}
    \hline
    \textbf{Aspek} & \textbf{Kelebihan} & \textbf{Kekurangan} \\
    \hline
    Validasi Kualitas &
    Memberikan mekanisme \textit{review} lintas tim sehingga hasil \textit{EA Impact Analysis} lebih akurat dan tidak ada aspek yang terlewat. &
    Membutuhkan komitmen waktu dari perwakilan unit yang terlibat sehingga dapat menambah beban koordinasi. \\
    \hline
    Tata Kelola &
    Menghadirkan proses pengawasan yang lebih formal dibandingkan hanya \textit{checklist}. &
    Jika tidak dijalankan konsisten, forum berpotensi menjadi formalitas tanpa peningkatan kualitas yang nyata. \\
    \hline
    Kolaborasi Lintas Unit &
    Memperkuat kolaborasi sehingga persepsi dampak lebih menyeluruh. &
    Tanpa dokumentasi yang kuat, keputusan forum dapat sulit ditelusuri atau direferensikan kembali. \\
    \hline
    \end{tabular}
    \caption{Kelebihan dan Kekurangan Alternatif 3: Forum Peninjauan \textit{EA Impact Analysis}}
    \label{tbl:alt3}
    \end{table}
    
\end{enumerate}

Tiga alternatif solusi yang telah dijelaskan sebelumnya merupakan pilihan dalam perbaikan tata kelola AE Paragon Corp terkhusus pada proses \textit{EA Impact Analysis}. 
Setiap solusi memiliki karakteristik, keunggulan, serta potensi tantangannya sendiri. 
Pada bagian berikutnya, akan dilakukan analisis penentuan solusi untuk memilih alternatif yang paling sesuai dengan kebutuhan dan tujuan organisasi.

\subsection{Analisis Penentuan Solusi}
Untuk memilih solusi terbaik dari ketiga alternatif solusi perbaikan tata kelola EA pada proses \textit{EA Impact Analysis} dipilih metode \textit{decision matrix}. 
Metode ini digunakan karena pemilihan solusi berdasarkan pendekatan yang objektif dan terukur.
\textit{Decision matrix} memberikan keputusan melalui bobot dan skor pada setiap kriteria, sehingga keputusan akan mempertimbangkan faktor yang relevan dengan kebutuhan Paragon Corp.

Berikut adalah kriteria dan bobot yang dijadikan sebagai faktor dalam memutuskan solus terbaik.
\begin{enumerate}
  \item Kesesuaian dengan Tata Kelola AE (Bobot 30\%)\\
  Kriteria ini diberi bobot tertinggi karena inti masalah berupa ketidaksesuaian proses \textit{EA Impact Analysis} dengan tata kelola yang diharapkan.
  Solusi yang dipilih harus dapat mendukung perbaikan tata kelola secara langsung.
  Oleh karena itu, kesesuaian dengan prinsip dan standar AE menjadi prioritas utama. \\

  \item Dampak terhadap Konsistensi Proses (Bobot 25\%) \\
  Masalah yang diangkat yaitu ketidakkonsistenan pelaksanaan \textit{EA Impact Analysis}.
  Solusi harus memperkuat keseragaman proses di seluruh unit sehingga konsistensi menjadi prioritas kedua.
  Bobot diberikan 25\% karena kriteria ini menentukan apakah solusi akan benar-benar menyelesaikan akar permasalahan.\\

  \item Kemudahan Implementasi (Bobot 20\%) \\
  Paragon Corp membutuhkan solusi yang dapat diterapkan tanpa menghambat proses berjalan, sehingga kemudahan implementasi menjadi penting.
  Bobot tidak setinggi kriteria kesesuaian tata kelola karena solusi tetap harus mencapai tujuan perbaikan tata kelola.  \\

  \item Kebutuhan Koordinasi Antar Unit (Bobot 15\%) \\
  Koordinasi lintas unit merupakan bagian penting dari \textit{EA Impact Analysis}.
  Namun, bobotnya tidak terlalu tinggi karena koordinasi adalah konsekuensi pelaksanaan, bukan tujuan utama solusi.
  Meskipun lebih banyak koordinasi bisa meningkatkan kualitas, hal tersebut juga menambah beban proses.\\

  \item Kesesuaian dengan SAP LeanIX (Bobot 10\%) \\
  SAP LeanIX adalah \textit{tool} utama dokumentasi EA di Paragon Corp.
  Kesesuaian dengan LeanIX penting untuk keberlanjutan dokumentasi, tetapi bukan kriteria utama untuk menentukan efektivitas tata kelola. \\
\end{enumerate}

Tabel \ref{tbl:decision_matrix_EA} menunjukkan hasil dari \textit{decision matrix} dalam memilih solusi terbaik.
Dari hasil penilaian menggunakan \textit{decision matrix}, solusi dengan skor total tertinggi yaitu Penyusunan Mekanisme \textit{EA Impact Analysis} yang Didukung SAP LeanIX.
Alternatif ini memperoleh skor total tertinggi sebesar 4.45, mengungguli alternatif lainnya yaitu alternatif 2 (3.75) dan alternatif 3 (3.55).
\begin{table}[h]
    \centering
    \begin{tabular}{|p{4cm}|p{2cm}|p{2cm}|p{2cm}|p{2cm}|}
    \hline
    \textbf{Kriteria} & \textbf{Bobot (\%)} & \textbf{Alternatif 1} & \textbf{Alternatif 2} & \textbf{Alternatif 3} \\
    \hline
    Kesesuaian dengan Tata Kelola EA & 30 & 5 & 3 & 4 \\
    \hline
    Kemudahan Implementasi & 20 & 3 & 5 & 3 \\
    \hline
    Dampak terhadap Konsistensi Proses & 25 & 5 & 3 & 4 \\
    \hline
    Kebutuhan Koordinasi Antar Unit & 15 & 4 & 5 & 3 \\
    \hline
    Kesesuaian dengan SAP LeanIX & 10 & 5 & 3 & 3 \\
    \hline
    \textbf{Total nilai}       & \textbf{100} & \textbf{4.45} & \textbf{3.75} & \textbf{3.55} \\
    \hline
    \end{tabular}
    \caption{\textit{Decision Matrix} Penentuan Alternatif Solusi Perbaikan Proses \textit{EA Impact Analysis}}
    \label{tbl:decision_matrix_EA}
    \end{table}
    

Dalam aspek tata kelola, alternatif 1 memperoleh skor tertinggi karena langsung berbasis TOGAF.
Dengan mekanisme tersebut, perusahaan mendapatkan pedoman tunggal yang akan mengurangi variasi kerja dalam pelaksanaan \textit{EA Impact Analysis}.
Dari aspek konsistensi, alternatif 1 juga memperoleh keunggulan karena memasukkan LeanIX sebagai pendukung implementasi.
LeanIX merupakan \textit{platform} yang sudah digunakan Paragon Corp dalam pengelolaan artefak AE.
Ketika mekanisme dirancang untuk berjalan beriringan dengan LeanIX, maka proses yang sebelumnya tersebar dan tidak terdokumentasi dengan baik dapat lebih terpusat.

Jika dibandingkan dengan alternatif 2, solusi tersebut lebih cocok untuk meningkatkan efisiensi aktivitas dokumentasi, namun kurang kuat dalam aspek tata kelola dan konsistensi, sehingga tidak dapat sepenuhnya menjawab akar permasalahan.
Sementara itu, untuk alternatif 3 memang meningkatkan kualitas evaluasi lintas unit, tetapi memerlukan koordinasi yang lebih intensif dan berpotensi meningkatkan kompleksitas proses, sehingga kurang ideal sebagai solusi utama.

Dengan hasil analisis \textit{decision matrix} dan kebutuhan Paragon Corp, alternatif 1 menjadi solusi yang sesuai.
Solusi ini tidak hanya memberikan struktur tata kelola yang lebih tegas, tetapi juga realistis untuk diimplementasikan dalam lingkungan organisasi yang sudah menggunakan LeanIX sebagai \textit{platform} utama AE.

