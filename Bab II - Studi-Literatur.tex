% ==========================================
% BAB II STUDI LITERATUR
% ==========================================
\chapter{STUDI LITERATUR}
\label{chap:studi-literatur}
\section{Konsep Arsitektur \textit{Enterprise}}

\subsection{Pengertian dan Tujuan \textit{Enterprise Architecture}}
Arsitektur \textit{Enterprise} (AE) adalah metode yang membantu organisasi memetakan strategi bisnis, proses, informasi, dan teknologi yang mendukungnya.
AE berfungsi sebagai kerangka yang memastikan proses bisnis dan Teknologi Informasi (TI) berjalan sesuai standar untuk mencapai tujuan operasional perusahaan \autocite{ross2006enterprise}.

\textcite{ahlemann2012strategic} menjelaskan AE merupakan suatu prinsip, aturan, dan tata kelola yang digunakan oleh organisasi untuk memastikan konsistensi arsitektur dalam mendukung tujuan perusahaan.
Definisi ini menegaskan bahwa AE bukan hanya sekedar dokumentasi, tetapi juga pendekatan manajemen agar tercapainya tujuan perusahaan.

Sementara itu, TOGAF menjelaskan AE sebagai metode dalam merancang, mengembangkan, dan mengelola arsitektur bisnis, aplikasi, data, dan teknologi.
AE berfungsi sebagai peta yang menggambarkan bagaimana organisasi terhubung dan berkembang untuk mendukung strategi perusahaan \autocite{TOGAF_Standard_10th}.

Dari berbagai definisi tersebut, disimpulkan bahwa AE berperan sebagai landasan arsitektur yang menghubungkan strategi perusahaan dan operasional secara terstruktur.
Dengan demikian, AE dapat meningkatkan efisiensi operasional, mengurangi redundansi aplikasi, dan memastikan Teknologi Informasi (TI) tetapi selaras dengan kebutuhan bisnis.

\subsection{Domain Utama Arsitektur \textit{Enterprise}}
Arsitektur \textit{Enterprise} (AE) dibentuk oleh beberapa komponen yang disebut domain arsitektur.
domain ini mengategorikan aspek yang harus diperhatikan oleh arsitek untuk merancang sistem yang kompleks \autocite{jager2023getting}.

Dalam TOGAF, AE terbagi menjadi empat domain arsitektur yang berfungsi sebagai subsistem dari keseluruhan arsitektur perusahaan.
Pembagian domain ini menjadi struktur utama dalam \textit{Architecture Development Method} (ADM) \autocite{TOGAF_Standard_10th}.
Gambar \ref{gambar:adm} menunjukkan siklus \textit{Architecture Development Method} (ADM) yang digunakan untuk mengembangkan, mengelola, dan memelihara siklus AE pada organisasi.
Sementara itu, Gambar \ref{gambar:domain} menunjukkan struktur domain arsitektur AE.

\begin{figure}[h] % pilihan opsi yang disarankan: t = top, b = bottom, h = here
	\centering
  \captionsetup{justification=centering}
    	\includegraphics[width=0.7\textwidth]{image/adm.png}
	\caption{Siklus \textit{Architecture Development Method} (ADM)}
	\label{gambar:adm}
\end{figure}

\begin{figure}[t] % pilihan opsi yang disarankan: t = top, b = bottom, h = here
	\centering
  \captionsetup{justification=centering}
    	\includegraphics[width=0.7\textwidth]{image/domainlayer.png}
	\caption{Domain Arsitektur \textit{Layer} AE}
	\label{gambar:domain}
\end{figure}

\begin{enumerate}
    \item Arsitektur Bisnis \\
    Domain ini terletak pada fase B dari \textit{Architecture Development Method} (ADM) yang menjelaskan tentang strategi, tujuan, kapabilitas, dan proses bisnis organisasi. 
    Domain ini menggambarkan cara organisasi untuk mencapai nilai yang diberikan kepada pelanggan.\\

    \item Arsitektur Teknologi Informasi \\
    Domain ini berperan dalam menjembatani kebutuhan bisnis dengan Teknologi Informasi (TI) yang mendukungnya, dan dipecah menjadi dua sub-komponen utama.
    \begin{enumerate}[a.]
        \item Arsitektur Data \\
        Domain ini terletak pada fase C dari \textit{Architecture Development Method} (ADM) yang menjelaskan struktur dan jenis data yang dapat mendukung proses bisnis secara efektif. 

        \item Arsitektur Aplikasi \\
        Domain ini terletak pada fase C dari \textit{Architecture Development Method} (ADM) yang menjelaskan jenis aplikasi aplikasi yang digunakan untuk mendukung proses bisnis. \\
    \end{enumerate}

    \item Arsitektur Teknologi \\
    Domain ini terletak pada fase D dari \textit{Architecture Development Method} (ADM) yang menjelaskan infrastruktur Teknologi Informasi (TI) yang diperlukan untuk menjalankan aplikasi dan mengelola data perusahaan. 
\end{enumerate}

\subsection{Peran Arsitektur \textit{Enterprise} dalam Transformasi Digital}
Menurut \textcite{josey2017togaf}, Arsitektur \textit{Enterprise} (AE) berperan dalam melakukan transformasi digital melalui metode \textit{Architecture Development Method} (ADM).
Selain itu AE membantu perusahaan dalam menata ulang proses bisnis dan Teknologi Informasi (TI) melalui proses tata kelola yang disediakan TOGAF.

\textcite{vandewetering2021dynamicenterprisearchitecturecapabilities} menjelaskan bahwa penerapan AE akan membantu perusahaan dalam meningkatkan kegiatan operasional dan mendukung keberhasilan transformasi digital.
Kapabilitas AE seperti kemampuan adaptif dan integrasi antar domain akan membantu perusahaan terhadap inovasi proses dan keselarasan bisnis dan Teknologi Informasi (TI).

AE tidak akan efektif tanpa mekanisme tata kelola arsitektur, meskipun memiliki kerangka arsitektur perusahaan.
TOGAF menjelaskan bahwa tata kelola arsitektur diperlukan supaya pengembangan dan implementasi arsitektur berjalan secara konsisten dan terkontrol \autocite{TOGAF_Standard_10th}.
Oleh karena itu, perusahaan harus memiliki tata kelola arsitektur agar AE dapat tetap berjalan dan mendukung tujuan strategis perusahaan.

\section{Tata Kelola Arsitektur \textit{Enterprise}}
\subsection{Definisi dan Tujuan Tata Kelola Arsitektur \textit{Enterprise}}
Menurut \textcite{korhonen2009ea}, tata kelola Arsitektur \textit{Enterprise} (AE) adalah mekanisme kontrol yang berorientasi ke masa depan yang bertujuan untuk merancang kondisi arsitektur perusahaan di masa depan demi mendukung strategi bisnis. 
Tata kelola AE menekankan pada perencanaan dan efektivitas eksternal organisasi secara keseluruhan.
Tujuan tata kelola AE adalah memastikan pengembangan arsitektur bersifat sistemik dan mengarah pada tujuan strategis jangka panjang.

TOGAF menjelaskan bahwa tata kelola AE merupakan proses yang digunakan untuk mengelola dan mengawasi pengembangan implementasi arsitektur perusahaan supaya sesuai dengan tujuan perusahaan \autocite{TOGAF_Standard_10th}.
\textcite{foorthuis2016theory} menambahkan bahwa tata kelola AE diperlukan supaya arsitektur dijalankan secara konsisten dalam setiap proyek dan keputusan arsitektur dapat dipertanggungjawabkan.
Praktik AE akan efektif ketika perusahaan memiliki mekanisme pengawasan, kontrol terhadap kepatuhan, dan forum resmi dalam mengambil keputusan.

\subsection{Komponen Utama Tata Kelola Arsitektur \textit{Enterprise}}
Menurut TOGAF, tata kelola AE memiliki komponen-komponen utama sebagai berikut \autocite{TOGAF_Standard_10th}:
\begin{enumerate}
  \item Struktur Tata Kelola \\
  Struktur tata kelola menjelaskan tentang peran formal seperti \textit{architecture board}.
  TOGAF menjelaskan bahwa \textit{architecture board} merupakan komponen penting dalam tata kelola yang berfungsi dalam menjaga konsistensi dan pengawasan arsitektur pada perusahaan. \\

  \item Proses Tata Kelola \\
  Menurut TOGAF, proses tata kelola berfungsi untuk mengidentifikasi, mengelola, dan menyebarluaskan informasi yang berkaitan dengan arsitektur.
  Proses tata kelola yang dijelaskan dalam TOGAF meliputi: 
  \begin{enumerate}[a.]
    \item \textit{Policy Management and Take-On} \\
    Proses ini mengatur perubahan arsitektur dan artefak yang baru dipublikasikan.

    \item \textit{Compliance} \\
    TOGAF menyatakan bahwa kepatuhan terhadap standar harus dinilai secara konsisten untuk memastikan kesesuaian arsitektur.

    \item \textit{Dispensation (Waiver)} \\
    TOGAF menyediakan mekanisme \textit{waiver} jika desain atau teknologi tidak dapat memenuhi standar.

    \item \textit{Monitoring and Reporting} \\
    Proses ini memastikan arsitektur dipantau berdasarkan kriteria yang telah disepakati.

    \item \textit{Business Control and Environment Management} \\
    Proses ini terkait dengan pengelolaan dokumen, arsitektur, dan pengelolaan informasi. \\
  \end{enumerate}

  \textcite{foorthuis2016theory} juga menekankan bahwa proses tata kelola berperan penting dalam mencegah ketidakkonsistenan arsitektur.\\

  \item Artefak Arsitektur \\
  TOGAF menegaskan bahwa semua artefak arsitektur harus dikelola melalui proses tata kelola.
  Artefak ini berfungsi sebagai pedoman bagi seluruh tim dalam mengembangkan solusi Teknologi Informasi (TI) yang konsisten.\\
\end{enumerate}

\subsection{Implementasi Tata Kelola Arsitektur \textit{Enterprise}}
Menurut TOGAF, implementasi tata kelola diatur dalam fase G dari \textit{Architecture Development Method} (ADM).
Pada bagian ini memastikan bahwa proyek implementasi mengikuti arsitektur yang telah disetujui dan mematuhi standar perusahaan \autocite{TOGAF_Standard_10th}.
Langkah-langkah dalam fase G adalah sebagai berikut:
\begin{enumerate}
  \item Konfirmasi Ruang Lingkup dan Prioritas untuk \textit{Deployment} \\
  Pada tahap awal, ruang lingkup implementasi dan prioritas dikonfirmasi kembali seperti meninjau \textit{roadmap}, dependensi, dan analisis \textit{gap} arsitektur saat ini dan arsitektur target. \\

  \item Mengidentifikasi Sumber Daya \\
  Perusahaan perlu mengidentifikasi sumber daya dan kompetensi yang diperlukan untuk implementasi.
  TOGAF menekankan perlu kesiapan tim proyek, kesesuaian metode, dan terjalinnya komunikasi antar unit.\\

  \item Memandu Pengembangan \textit{Deployment} \\
  Pada proses ini, arsitek memberikan arahan selama implementasi, termasuk standar, batasan teknis, dan model integrasi.\\

  \item Meninjau Kepatuhan Arsitektur \\
  Selama implementasi, peninjauan kepatuhan dilakukan untuk memastikan desain dan hasil sesuai dengan arsitektur yang telah ditetapkan.\\

  \item Implementasi Proses Bisnis dan Teknologi Informasi \\
  Solusi yang dibangun akan dijalankan pada lingkungan operasional yang akan diikuti dokumentasi dan \textit{updating baseline architecture} di repositori perusahaan.\\

  \item Meninjau \textit{Post Implementation} dan \textit{Close Implementation} \\
  Setelah sistem diluncurkan, TOGAF mengharuskan tinjauan untuk menilai pencapaian tujuan arsitektur dan memperbarui artefak arsitektur.
  Setelah itu proyek ditutup secara formal.
\end{enumerate}

\subsection{Tantangan Penerapan Tata Kelola Arsitektur \textit{Enterprise}}
Menurut \textcite{korhonen2009ea}, tantangan penerapan tata kelola Arsitektur \textit{Enterprise} (AE) antara lain sebagai berikut:
\begin{enumerate}
  \item Konsep tata kelola AE belum didefinisikan secara memadai yang menyulitkan implementasi konsisten dan terarah.
  \item Minimnya keterlibatan sisi bisnis sehingga potensi AE di perusahaan tidak terealisasi.
  \item Proses manajemen yang digunakan dalam tata kelola Teknologi Informasi (TI) tidak memadai untuk tata kelola AE yang bersifat holistik dan strategis.
  \item Kurangnya badan tata kelola perantara \textit{(intermediating governance body)} antara \textit{Chief Enterprise Architect} dan Eksekutif Bisnis.\\
\end{enumerate}

Berdasarkan penjelasan-penjelasan sebelumnya, dapat disimpulkan bahwa tata kelola AE menjadi dasar untuk mencapai keberhasilan implementasi arsitektur perusahaan.
Pemahaman terhadap tata kelola ini, menjadi landasan penilaian tingkat kematangan arsitektur \textit{enterprise}, karena tata kelola merupakan salah satu aspek dalam mengukur sejauh mana arsitektur diterapkan pada perusahaan.

\section{\textit{Enterprise Architecture Maturity Model}}
\subsection{Pengertian dan Tujuan \textit{Enterprise Architecture Maturity Model}}
Menurut \textcite{jager2023getting}, \textit{Enterprise Architecture Maturity Model} (EAMM) adalah kerangka untuk menilai sejauh mana penerapan Arsitektur \textit{Enterprise} (AE) dalam perusahaan telah berjalan secara efektif.
Model ini membantu perusahaan memahami posisi mereka dalam perjalanan pengembangan arsitektur perusahaan dan memberikan panduan untuk meningkatkan kualitas tata kelola dan implementasi AE secara bertahap. 
Tujuan utama dari EAMM adalah alat untuk mengevaluasi kemampuan organisasi dalam mengelola proses arsitektur dan mengidentifikasi area yang perlu ditingkatkan.

\subsection{\textit{Enterprise Architecture Maturity Model} Menurut TOGAF}
TOGAF menyediakan pendekatan \textit{maturity} yang memanfaatkan \textit{Architecture Maturity Models} (AMM) sebagai alat evaluasi dalam fase \textit{Architecture Capability Framework}.
TOFAF mengacu pada prinsip bahwa \textit{maturity} berkembang secara bertahap \autocite{TOGAF_Standard_10th}.
Pada Tabel \ref{tbl:EAMM} menunjukkan tingkat-tingkat \textit{maturity} AE yang umum digunakan dengan referensi TOGAF.

\begin{landscape}
\begin{longtable}{@{\extracolsep{\fill}}
    p{2.5cm}
    p{2.3cm}
    p{2.3cm}
    p{2.3cm}
    p{2.3cm}
    p{2.3cm}}
\caption{\textit{Enterprise Architecture Maturity Model}}
\label{tbl:EAMM} \\
\toprule
\textbf{Dimensi} &
\textbf{Level 1 (\textit{Initial})} &
\textbf{Level 2 (\textit{Development})} &
\textbf{Level 3 (\textit{Defined})} &
\textbf{Level 4 (\textit{Managed})} &
\textbf{Level 5 (\textit{Measured})} \\
\midrule
\endfirsthead

\caption{\textit{Enterprise Architecture Maturity Model} (lanjutan)} \\
\toprule
\textbf{Dimensi} &
\textbf{Level 1 (\textit{Initial})} &
\textbf{Level 2 (\textit{Development})} &
\textbf{Level 3 (\textit{Defined})} &
\textbf{Level 4 (\textit{Managed})} &
\textbf{Level 5 (\textit{Measured})} \\
\midrule
\endhead

\midrule
\multicolumn{6}{r}{\textit{Bersambung ke halaman berikutnya}} \\
\endfoot

\bottomrule
\endlastfoot

% ========== ROWS ==========

\textbf{\textit{Architecture Process}} &
Proses arsitektur masih \textit{ad-hoc}, tidak konsisten, dan bergantung pada individu. &
Proses dasar telah terdokumentasi dan peran mulai ditetapkan. &
Proses arsitektur terdefinisi, dikomunikasikan, dan konsisten; Sudah ada analisis \textit{gap} dan rencana migrasi. &
Proses menjadi budaya organisasi; kualitas proses diukur. &
Proses terus diperbaiki secara berkelanjutan menggunakan metrik. \\

\textbf{\textit{Architecture Development}} &
Dokumentasi dan standar masih sporadis dan tidak terpadu. &
TRM dan \textit{standards profile} lengkap; \textit{gap analysis} dan rencana migrasi diterapkan di seluruh \textit{domain}. &
TRM dan \textit{standards profile} lengkap; analisis kesenjangan dan rencana migrasi selesai. &
Dokumentasi diperbarui rutin; seluruh domain mengikuti standar formal. &
Ada proses standar dan mekanisme pengecualian untuk meningkatkan kualitas pengembangan. \\

\textbf{\textit{Business Alignment}} &
Hubungan dengan strategi bisnis minim dan tidak terdokumentasi. &
Keterkaitan EA dan strategi bisnis dijelaskan eksplisit. &
EA terintegrasi dengan perencanaan investasi dan pengendalian proyek. &
Keputusan investasi diperbarui berdasarkan masukan EA; \textit{business drivers} ditinjau berkala. &
Metrik EA mengoptimalkan hubungan bisnis; unit bisnis aktif dalam perbaikan berkelanjutan. \\

\textbf{\textit{Organization}} &
Manajemen belum terlibat; partisipasi unit rendah. &
Manajemen mulai menyadari pentingnya EA; beberapa unit terlibat. &
Manajemen mendukung penuh; seluruh unit terlibat secara berkelanjutan dan kolaboratif. &
Manajemen meninjau langsung EA; seluruh unit terlibat konsisten. &
Semua unit memberi umpan balik rutin; manajemen mendorong peningkatan berkelanjutan. \\

\textbf{\textit{Architecture Governance}} &
Tidak ada tata kelola; kepatuhan sangat rendah. &
Tata kelola mulai diterapkan pada beberapa standar. &
Tata kelola terdokumentasi dan mencakup mayoritas investasi TI; Terdapat mekanisme pengecualian (\textit{waiver}). &
Tata kelola menyeluruh; pengelolaan deviasi terintegrasi dengan EA. &
Tata kelola matang; mekanisme waiver mendukung peningkatan berkelanjutan; tidak ada investasi TI tak terencana. \\

\textbf{\textit{Architecture Communication}} &
Dokumentasi ada, tetapi komunikasi terbatas dan lokal. &
Media komunikasi EA diperbarui dan digunakan menyimpan dokumen arsitektur. &
Dokumentasi arsitektur diperbarui periodik dan dikomunikasikan ke TI dan bisnis secara rutin. &
Dokumentasi dan proses komunikasi mengikuti perkembangan terbaru. &
Dokumen EA digunakan oleh seluruh pengambil keputusan; komunikasi menjadi mekanisme utama EA. \\

\end{longtable}

\end{landscape}

\subsection{Keterkaitan \textit{Enterprise Architecture Maturity Model} dengan Tata Kelola Arsitektur \textit{Enterprise}}
\textit{Enterprise Architecture Maturity Model} dan Tata Kelola Arsitektur \textit{Enterprise} (AE) memiliki hubungan yang saling berkaitan.
Tata kelola yang kuat merupakan prasyarat untuk mencapai tingkat kematangan yang lebih tinggi.
Namun, praktik AE tidak dapat berkembang secara stabil jika proses dan struktur peran belum terdefinisi.
TOGAF menekankan bahwa tata kelola merupakan komponen yang fundamental dari \textit{Architecture Capability Framework}, yang juga menjadi dasar penilaian \textit{maturity} \autocite{TOGAF_Standard_10th}.
\textcite{foorthuis2016theory} menyatakan bahwa proses tata kelola yang baik akan meningkatkan kualitas implementasi AE yang nantinya akan menaikkan \textit{maturity} secara keseluruhan.

\section{SAP LeanIX sebagai \textit{Enabler} Tata Kelola Arsitektur \textit{Enterprise}}
SAP LeanIX berperan dalam mengoperasikan Tata Kelola Arsitektur \textit{Enterprise} (AE) di berbagai perusahaan.
LeanIX menyediakan repositori terpusat untuk dokumentasi arsitektur yang dilakukan secara konsisten.
Melalui struktur \textit{fact sheet}, LeanIX menjaga standar dokumentasi lintas tim, mulai dari bisnis, aplikasi, data, dan infrastruktur.
Fitur ini mendukung proses inti tata kelola seperti dokumentasi dan peninjauan arsitektur.
Selain itu, keberadaaan repositori tunggal ini mencegah duplikasi sistem \autocite{LeanixEAGov}.

\section{Studi dan Penelitian Terkait}
\subsection{Penilaian Arsitektur \textit{Enterprise} di Pemerintahan}
Penelitian yang dilakukan oleh \textcite{hanafi2023implementation} menguji hubungan antara kapabilitas Arsitektur \textit{Enterprise} (AE) dan tata kelola AE terhadap kinerja organisasi.
Penelitian ini berfokus pada pemerintah daerah Provinsi Jawa Barat dalam konteks implementasi Sistem Pemerintahan Berbasis Elektronik (SPBE).
Hasil analisis menunjukkan bahwa kapabilitas AE dan tata kelola AE memberikan pengaruh positif dan signifikan terhadap kinerja organisasi.
Analisis ini memperkuat teori bahwa tata kelola AE merupakan aspek untuk meningkatkan kinerja organisasi.

Meskipun demikian, fokus utama penelitian belum menguraikan proses tata kelola AE dan belum mengusulkan perbaikan tata kelola AE.
Selain itu, penelitian ini belum fokus ke perancangan solusi operasional dengan \textit{platform Enterprise Architecture Management} (EAM) tertentu.

\subsection{\textit{Enterprise Architecture Governance of Excellence}}
Penelitian oleh \textcite{hillmann2024enterprise} berfokus kepada pengembangan tata kelola AE dalam konteks lingkungan federasi, yaitu organisasi besar yang terdiri dari unit-unit bisnis atau segmen yang beroperasi secara semi-otonom.
Penelitian ini berfokus kepada komponen-komponen yang harus ada dalam tata kelola AE, seperti struktur peran dan tanggung jawab, kebijakan formal, pengambilan keputusan arsitektur, mekanisme kontrol dan pengawasan, portofolio AE, 
dan penggunaan repositori AE untuk mengelola artefak dan keputusan arsitektur.
Penelitian ini mennyajikan gambaran mengenai bagaimana komponen-komponen tersebut saling berkaitan dan membentuk sistem tata kelola yang konsisten.

Namun, penelitian ini masih dominan bersifat konseptual dengan konteks implementasi yang spesifik, sehingga belum dikaitkan secara eksplisit dengan \textit{Enterprise Architecture Maturity Model} dan belum dievaluasi pada industri.

\section{Kerangka Konseptual}
Arsitektur \textit{Enterprise} (AE), Tata Kelola AE, dan \textit{Enterprise Architecture Maturity Model} merupakan tiga komponen yang saling berkaitan erat dalam pengelolaan arsitektur perusahaan.
\begin{enumerate}
  \item Arsitektur \textit{Enterprise} Sebagai Fondasi Integrasi Strategi dan Teknologi \\
  AE berfungsi sebagai kerangka yang mendefinisikan hubungan antara bisnis dan teknologi.
  AE akan membantu perusahaan mengatur elemen bisnis, aplikasi, data, dan infrastruktur agar dapat menghasilkan keputusan yang strategis \autocite{TOGAF_Standard_10th}. 
  Selain itu, \textcite{bernard2012introduction} menekankan bahwa AE akan memberikan panduan struktural untuk meningkatkan integrasi proses bisnis dan efisiensi teknologi.
  \textcite{ross2006enterprise} juga menjelaskan bahwa arsitektur yang terdokumentasi akan membantu perusahaan menurunkan kompleksitas dan mempercepat pengambilan keputusan. 
  Dengan demikian, AE berperan dalam menyusun artefak arsitektur yang menggambarkan kondisi saat ini dan target perusahaan.\\

  \item Tata Kelola AE Sebagai Pengendali dan Penjamin Konsisten\\
  Tata kelola AE bertujuan untuk memastikan bahwa implementasi solusi atau teknologi mematuhi standar dan prinsip arsitektur \autocite{TOGAF_Standard_10th}.
  \textcite{foorthuis2016theory} menegaskan bahwa proses tata kelola seperti \textit{review, compliance} dan standarisasi berpengaruh terhadap keberhasilan AE.
  Selain itu, \textcite{tamm2011does} menjelaskan bahwa AE tidak muncul hanya karena memiliki kapabilitas yang bagus, akan tetapi AE akan terealisasi ketika memiliki tata kelola secara efektif dengan menerapkan aset kapabilitas tersebut 
  dalam setiap mengambil suatu keputusan.\\

  \item \textit{Enterprise Architecture Maturity Model} Sebagai Alat Ukur Kapabilitas Arsitektur \\
  \textit{Enterprise Architecture Maturity Model} merupakan kerangka untuk menilai sejauh mana kapabilitas AE dan tata kelola diterapkan \autocite{jager2023getting}.\\
\end{enumerate}

Dari tiga komponen yang telah dijelaskan sebelumnya, dapat disimpulkan bahwa adanya keterkaitan fungsional yang erat antar komponen tersebut.
Gambar \ref{gambar:corr} menunjukkan hubungan konseptual antara Arsitektur \textit{Enterprise} (AE), Tata Kelola AE, dan \textit{Enterprise Architecture Maturity Model}.
\begin{figure}[h] % pilihan opsi yang disarankan: t = top, b = bottom, h = here
  \centering
  \captionsetup{justification=centering}
      \includegraphics[width=1\textwidth]{image/konseptual2.png}
  \caption{Hubungan Konseptual Antara AE, Tata Kelola, dan EAMM}
  \label{gambar:corr}
\end{figure}