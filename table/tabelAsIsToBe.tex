\begin{longtable}{@{\extracolsep{\fill}}
    >{\raggedright\arraybackslash}p{3cm}
    >{\raggedright\arraybackslash}p{4.5cm}
    >{\raggedright\arraybackslash}p{4.5cm}
}
\caption{Perbandingan Mekanisme EA Impact Analysis (AS-IS vs TO-BE)}
\label{tbl:perbandinganKondisi} \\
\toprule
\textbf{Aspek} &
\textbf{AS-IS} &
\textbf{TO-BE} \\
\midrule
\endfirsthead

\caption[]{Perbandingan Mekanisme EA Impact Analysis (AS-IS vs TO-BE) (lanjutan)} \\
\toprule
\textbf{Aspek} &
\textbf{AS-IS} &
\textbf{TO-BE} \\
\midrule
\endhead

\midrule
\multicolumn{3}{r}{\textit{Bersambung ke halaman berikutnya}} \\
\endfoot

\bottomrule
\endlastfoot

% ====================== ROWS ======================

Domain yang Terlibat &
Domain yang dianalisis yaitu \textit{Business, Application,} dan \textit{Data}.  
Sementara itu, untuk \textit{Technology} dan \textit{Security} tidak dianalisis. &
Seluruh domain dianalisis sesuai TOGAF. \\

Analisis \textit{Technology Architecture} &
Tidak disertakan dalam dokumen URS dan harus melihat LeanIX secara manual. &
Menjadi aktivitas wajib pada proses \textit{EA Impact Analysis}. \\

Analisis \textit{Security Architecture} &
Tidak disertakan dalam dokumen URS dan harus melihat LeanIX secara manual. &
\textit{Security} menjadi \textit{architecture concern} yang dianalisis pada proses \textit{EA Impact Analysis}. \\

Konsolidasi Hasil Analisis &
Tidak ada dokumen yang menyatukan seluruh hasil analisis &
Menghasilkan satu dokumen sebagai ringkasan dan rangkuman untuk keputusan hasil proses \textit{EA Impact Analysis}. \\

Peran Domain \textit{Architect} &
Peran masing-masing domain tidak didefinisikan dengan jelas. &
Domain \textit{architect} terdefinisi jelas. \\

Kelengkapan dan Akurasi &
Struktur dan proses masih belum jelas. &
Tervalidasi melalui struktur TOGAF ADM dan \textit{Content Metamodel} sehingga lebih terstruktur. \\

\end{longtable}
