\begin{longtable}{@{\extracolsep{\fill}}
    >{\raggedright\arraybackslash}p{3.0cm}
    >{\raggedright\arraybackslash}p{4.5cm}
    >{\raggedright\arraybackslash}p{4.5cm}
}
\caption{Analisis Risiko dan Mitigasi Penyusunan Mekanisme \textit{EA Impact Analysis}}
\label{tbl:resiko-mitigasi} \\
\toprule
\textbf{Risiko} &
\textbf{Deskripsi Risiko} &
\textbf{Mitigasi} \\
\midrule
\endfirsthead

\caption[]{Analisis Risiko dan Mitigasi Penyusunan Mekanisme \textit{EA Impact Analysis} (lanjutan)} \\
\toprule
\textbf{Risiko} &
\textbf{Deskripsi Risiko} &
\textbf{Mitigasi} \\
\midrule
\endhead

\midrule
\multicolumn{3}{r}{\textit{Bersambung ke halaman berikutnya}} \\
\endfoot

\bottomrule
\endlastfoot

% ====================== ROWS ======================

Ketidaksesuaian dengan TOGAF &
Mekanisme yang dirancang mungkin tidak sepenuhnya sesuai dengan TOGAF. &
Menggunakan dokumentasi resmi TOGAF sebagai acuan dan membuat \textit{checklist} verifikasi berbasis TOGAF. \\

Perbedaan interpretasi \textit{domain} &
Risiko salah menafsirkan arsitektur \textit{Business, Application, Data, Technology}, dan \textit{Security} karena TOGAF memiliki cakupan yang luas. &
Menyusun definisi setiap elemen berdasarkan referensi TOGAF. \\

Keterbatasan akses artefak LeanIX &
Tidak semua artefak di LeanIX dapat diakses sehingga simulasi tidak mencerminkan kondisi aktual. &
Menggunakan URS dan artefak yang tersedia dan meminta akses informasi kepada \textit{Enterprise Architect} penggunaan artefak. \\

Waktu pengerjaan tidak mencukupi &
Penyusunan mekanisme, revisi, dan evaluasi membutuhkan waktu yang lebih panjang dari estimasi. &
Membuat jadwal internal terstruktur; melakukan revisi secara bertahap dan tidak menumpuk; memprioritaskan elemen mandatory menurut TOGAF, sementara elemen opsional disesuaikan dengan waktu. \\

Penolakan konsep mekanisme &
Mekanisme dianggap terlalu kompleks atau sulit dijelaskan. &
Menyusun mekanisme dengan detail yang proporsional; menyediakan diagram, ilustrasi, dan contoh pengisian; dan memastikan kesesuaian konsep melalui klarifikasi dengan \textit{Enterprise Architect}. \\

Mekanisme tidak berjalan lancar saat simulasi &
Simulasi manual dapat menemukan langkah yang ambigu, terlalu kompleks, atau \textit{template} sulit diisi. &
Melakukan dokumentasikan kendala; memperbaiki mekanisme dan \textit{template} berdasarkan temuan; dan melakukan simulasi ulang. \\

\end{longtable}
