\begin{longtable}{@{\extracolsep{\fill}}
    >{\raggedright\arraybackslash}p{2.5cm}
    >{\raggedright\arraybackslash}p{4cm}
    >{\raggedright\arraybackslash}p{4.5cm}
    >{\raggedright\arraybackslash}p{3cm}
}
\caption{Penilaian Dimensi EA pada Level 3 \textit{Defined} Berdasarkan TOGAF dan Kondisi Aktual Paragon Corp}
\label{tbl:Level3EAMM} \\
\toprule
\textbf{Dimensi} &
\textbf{Indikator Level 3 (\textit{Defined}) (TOGAF)} &
\textbf{Kondisi Aktual Paragon Corp (Wawancara)} &
\textbf{Penjelasan Level Capaian} \\
\midrule
\endfirsthead

\caption[]{Penilaian Dimensi EA pada Level 3 \textit{Defined} (lanjutan)} \\
\toprule
\textbf{Dimensi} &
\textbf{Indikator Level 3 (\textit{Defined}) (TOGAF)} &
\textbf{Kondisi Aktual Paragon Corp (Wawancara)} &
\textbf{Penjelasan Level Capaian} \\
\midrule
\endhead

\midrule
\multicolumn{4}{r}{\textit{Bersambung ke halaman berikutnya}} \\
\endfoot

\bottomrule
\endlastfoot

% ====================== ROWS ======================

\textbf{\textit{Architecture Process}} &
Proses arsitektur terdefinisi, dikomunikasikan, dan konsisten; sudah ada analisis \textit{gap} dan rencana migrasi. &
Prosedur utama sudah dibakukan dan disosialisasikan melalui dokumen internal dan SAP LeanIX; proses diterapkan konsisten di unit kunci, tetapi belum menyeluruh pada seluruh organisasi. &
2–3: Proses sudah jelas dan terdokumentasi; belum sepenuhnya jadi kebiasaan lintas unit. \\

\textbf{\textit{Architecture Development}} &
TRM dan \textit{standards profile} lengkap; \textit{gap analysis} dan rencana migrasi diterapkan di seluruh \textit{domain}. &
\textit{Domain} arsitektur bisnis dan aplikasi diperbarui rutin; artefak data, infra, dan keamanan baru dilengkapi setelah pelaksanaan, belum proaktif sejak awal. &
2–3: Bisnis/aplikasi matang; data/teknologi perlu penguatan untuk capai Level 3 penuh. \\

\textbf{\textit{Business Alignment}} &
EA terintegrasi dengan perencanaan investasi dan pengendalian proyek. &
\textit{Business Architect} aktif selaraskan kebutuhan IT dan bisnis namun investasi belum sepenuhnya berbasis analisis EA formal. &
2–3: Integrasi formal baru diterapkan di sebagian proses, belum menyeluruh semua inisiatif. \\

\textbf{\textit{Organization}} &
Manajemen mendukung penuh; seluruh unit terlibat secara berkelanjutan dan kolaboratif. &
Struktur peran sudah ditetapkan; kolaborasi sudah berjalan dalam beberapa forum  tetapi partisipasi lintas unit masih situasional. &
2: Dukungan dan kolaborasi ada, tetapi belum konsisten dan belum semua unit terlibat rutin. \\

\textbf{\textit{Architecture Governance}} &
Tata kelola terdokumentasi dan mencakup mayoritas investasi TI; Terdapat mekanisme pengecualian (\textit{waiver}). &
Tanggung jawab sudah dibagi dan adanya monitoring artefak; forum \textit{Architecture Review Board} (ARB) dan \textit{waivr} belum berjalan; Kepatuhan masih berdasarkan himbauan, bukan mekanisme wajib. &
2: Tata kelola sudah diarahkan formal tetapi belum sepenuhnya konsisten; mekanisme \textit{waiver} belum matang. \\

\textbf{\textit{Architecture Communication}} &
Dokumentasi arsitektur diperbarui periodik dan dikomunikasikan ke TI dan bisnis secara rutin. &
SAP LeanIX sudah menjadi sumber referensi utama; update artefak mulai periodik.  
Namun penyebaran dokumen masih melalui pelaporan teknis dan belum mencakup seluruh unit. &
Level 2–3: Dokumentasi periodik, tetapi komunikasi belum merata ke semua pemangku kepentingan. \\

\end{longtable}
