\begin{table}[h]
    \centering
    \begin{tabular}{|p{3cm}|p{4cm}|p{4cm}|}
    \hline
    \textbf{Aspek} & \textbf{Kelebihan} & \textbf{Kekurangan} \\
    \hline
    Kejelasan Proses &
    Memberikan alur \textit{EA impact analysis} yang jelas dan terstruktur karena mengikuti praktik TOGAF. &
    Penyusunan SOP membutuhkan waktu, diskusi lintas unit, serta validasi berulang agar disepakati semua pihak. \\
    \hline
    Standarisasi &
    Menghasilkan keseragaman proses sehingga seluruh tim bekerja dengan pedoman yang sama. &
    Standarisasi dapat dirasakan terlalu formal oleh beberapa tim yang belum terbiasa bekerja dengan prosedur terstruktur. \\
    \hline
    Integrasi LeanIX &
    LeanIX mendukung dokumentasi sehingga mempermudah pencatatan dan pelacakan hasil analisis. &
    Perlu penyesuaian konfigurasi LeanIX agar sepenuhnya selaras dengan SOP baru yang disusun. \\
    \hline
    \end{tabular}
    \caption{Kelebihan dan Kekurangan Alternatif 1: SOP \textit{EA Impact Analysis} berbasis TOGAF dan LeanIX}
    \label{tbl:alt1}
    \end{table}
    