% ==========================================
% BAB V RENCANA SELANJUTNYA
% ==========================================
\chapter{RENCANA SELANJUTNYA}
\label{chap:rencana-selanjutnya}
\section{\textit{Timeline} Implementasi}
Tahapan dalam implementasi Tugas Akhir ini dilakukan selama 14 minggu. 
\textit{timeline} disusun dengan mempertimbangkan aktivitas setiap tahap dan kesesuaian dengan jadwal kalender akademik.
Tabel \ref{tbl:timeline} menunjukkan detail rencana \textit{implementasi} selama 14 minggu. \\
\begin{longtable}{@{\extracolsep{\fill}}
    >{\raggedright\arraybackslash}p{2.0cm}
    >{\raggedright\arraybackslash}p{5.0cm}
    >{\raggedright\arraybackslash}p{5.0cm}
}
\caption{Timeline Rencana Implementasi Mekanisme \textit{EA Impact Analysis}} 
\label{tbl:timeline} \\
\toprule
\textbf{Periode} &
\textbf{Aktivitas} &
\textbf{Output} \\
\midrule
\endfirsthead

\caption[]{Timeline Rencana Implementasi Mekanisme \textit{EA Impact Analysis} (lanjutan)} \\
\toprule
\textbf{Periode} &
\textbf{Aktivitas} &
\textbf{Output} \\
\midrule
\endhead

\midrule
\multicolumn{3}{r}{\textit{Bersambung ke halaman berikutnya}} \\
\endfoot

\bottomrule
\endlastfoot

% ====================== ROWS ======================

Minggu 1-3 &
Studi literatur mendalam mengenai TOGAF ADM, \textit{Content Metamodel}, dan \textit{Impact Assessment} untuk memastikan mekanisme selaras dengan standar TOGAF. &
Ringkasan referensi TOGAF relevan untuk \textit{EA Impact Analysis}. \\

Minggu 4-6 &
Perancangan mekanisme \textit{EA Impact Analysis} \textit{To-Be} meliputi alur proses, peran, \textit{input} dan \textit{output}, serta elemen analisis per domain. &
Rancangan mekanisme \textit{EA Impact Analysis To-Be}. \\

Minggu 7-9 &
Penyusunan \textit{template} analisis per domain sesuai TOGAF. &
Rancangan \textit{template} domain. \\

Minggu 10-11 &
Konsultasi dan \textit{expert review} dengan \textit{Enterprise Architect} untuk mendapatkan masukan terhadap mekanisme dan \textit{template}. &
Daftar masukan terkait mekanisme dan \textit{template}. \\

Minggu 12 &
Revisi mekanisme dan \textit{template} berdasarkan masukan hasil \textit{expert review}. &
Final rancangan mekanisme dan \textit{template} \textit{EA Impact Analysis}. \\

Minggu 13 &
Penyusunan evaluasi berupa \textit{checklist} verifikasi berbasis TOGAF dan daftar pertanyaan validasi. &
Hasil evaluasi (verifikasi dan validasi). \\

Minggu 14 &
Simulasi proses menggunakan satu contoh URS untuk menguji kelayakan mekanisme \textit{To-Be}. &
Hasil simulasi dan catatan perbaikan. \\

Minggu 15-16 &
Finalisasi laporan dan penyusunan lampiran mekanisme, \textit{template}, dan hasil evaluasi. &
Hasil laporan akhir dan lampiran lengkap. \\

\end{longtable}


\section{Desain Pengujian dan Evaluasi}
\subsection{Verifikasi Kesesuaian terhadap Standar TOGAF}
Tahapan verifikasi dilakukan untuk memastikan mekanisme yang dibuat mengikuti konsep TOGAF.
Langkah verifikasi dilakukan dengan menyusun \textit{checklist} yang berisi aspek arsitektur yang dianalisis pada setiap \textit{domain}, 
lalu memeriksa mekanisme yang digunakan telah memuat seluruh aspek tersebut. 
Keluaran tahap ini berupa \textit{checklist} verifikasi dan catatan koreksi jika ditemukan ketidaksesuaian.

\subsection{Validasi Konseptual}
Tahapan validasi dilakukan untuk melihat apakah mekanisme yang diracang sudah cukup jelas dan sesuai dengan kebutuhan arsitektur.
Penilaian dilakukan dengan meninjau kembali alur mekanisme, kelengkapan aspek analisis, dan kejelasan \textit{template} yang digunakan.
Selanjutnya, konsep dan struktur mekanisme dikonsultasikan kepada Tim \textit{Enterprise Architecture} sebagai pihak yang memahami proses arsitektur di perusahaan.
Keluaran tahap ini berupa catatan evaluasi dan revisi yang diperlukan.

\subsection{Simulasi Mekanisme}
Simulasi dilakukan untuk melihat bagaimana mekanisme bekerja apabila diterapkan pada contoh kasus sederhana.
Satu dokumen URS digunakan sebagai bahan uji, kemudian seluruh langkah analisis dijalankan secara berurutan.
Dari proses ini, dicatat apakah ada bagian yang kurang jelas, terlalu rumit, atau tidak sesuai dengan alur yang direncanakan.
Hasil pengamatan digunakan untuk memperbaiki dan menyempurnakan mekanisme sebelum memasuki tahap akhir.
Keluaran tahap ini berupa hasil simulasi serta daftar perbaikan yang diperlukan.

\section{Analisis Risiko dan Mitigasi}
Pada pelaksanaan Tugas Akhir, terdapat beberapa potensi risiko yang mungkin dapat memengaruhi keberhasilan implementasi.
Tabel \ref{tbl:resiko-mitigasi} menunjukkan risiko-risiko yang mungkin terjadi dalam pengerjaan Tugas Akhir ini.
\begin{longtable}{@{\extracolsep{\fill}}
    >{\raggedright\arraybackslash}p{3.0cm}
    >{\raggedright\arraybackslash}p{4.5cm}
    >{\raggedright\arraybackslash}p{4.5cm}
}
\caption{Analisis Risiko dan Mitigasi Penyusunan Mekanisme \textit{EA Impact Analysis}}
\label{tbl:resiko-mitigasi} \\
\toprule
\textbf{Risiko} &
\textbf{Deskripsi Risiko} &
\textbf{Mitigasi} \\
\midrule
\endfirsthead

\caption[]{Analisis Risiko dan Mitigasi Penyusunan Mekanisme \textit{EA Impact Analysis} (lanjutan)} \\
\toprule
\textbf{Risiko} &
\textbf{Deskripsi Risiko} &
\textbf{Mitigasi} \\
\midrule
\endhead

\midrule
\multicolumn{3}{r}{\textit{Bersambung ke halaman berikutnya}} \\
\endfoot

\bottomrule
\endlastfoot

% ====================== ROWS ======================

Ketidaksesuaian dengan TOGAF &
Mekanisme yang dirancang mungkin tidak sepenuhnya sesuai dengan TOGAF. &
Menggunakan dokumentasi resmi TOGAF sebagai acuan dan membuat \textit{checklist} verifikasi berbasis TOGAF. \\

Perbedaan interpretasi \textit{domain} &
Risiko salah menafsirkan arsitektur \textit{Business, Application, Data, Technology}, dan \textit{Security} karena TOGAF memiliki cakupan yang luas. &
Menyusun definisi setiap elemen berdasarkan referensi TOGAF. \\

Keterbatasan akses artefak LeanIX &
Tidak semua artefak di LeanIX dapat diakses sehingga simulasi tidak mencerminkan kondisi aktual. &
Menggunakan URS dan artefak yang tersedia dan meminta akses informasi kepada \textit{Enterprise Architect} penggunaan artefak. \\

Waktu pengerjaan tidak mencukupi &
Penyusunan mekanisme, revisi, dan evaluasi membutuhkan waktu yang lebih panjang dari estimasi. &
Membuat jadwal internal terstruktur; melakukan revisi secara bertahap dan tidak menumpuk; memprioritaskan elemen mandatory menurut TOGAF, sementara elemen opsional disesuaikan dengan waktu. \\

Penolakan konsep mekanisme &
Mekanisme dianggap terlalu kompleks atau sulit dijelaskan. &
Menyusun mekanisme dengan detail yang proporsional; menyediakan diagram, ilustrasi, dan contoh pengisian; dan memastikan kesesuaian konsep melalui klarifikasi dengan \textit{Enterprise Architect}. \\

Mekanisme tidak berjalan lancar saat simulasi &
Simulasi manual dapat menemukan langkah yang ambigu, terlalu kompleks, atau \textit{template} sulit diisi. &
Melakukan dokumentasikan kendala; memperbaiki mekanisme dan \textit{template} berdasarkan temuan; dan melakukan simulasi ulang. \\

\end{longtable}
