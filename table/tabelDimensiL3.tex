\begin{longtable}{@{\extracolsep{\fill}}
    >{\raggedright\arraybackslash}p{3.0cm}
    >{\raggedright\arraybackslash}p{5.0cm}
    >{\raggedright\arraybackslash}p{5.0cm}
}
\caption{Penilaian Dimensi AE pada Level 3 \textit{Defined} Berdasarkan TOGAF dan Kondisi Aktual Paragon Corp}
\label{tbl:Level3EAMM} \\
\toprule
\textbf{Dimensi} &
\textbf{Indikator Level 3 (\textit{Defined}) (TOGAF)} &
\textbf{Kondisi Aktual Paragon Corp (Wawancara)} \\
\midrule
\endfirsthead

\caption[]{Penilaian Dimensi EA pada Level 3 \textit{Defined} (lanjutan)} \\
\toprule
\textbf{Dimensi} &
\textbf{Indikator Level 3 (\textit{Defined}) (TOGAF)} &
\textbf{Kondisi Aktual Paragon Corp (Wawancara)} \\
\midrule
\endhead

\midrule
\multicolumn{3}{r}{\textit{Bersambung ke halaman berikutnya}} \\
\endfoot

\bottomrule
\endlastfoot

% ====================== ROWS ======================

\textbf{\textit{Architecture Process}} &
Proses arsitektur sudah dijelaskan dengan jelas, terdokumentasi, dan disosialisasikan. Selain itu sudah ada \textit{gap analysis} dan rencana migrasi menuju kondisi ideal. &
Proses sudah disosialisasikan melalui dokumen dan SAP LeanIX. Namun, dalam pelaksanaannya, beberapa tim masih menjalankan proses dengan cara yang berbeda-beda. 
\textit{Gap analysis} juga sudah didokumentasikan.\\

\textbf{\textit{Architecture Development}} &    
TRM dan \textit{standards profile} sudah lengkap. Selain itu, \textit{gap analysis} dan rencana migrasi juga sudah selesai dibuat. &
Dokumentasi arsitektur untuk \textit{domain} bisnis dan aplikasi akan dimulai sejak awal perencanaan inisiatif. 
Namun, untuk \textit{domain} arsitektur dan keamanan dimulai beriringan berjalannya suatu inisiatif.\\

\textbf{\textit{Business Alignment}} &
AE sudah menjadi bagian dari proses perencanaan investasi dan pengendalian proyek. &
\textit{Business Architect} aktif menjaga agar kebutuhan TI tetap selaras dengan tujuan bisnis. 
Namun, beberapa keputusan masih berjalan berdasarkan kebutuhan operasional, bukan berdasarkan evaluasi AE secara menyeluruh.\\

\textbf{\textit{Organization}} &
Manajemen memberikan dukungan penuh dan keterlibatan lintas unit berjalan secara berkelanjutan dan kolaboratif. &
Struktur peran sudah ditetapkan dengan jelas. Namun, belum dilakukan secara proaktif tanpa dorongan konteks tertentu. \\

\textbf{\textit{Architecture Governance}} &
Tata kelola sudah terdokumentasi dan mencakup sebagian besar investasi TI dan adanya mekanisme pengecualian resmi (\textit{waiver}). &
Tanggung jawab dalam pengelolaan arsitektur sudah dibagi, dan terdapat proses pemantauan artefak.
Namun, forum \textit{Architecture Review Board} (ARB) belum berjalan secara formal.
Selain itu, kepatuhan terhadap standar masih bersifat himbauan, sehingga belum ada aturan wajib yang harus dipatuhi oleh semua tim. \\

\textbf{\textit{Architecture Communication}} &
Dokumentasi arsitektur diperbarui secara berkala dan dikomunikasikan secara rutin ke tim TI dan bisnis. &
SAP LeanIX sudah digunakan sebagai sumber referensi utama untuk dokumen arsitektur, dan pembaruan dokumen mulai dilakukan secara rutin.\\

\end{longtable}